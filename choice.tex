\newpage
\label{soft_choice}
\section{Dobór elementów oprogramowania autopilota}
{
    \subsection{Wybór właściwego sterownika napędu}
    {
        Napęd który należy dobrać charakteryzuje się następującymi cechami:
        \begin{itemize}
            \item Pozwoli robotowi skręcać w miejscu.
            \item Pozwoli na uzyskanie wystarczającej dokładności podczas obrotu.
        \end{itemize}
        W takim wypadku właściwym wydaje się wybór napędu różnicowego, który pozwoli spełnić powyższe założenia. Taki napęd również uprości budowę robota.
        Aby jak najlepiej oddać jak najwierniej rzeczywistość oraz zastosować napęd różnicowy wybieramy \textbf{skid\_steer\_drive\_controller}. Ta~odmiana sterownika napędu różnicowego zapewniona przez wspomniany pakiet dodatkowo wprowadza możliwość poślizgów kół, co przybliża nas do rzeczywistości.
    }
    \subsection{Pakiety symulujące dane czujniki}
    {
        \label{sensor_pkgs}
        W symulatorze Gazebo oraz środowisku ROS istnieje możliwość symulacji wszystkich niezbędnych czujników. Poniższe pakiety zapewniają możliwość definiowania modelu zakłóceń.
        \begin{itemize}
            \item Pakiet \textbf{libhector\_gazebo\_ros\_gps} służy do symulacji odbiornika GPS, 
            posiada on interfejs do definiowania postaci symulowanych szumów.
            \item Pakiet \textbf{libhector\_gazebo\_ros\_magnetic}  symuluje wskazania magnetometru.
            \item Pakiet \textbf{libgazebo\_ros\_imu} symuluje wskazania IMU, czyli umożliwia agregację pomiarów z akcelerometru oraz żyroskopu. 
        \end{itemize}
    }
    \subsection{Filtr wygładzający odczyty orientacji robota}
    {
        Odnosząc się do analiz w rozdziale \ref{compass_analysis} wywnioskowano, że skorzystanie z filtru Kalmana zapewni wygładzenie danych, w sposób pozwalający na bezproblemowe korzystanie z algorytmów autopilota. Filtr ten zapewnia odpowiednią reaktywność względem danych surowych oraz cechuje się brakiem opóźnień. Reaktywność należy rozumieć jako szybkość nadążania za zmianami sygnału oryginalnego.

        Metodą kalibracji filtru Kalmana jest dobranie odpowiednich macierzy stałych oraz macierzy startowych, gdzie w przypadku filtrów Sawickiego-Golaya średniej ruchomej należy odpowiednio manipulować rozmiarem okna pomiarowego. W przypadku Kalmana wartości kalibracyjne są wielkościami fizycznymi, zatem metoda kalibracji jest intuicyjna. Kalibracja filtru Sawickiego-Golaya jest trudna i nieintuicyjna, ze względu na potrzebę doboru stopnia wielomianu oraz abstrakcyjnych właściwości filtra.

        Jeśli chodzi o złożoność obliczeniową, to filtr Sawickiego-Golaya charakteryzuje się dużą złożonością obliczeniową. Filtr średniej ruchomej posiada wysoką złożoność ze względu na dodatkowe zabiegi opisane w rozdziale \ref{av_filter}. Zastosowany filtr Kalmana ma niską złożoność, gdyż jeden krok składa się tylko z kilku prostych obliczeń. 
        Jeśli chodzi o reaktywność względem danych surowych, to w przypadku filtrów Sawickiego-Golaya jest ona mniejsza względem reszty przetestowanych filtrów. Filtry średniej ruchomej oraz filtr Kalmana ma większą reaktywność. 
        Jeśli chodzi o wzrost jakości wygładzenia względem wzrostu rozmiaru okna pomiarowego, to wzrost ten jest szybszy dla filtrów średniej ruchomej niż Sawickego-Golaya. Dla filtrów Kalmana szerokość okna pomiarowego jest zawsze stała.
        By osiągnąć wymagane wygładzenie, to minimalny rozmiar okna pomiarowego dla filtru Sawickiego-Golaya był większy niż 51 próbek. Dla filtru średniej ruchomej wystarczyło 35 próbek.
        Tylko filtr Kalmana pozwala na fuzję danych z wielu czujników.
        Zatem ostatecznie został wybrany filtr Kalmana.
    }
    \subsection{Filtr wygładzający odczyty pozycji robota}
    {
        Odnosząc się do analiz wykonanych w rozdziale \ref{gps_analysis} wywnioskowano, że w tym przypadku również skorzystanie z filtru Kalmana, ale w postaci macierzowej, będzie odpowiednim rozwiązaniem. Dołożenie pomiarów z odometrii nie wnosi poprawienia jakości odczytów, więc ten etap nie został zastosowany.
    }
}