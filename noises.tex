\newpage
\section{Zakłócenia przy współpracy z czujnikami}
{
    % \large
    \subsection{Ogólnie o zakłóceniach}
    {
        W każdym urządzeniu pomiarowym lub czujniku występują zakłócenia, które zniekształcają otrzymany sygnał. Zakłócenia te wprowadzają szum w nasz pomiar i przez to jego wartość może on przyjąć wartość całkowicie odmienną od oczekiwanej.
        Pomiary obarczone zakłóceniami można wyrazić za pomocą wzoru \ref{noises_eqtn}.

        \begin{equation}\label{noises_eqtn}
            y(t) = x(t) + \epsilon(t)
        \end{equation}
        gdzie,\\
        $y(t)$ -- otrzymany pomiar,\\
        $x(t)$ -- rzeczywista wartość,\\
        $\epsilon(t)$ -- wartość szumu\\

        Istnieją metody do eliminacji szumów. Szumy można zlikwidować za pomocą filtrów, które są opisane w rozdziale \ref{filters}. 
    }
    % \subsection{Zakłócenia odometrii}
    % {
       
    % }
    \label{imu_noises}
    \subsection{Zakłócenia IMU }
    {
        Ogólnie zakłócenia IMU można podzielić na dwie kategorie \cite{imu_noise}:
        \begin{itemize}
            \item stochastyczne,
            \item deterministyczne.
        \end{itemize}

        Deterministyczne są spowodowane głównie wadami produkcyjnymi tych urządzeń. Błędy deterministyczne są zwykle stałe podczas całego czasu pracy urządzenia.
        Błędy stochastyczne~są~błędami o naturze losowej. Można je reprezentować jako zmienne losowe o pewnym rozkładzie losowym.
        Elementy mechaniczne oraz elektryczne w IMU zużywają się również z czasem. Zużycie powoduje narastanie błędów.
        % \subsubsection{Zakłócenia akcelerometru}
        % {
            
        % }
        % \subsubsection{Zakłócenia żyroskopu}
        % {
        % }
    }

    \label{mag_noises}
    \subsection{Zakłócenia magnetometru}
    {
        Magnetometry są głównie osadzane w układach scalonych. Głównym źródłem zakłóceń są pola magnetyczne od cewek silników napędowych, które zakłócają odczyt pola magnetycznego Ziemi. Kolejnym źródłem zakłóceń są szumy wywołane przez prąd przepływający przez obwód \cite{mag_noise}. Prąd ten generuje pole magnetyczne wokół przewodów. Dodatkowo materiały ferromagnetyczne (np.~w~szkielecie robota będącymi stopami żelaza) wprowadzają stały błąd pomiaru.
    }
    \subsection{Zakłócenia czujników}
    {
        Również sam pomiar jest obarczony błędem. Czujnik po zebraniu sygnału analogowego przetwarza go na sygnał cyfrowy. Takie przetworzenie, czyli kwantyzacja zmienia  dane pomiarowe (dokładniej mówiąc, klasyfikuje je do najbliższej pasującej wartości, która może być reprezentowana przez liczbę z odpowiednią rozdzielczością).
    }
    \subsection{Precyzja systemu GPS oraz opis zakłóceń}
    {
        System GPS jest złożony, co oznacza, że w porównaniu do wymienionych już czujników posiada on więcej źródeł błędów i zakłóceń. Jakość odczytów i intensywność pojawiania się błędów jest zależna od:
        \begin{itemize}
            \item liczby widocznych satelitów,
            \item wpływu tzw. rozmycia pozycji (ang. \textit{Dilution of Position}),
            \item błędów sygnału GPS.
        \end{itemize}

        Precyzja wskazań GPS znacząco różni się w zależności od liczby widocznych satelitów \cite{gps_principles}. Im więcej widocznych satelitów, tym lepsze przybliżenie aktualnej pozycji odbiornika. 
        Rozmycie precyzji jest miarą otrzymania rozstawienia satelitów zmniejszającego liczbę błędów wynikających z aktualnej konfiguracji widocznych satelitów na niebie.
        Błędy systemu GPS mogą wynikać z błędów zegara odbiornika, które są spowodowane jego małą rozdzielczością w porównaniu do zegarów atomowych. Tutaj błąd ma raczej naturę wynikającą z kwantyzacji pomiarów. Kolejnym czynnikiem wprowadzającym błędy są szumy pochodzące z fal radiowych o podobnych częstotliwościach. Tę samą klasę błędu można uzyskać z powodu odbijania się fal radiowych o tej samej częstotliwości od przeszkód blisko odbiornika. Zakłócenia mogą być również generowane poprzez celowe zakłócanie sygnału GPS.
    }
    \subsection{Zakłócenia odometrii}
    {
        Odometria charakteryzuje się obecnością błędów systematycznych oraz niesystematycznych. Systematyczne błędy są jednym z największych wyzwań dla precyzyjnej kontroli pozycji. Różne źródła mogą przyczyniać się do ich powstawania, takie jak nierówność średnic kół. Również rozstaw kół, który jest inny od nominalnego oraz niewspółosiowość kół może być przyczyną błędów systematycznych. Dodatkowo, skończona rozdzielczość enkodera również może przyczynić się do powstawania tych błędów. 
        Błędy niesystematyczne mają charakter losowy i powstają na wskutek poślizgów kół oraz nierówności podłoża..
        Obie grupy źródeł zakłóceń sprawiają, że pomiary odometrii należy korygować. 
    }
    \subsection{Model zakłóceń w Gazebo}
    {
        By najwierniej oddać rzeczywistość, Gazebo wprowadza do każdego symulowanego czujnika szum \cite{gazebo_noise}. Parametry tego szumu da się ustawić w pliku konfiguracyjnym. Zwykle typ tego szumu jest szumem o rozkładzie normalnym. Ten sposób symulowania szumów oddaje w wiarygodny sposób prawdziwe zakłócenia. 
        Gęstość prawdopodobieństwa zmiennej $z$ jest dana wzorem \ref{random_eqtn}.

        \begin{equation}\label{random_eqtn}
            \rho_G(z) = \frac{1}{\sigma\sqrt{2\pi}} {e}^{-\frac{(z-\mu)^2}{2{\sigma}^2}}
        \end{equation}
        gdzie,\\
        $z$ -- zmienna losowa,\\
        $\mu$ -- średnia arytmetyczna,\\
        $\sigma$ -- odchylenie standardowe\\

        \newpage
        W Gazebo istnieją dwa rodzaje konfiguracji szumów w pliku konfiguracyjnym \cite{skid_steer_drive_controller}:
        \begin{itemize}
            \item W pliku konfiguracyjnym można podać wartość odchylenia standardowego $\sigma$ oraz wartość średnią $\mu$.
            \item W pliku konfiguracyjnym można podać tylko wartość odchylenia standardowego~$\sigma$. Przykładem tutaj jest IMU, które w pliku konfiguracyjnym przyjmuje tylko wartość odchylenia standardowego \cite{imu_std_dev}.
        \end{itemize}
        
        
    }

    \label{filters}
    \newpage
    \section{Filtry}
    {
    \subsection{Opis działania filtrów}
    {
        Podczas odbierania sygnałów z czujników, zwykle towarzyszy im zniekształcenie spowodowane przez szum \cite{filters}. Sygnał nadający się do użycia można otrzymać wtedy, gdy szum zostanie zredukowany. Szum jest często zmienny w czasie. Filtry są algorytmami lub urządzeniami, które wygładzają surowy sygnał. Nie usuwają one szumu, lecz zmniejszają jego wpływ. 
    }
    \label{avg_filter}
    \subsection{Filtr średniej ruchomej}
    {
        Jest to filtr FIR mający wszystkie współczynniki równe sobie \cite{filters}. Właściwość ta zaprezentowana jest we wzorze \ref{fir_eqtn}, gdzie $L$ jest długością filtra, czyli szerokość okna pomiarowego.
        
        \begin{equation}\label{fir_eqtn}
            h_i=\frac{1}{L}
        \end{equation}
        
        Długość filtra $L$ jest liczbą próbek zebranych wstecz od ostatniego pomiaru ($n$ najnowszych próbek). Filtr średniej ruchomej jest podany wzorem \ref{avg_eqtn}, gdzie $x(i)$ są kolejnymi ostatnimi pomiarami, a $y(n)$ jest otrzymaną przefiltrowaną wartością.
        
        \begin{equation}\label{avg_eqtn}
            y(n) = \frac{1}{L} (x(n) + x(n-1) + ... + x(n - L + 1))
        \end{equation}    
    }
    \label{kalman_one_dim}
    \subsection{Jednowymiarowy filtr Kalmana}
    {
        Filtr Kalmana jest algorytmem szacującym, który jest stosowany do przetwarzania danych ~z~czujników, które generują dane z błędem. Głównym celem filtru Kalmana jest obliczenie najlepszej estymacji dla stanu systemu, na podstawie dostępnych danych pomiarowych i modelu matematycznego opisującego system \cite{kalman_1d}. Filtr Kalmana ma szerokie zastosowanie w wielu dziedzinach, ponieważ jest on skuteczny w radzeniu sobie z problemami związanymi z brakiem danych, zakłóceniami i szumami. Jest szeroko stosowany w robotyce do bezwładnościowego pozycjonowania robota, śledzenia obiektów i nawigacji. Jest on także elastyczny i może być stosowany do różnych rodzajów problemów, takich jak filtracja danych ciągłych lub dyskretnych, z modelami linearnymi lub nieliniowymi.
        Jednowymiarowy filtr Kalmana \cite{kalman_1d} jest filtrem wykorzystującym w swoich obliczeniach:
        \begin{itemize}
            \item stan systemu oraz jego niepewność,
            \item przychodzące pomiary oraz ich niepewności,
            \item estymację nowego stanu oraz niepewność tej estymacji.
        \end{itemize}

        \newpage
        Poniżej przedstawiono algebraiczne równania filtru Kalmana dla układów liniowych. Filtr ten składa się z pięciu równań. Ta forma filtru ma zastosowanie do obliczeń, gdzie estymacji poddawana jest wartość skalarna.
        
        \begin{equation}\label{update_equation}
              \hat{x}_{n,n}=\hat{x}_{n,n-1}+K_n(z_N - \hat{x}_{n,n-1})
        \end{equation}
        \begin{equation}\label{state_est_equtaion}
            \hat{x}_{n+1,n}=\hat{x}_{n,n} + \Delta t \hat{\dot{x}}_{n,n}
        \end{equation}
        \begin{equation}\label{kalman_gain_equation}
            K_n=\frac{p_{n,n-1}}{p_{n,n-1} + r_n}
        \end{equation}
        \begin{equation}\label{uncertain_est_equation}
            p_{n,n}=(1-K_n)p_{n,n-1}
        \end{equation}
        \begin{equation}\label{est_uncertain_est_equation}
            p_{n+1,n}=p_{n,n}+q_n
        \end{equation}
        gdzie,\\
        $\hat{x}_{n,n-1}$ -- poprzednia estymacja stanu,\\
        $\hat{x}_{n,n}$ -- estymacja aktualnego stanu,\\
        $\hat{x}_{n+1,n}$ -- prognoza estymacji stanu(lub po prostu wstępna estymacja stanu),\\
        $p_{n,n-1}$ -- niepewność poprzedniej estymacji stanu,\\
        $p_{n,n}$ -- aktualna niepewność estymacji stanu,\\
        $p_{n+1,n}$ -- prognoza niepewności estymacji stanu,\\
        $\hat{\dot{x}}_{n,n}$ -- pochodna estymacji aktualnego stanu,\\
        $K_n$ -- wzmocnienie Kalmana,\\
        $z_N$ -- pomiar stanu,\\
        $\Delta t$ -- różnica czasu pomiędzy przeskokami okna pomiarowego,\\
        $q_n$ -- szum procesowy,\\
        $r_n$ -- niepewność pomiaru\\

        Poniżej przedstawiono nazwy równań filtru Kalmana w odpowiadającej im kolejności.
        \begin{itemize}
            \item Wzór \ref{update_equation} zwany jest równaniem aktualizacji.
            \item Wzór \ref{state_est_equtaion} nazywa się równaniem prognozy stanu.
            \item Wzór \ref{kalman_gain_equation} nazywa się estymacją wzmocnienia Kalmana.
            \item Wzór \ref{uncertain_est_equation} nazywa się równaniem aktualizacji niepewności estymacji stanu.
            \item Wzór \ref{est_uncertain_est_equation} nazywa się prognozą niepewności estymacji stanu.
        \end{itemize}

        Podczas obliczeń wykorzystuje się niepewność estymacji $p_n$ oraz niepewność pomiaru $r_n$. Niepewność estymacji zmienia się wraz z każdą iteracją filtru, a niepewność pomiaru jest z góry zadana i stała przez cały czas (np. niepewność pomiaru woltomierza). Wielkość $z_N - \hat{x}_{n,n-1}$ jest czasami zwana innowacją.
        Wzmocnienie Kalmana jest najważniejszym elementem i należy je należy rozumieć jako:
        \begin{equation}
            K_n=\frac{Niepewnosc\:estymacji}{Niepewnosc\:estymacji + niepewnosc\:pomiaru}
        \end{equation}

        Wartość wzmocnienia Kalmana zawiera się w przedziale $0<K_n<1$. Jeśli jego wartości są bliskie zeru, to zbieżność niepewności estymacji do zera jest bardzo powolna. Jednak jeśli wzmocnienie Kalmana jest bliskie 1, to niepewność estymacji szybko zbiega do 0.

        Losowy szum procesowy $q_n$ jest wartością fluktuacji wartości prawdziwej (przykładowo prawdziwa wartość napięcia  czasami może być równa 15V lub 14.998V). Szum procesowy wzmaga błędy estymacji. Tę wartość zwykle zakłada się odgórnie i jest stała podczas obliczeń. Tę wartość należy dobrać eksperymentalnie, gdyż należy ją dopasować do fluktuacji systemu. Nieodpowiednie jej dopasowanie prowadzi do zaniku skuteczności filtracji lub nienadążania wyjścia filtru za sygnałem oryginalnym.

        Działanie filtru opiera się na czterech fazach:
        
        \begin{itemize}
            \item inicjalizacji,
            \item pomiaru,
            \item aktualizacji stanu,
            \item predykcji stanu.
        \end{itemize}

        Podczas inicjalizacji zakładane są odgórnie estymacja stanu początkowego oraz niepewność tej estymacji. Dowolność doboru tych wartości jest duża, gdyż filtr Kalmana i tak całkiem sprawnie zbiegnie do takiej estymacji, gdzie wartości estymowane będą bliskie wartościom rzeczywistym. Etap inicjalizacji odbywa się tylko raz (np. przy włączeniu urządzenia).
        Etap pomiaru nie jest tak naprawdę właściwym etapem tego filtru, lecz tylko obsługą czujników i odebrania z nich danych.
        Etap aktualizacji polega na obliczeniu wzmocnienia Kalmana $K_n$, estymacji aktualnego stanu~$\hat{x}_{n,n}$ oraz obliczeniu niepewności tej estymacji~$p_{n,n}$.
        Etap predykcji polega obliczeniu prognoz estymacji następnego stanu~$\hat{x}_{n+1,n}$ oraz niepewności tej estymacji~$p_{n+1,n}$. 
        Na rysunku \ref{Schemat algorytmu filtru Kalmana} został zaprezentowany algorytm filtru Kalmana w formie graficznej wraz z podanymi wejściami i wyjściami do i z filtru.

        \singlesizedimage{images/kalman.png}{Schemat algorytmu filtru Kalmana}{1.0}
    }

    \label{fusion}
    \subsection{Fuzja wskazań z czujników}
    {
        Fuzja wskazań (ang. \textit{sensor fusion}) z czujników polega na łączeniu informacji pochodzących z różnych czujników, tak aby uzyskać dokładniejsze i kompletne dane \textit{sensor fusion}.
        Jeśli istnieje możliwość skorzystania z kilku czujników, które mierzą tę samą wielkość, to można wykorzystać ich pomiary w celu otrzymania wartości bardziej zbliżonej do wartości rzeczywistej. Jest to sposób filtracji danych, który wykorzystuje inne filtry (np. filtr Kalmana). Można do tego wykorzystać jednowymiarowy, wielowymiarowy filtr Kalmana lub Rozszerzony Filtr Kalmana (EKF) \cite{sensor_fusion}. Wielowymiarowy filtr Kalmana opisano w rozdziale \ref{multi_kalman}

        Taki sposób filtracji ma taką zaletę, że pozwala to na łagodzenie błędów wynikających z działania poszczególnych czujników \cite{sensor_fusion}, a także zakładając również, że zbierane są dane z czujników A i B. Zakładając również, że te czujniki są czujnikami działającymi na innych zasadach. Zastosowanie czujników różnego typu pozwala na wzajemne zmniejszanie szumów z różnych źródeł (np. pirometr i termometr rtęciowy mają inne źródła szumów, inny rozkład błędów pomiaru w czasie). 
        Mając rozkład błędów w czasie dla kilku czujników, to wykorzystanie fuzji tych czujników jest w stanie zapewnić wynikowy rozkład błędów, który jest uśredniony. Jeśli na rozkładzie błędów jednego z czujników w danym miejscu jest obecny pik, a na innym rozkładzie w tym miejscu błąd w porównaniu z tym pikiem jest dużo mniejszy, to ten pik zostanie wygładzony przez uśrednienie rozkładu błędów w tym miejscu.
        
        Nawet jeśli pomiary z dwóch czujników są z sobą skorelowane (np. odczyty z dwóch takich samych modeli odbiornika GPS, które są umieszczone prawie w tym samym miejscu) to i tak~ich~fuzja zapewni pomiar bliżej wartości prawdziwej.
    }

    \subsection{Wielowymiarowy filtr Kalmana}\label{multi_kalman}
    {
        Czasami jednak filtr musi mieć za zadanie dostarczyć na wyjściu wektor pewnych wartości. Wtedy istnieje potrzeba zastosowania równań filtru Kalmana w postaci macierzowej \cite{multi_kalman}. W takim wypadku można filtrować jednocześnie kilka wielkości oraz ich pochodne. Gdy mierzonych wartości jest bardzo dużo, wtedy ta metoda jest wydajniejsza numerycznie niż zastosowanie równań algebraicznych dla każdej mierzonej wartości z osobna.

        Należy jednak tutaj rozszerzyć pojęcie niepewności estymacji i zacząć je nazywać je wariancją. Dotychczas opisana była wariancja w jednym wymiarze. Natomiast mając do czynienia z kilkoma
        zmiennymi, należy wiedzieć, że może te zmienne łączyć jakiś stopień zależności (wyrażony jako skalar). Zależność określa tę się terminem kowariancja \cite{covariance} (wariancja odnosiła się do jednej zmiennej).
        Dla dwóch zmiennych losowych \textbf{x} oraz \textbf{y} zebranych w ilości \textbf{n} próbek kowariancję dla można wyrazić jako:

        \begin{equation}
            \sigma(x,y) = \frac{1}{n-1} \sum^n_{i=1} (x_i - \overline{x})(y_i - \overline{y})
        \end{equation}

        Przy operowaniu na wektorze zmiennych losowych \textbf{X} należy się posługiwać macierzami kowariancji (macierz kwadratowa), które są zdefiniowane jak poniżej.

        \begin{equation}
            \bm{P} = \frac{1}{n-1} \sum^n_{i=1} (X_i - \overline{X})(X_i - \overline{X})^T
        \end{equation}

        \begin{equation}
            \bm{P} \in \mathbb{R}^{d \times d}
        \end{equation}
        gdzie,\\
        $n$ - liczba próbek (macierz kowariancji zmienia się z każdą iteracją algorytmu0),\\
        $d$ - liczba zmiennych losowych (dla x,y,z $d=3$)\\

        Na przykład macierz kowariancji dla zmiennych \textbf{x} oraz \textbf{y} wygląda następująco:

        \begin{equation}
            \bm{P} = 
            \begin{bmatrix}
            \sigma(x,x) & \sigma(x,y) \\
            \sigma(y,x) & \sigma(y,y)
            \end{bmatrix}
        \end{equation}    

        Poniżej zaprezentowano filtr Kalmana w formie macierzowej.

        \begin{equation}
            \hat{\bm{x}}_{n+1,n}= \bm{F} \hat{x}_{n,n} + \bm{G}u_n
        \end{equation}
        \begin{equation}
            \hat{\bm{x}}_{n,n}=\hat{\bm{x}}_{n,n-1}+\bm{K}_n(\bm{z}_N - \bm{H}\hat{\bm{x}}_{n,n-1})
        \end{equation}
        \begin{equation}
            \bm{K}_n=\bm{P}_{n,n-1} \bm{H}^T(\bm{H}\bm{P}_{n,n-1}\bm{H}^T +\bm{R}_n)^{-1}
        \end{equation}
        \begin{equation}
            \bm{P}_{n,n}=(\bm{I} - \bm{K}_n\bm{H}) \bm{P}_{n,n-1} (\bm{I} - \bm{K}_n\bm{H})^T + \bm{K}_n \bm{R}_n \bm{K}_n^T
        \end{equation}
        \begin{equation}
            \bm{P}_{n+1,n}=\bm{F}\bm{P}_{n,n}\bm{F}^T + \bm{Q}
        \end{equation}
        gdzie,\\
        $n_x$ -- długość wektora stanu,\\
        $n_u$ -- długość wektora wejścia,\\
        $n_z$ -- długość wektora z danymi pomiarowymi\\
        $\hat{\bm{x}}_{n+1,n}$ -- estymowany wektor stanu w kroku czasowym $n+1$,\\
        $\hat{\bm{x}}_{n,n}$ -- estymowany wektor stanu w kroku czasowym $n$,\\
        $\hat{\bm{x}}_{n,n-1}$ -- estymowany wektor stanu w kroku czasowym $n-1$,\\
        $\bm{u}_n$ -- wektor wejścia,\\
        $\bm{z}_N$ -- wektor z danymi pomiarowymi,\\
        $\bm{F}$ -- macierz tranzycji stanu o wymiarze $n_x \times n_x$,\\
        $\bm{G}$ -- macierz tranzycji wejścia o wymiarze $n_x \times n_u$,\\
        $\bm{P}_{n+1,n}$ -- macierz kowariancji (niepewności estymacji) następnego (prognozy) stanu, jej wymiar to $n_x \times n_x$,\\
        $\bm{P}_{n,n}$ -- macierz kowariancji (niepewności estymacji) aktualnego  stanu, jej wymiar to $n_x \times n_x$,\\
        $\bm{P}_{n,n-1}$ -- wcześniejsza estymacja dla poprzedniego stanu (jak obliczano tę macierz, ten stan był wtedy aktualnym stanem) $n_x \times n_x$,\\
        $\bm{Q}$ -- macierz szumu procesowego, jej wymiar to $n_x \times n_x$ i może być ona niezależna(wyrazy niezerowe tylko na diagonali) lub zależna,\\
        $\bm{H}$ -- macierz obserwowalności, jest stała i decyduje o tym, które elementy wektora z pomiarami są brane pod uwagę i z jaką skalą. Jej wymiar to~$n_z \times n_x$,\\
        $\bm{R}_n$ -- macierz niepewności pomiaru (kowariancji szumu pomiarowego), jest stała i jej wymiar to $n_x \times n_x$,\\
        $\bm{K}_n$ -- macierz wzmocnienia Kalmana, jej wymiar to $n_x \times n_z$.\\

        \newpage

        Wielowymiarowy filtr Kalmana skonstruowany w postaci macierzowej pozwala na wykorzystanie fuzji czujników w wygodny sposób. Mając pomiary z tej samej chwili czasowej z kilku źródeł, to aby otrzymać ostateczną przefiltrowaną wielkość, należy dla każdego odczytu zaaplikować etap aktualizacji. Odbywa się to w bardzo prosty sposób, którego wizualizację pokazano na rysunku \ref{Fuzja czujników z zastosowaniem macierzowego filtru Kalmana}.

        \singlesizedimage{images/kalman_matrix_multiple_sensors.png}{Fuzja czujników z zastosowaniem macierzowego filtru Kalmana}{0.5}
    }
    \subsection{Filtr Sawickiego-Golaya}
    {
        Jest to jeden z najpopularniejszych filtrów. Filtr Sawickiego-Golaya \cite{filters} dopasowuje wielomian n-tego rzędu do wybranych odcinków sygnału (okna). Ten filtr stosuje tutaj metodę najmniejszych kwadratów. Podczas procesu filtracji filtr ten oblicza nowe dopasowanie wielomianowe za każdym razem, gdy okno pomiarowe przesunie się, czyli  zostanie dokonany nowy pomiar. Filtry te mają większą złożoność obliczeniową niż filtry średniej ruchomej, ale dobrze się dopasowują do sygnału oryginalnego.
    }
    }
}