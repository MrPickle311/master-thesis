\newpage
\section{Czujniki stosowane w robotyce}
{
    \subsection{Wstęp}
    {
        Czujniki są bardzo ważne dla robotów, ponieważ pozwalają im na orientację w danym środowisku. Czujniki umożliwiają robotom pobieranie danych z otoczenia, takich jak odległość~od~innych obiektów, położenie czy też informacje o pogodzie. Na podstawie tych informacji roboty są w stanie podejmować decyzje i wykonywać odpowiednie działania.
    }
    \subsection{IMU}
    {
        IMU (ang. \textit{Inertial Measurement Unit}) jest urządzeniem, które składa się z żyroskopu oraz akcelerometru \cite{imu}. Służy ono do nawigacji bezwładnościowej (inercyjnej). W celu określenia orientacji obiektu mierzy się jego prędkość kątową i działające na niego przyspieszenia.
        \cite{imu_article}
        \subsubsection{Akcelerometr}
        {
            Akcelerometr służy do mierzenia przyspieszeń liniowych \cite{acceleromter}. Pozwala to (za pomocą całkowania) na obliczenie wektora prędkości obiektu. Ma on możliwość dokonania pomiaru w dwóch lub trzech kierunkach. Wielokierunkowe akcelerometry składają się z kilku prostych akcelerometrów, które mierzą przyspieszenie~w~jednym kierunku. Dzięki uzyskanym wskazaniom możliwe jest uzyskanie kąta odchylenia akcelerometru od pionu. Zwykle akcelerometry wskazują przyspieszenie~w~jednostkach, które są iloczynem przyspieszenia grawitacyjnego, np. $2g$. Akcelerometry mierzą absolutne przyspieszenie, oznacza to, że jeśli na akcelerometr nie działa żaden wektor przyspieszenia liniowego skierowanego w kierunku przyspieszenia ziemskiego, to wskaże on przyspieszenie równe $1g$.
        }
        \subsubsection{Żyroskop}
        {
            Żyroskopy są urządzeniami służącymi do pomiaru bieżącej prędkości kątowej danego obiektu \cite{gyro}. Żyroskopy mogą być użyte jako urządzenia samodzielne lub wchodzić w skład bardziej skomplikowanych urządzeń, takich jak np. IMU lub AHRS \cite{ahrs}. Na rynku dostępnych jest kilka implementacji żyroskopów, między innymi takie jak żyroskopy mechaniczne, żyroskopy optyczne żyroskopy RLG \cite{fogs} oraz MEMS \cite{gyro}.  W układach elektronicznych stosuje się najczęściej żyroskopy typu MEMS, które działają na zasadzie detekcji prędkości kątowej za pomocą elementu mechanicznego doznającego wibracji. Takie żyroskopy nie muszą być wyposażone w elementy wymagające łożyskowania. Ta właściwość pozwala na miniaturyzację takiego żyroskopu i zastosowanie go~w~układach elektronicznych. Żyroskop pozwala na zmierzenie prędkości kątowej obiektu względem jego dowolnej osi. Pozwala~on również na obliczenie względnej orientacji (względem początkowego położenia) danego obiektu. Operacja określania względnej orientacji również odbywa się za pomocą całkowania. 
        }
    }
    \subsection{Odbiornik GPS}
    {
        Odbiorniki systemu GPS znalazły szerokie zastosowanie w sektorach militarnym, nawigacyjnym oraz automotive. Odbiornik współpracuje z globalnym systemem lokalizacji satelitarnej \cite{gps}. Nawigacji GPS używa się również na otwartych przestrzeniach oraz w miastach. Lokalizacji satelitarnej można użyć do lokalizacji obiektów, nawigacji, mierzenia prędkości oraz mierzenia czasu \cite{gps_principles}. Nawigacja satelitarna działa na zasadzie współpracy trzech elementów: satelitów, stacji kontrolnych oraz odbiorników GPS. Satelity obecne na orbicie wokół Ziemi wysyłają sygnał radiowy w sposób ciągły w jej kierunku. Sygnały te przebywają całą swoją trasę z prędkością bliską światła \cite{gps_principles}. Sygnały są generowane przez układy nadawcze satelitów. \cite{gps_principles}. Odbiornik odbiera nadchodzący sygnał radiowy~z~kilku widocznych satelitów i oblicza odległość od każdego z nich~z~osobna. Odległość odbiornika od satelity wyraża się poprzez iloczyn prędkości fali radiowej oraz różnicy czasu między wysłaniem fali radiowej przez satelitę a odebraniem jej przez odbiornik.

        Mając obliczone odległości od poszczególnych satelitów to dzięki trilateracji (metody określania tego, jak daleko dany punkt znajduje się od innych trzech punktów) można obliczyć szerokość geograficzną, długość geograficzną, wysokość oraz czas \cite{trilateration}. Trilateracja w procesowaniu GPS oznacza wyznaczenie przecięcia się jak największej ilości sfer (sfery są zbudowane z promieni, które~są~odległościami między odbiornikiem GPS a poszczególnymi satelitami) w pobliżu jednego punktu~i~na~tej~podstawie estymację położenia odbiornika i tym samym obiektu.
        
        \vspace{5mm}
        \singlesizedurlimage{images/gps_trit.png}{Wizualiacja algorytmu trilateracji} {0.8}{\cite{gps_sat_pic}}
        \vspace{5mm}
        
        Odbiorniki GPS podają swoją lokalizację w formacie NMEA \cite{nmea}. Protokół NMEA jest ogólnym protokołem stosowanym w nawigacji satelitarnej oraz w komunikacji między elektronicznym osprzętem obecnym na statkach. Protokół ten definiuje kilka komend, lecz najważniejszą jest \$GPGGA,~która zawiera w sobie pozycję geograficzną obiektu. Przykład takiej wiadomości podano poniżej:
        
        
        \begin{lstlisting}[language=xml]
            $GPGGA,055415.000,3804.8000,S,17617.4000,E,2,06,1.8,69.5,M,26.2,M,0.8,0000*54
         \end{lstlisting}

        Szerokość geograficzna jest podana na drugim miejscu, a długość geograficzną na trzecim miejscu. Po odpowiednich konwersjach otrzymanej wiadomości istnieje możliwość  poznania lokalizacji odbiornika wyrażonej we współrzędnych geograficznych. Minimalna liczba satelitów pozwalająca na określenie pozycji geograficznej wynosi 4 satelity. Większa liczba satelitów zwiększa precyzję oraz zmniejsza tendencję do nagłego przeskakiwania pozycji na mapie. Dokładność określenia położenia zależy od liczby widocznych satelitów.
    }
    \newpage
    \subsection{Magnetometr}
    {
        Jest to czujnik służący do pomiaru wektora (moduł i kierunek) pola magnetycznego \cite{magnetometer}. Magnetometry znajdują zastosowanie w takich sektorach jak wojskowość oraz nawigacja. Są też osprzętem często obecnym na telefonach oraz wszystkich urządzeniach, które muszą być mobilne np. drony. Zależnie od rodzaju magnetometr może być dwuosiowy lub trójosiowy. Na rynku dostępnych jest kilka różnych implementacji magnetometrów, są to między innymi magnetometry korzystające z efektu Halla, magnetometry GMR, magnetometry MTJ oraz AMR \cite{magneto_work}. Wybór danego rodzaju magnetometru podyktowany jest zwykle jego ceną oraz jakością wskazań. Najpopularniejszym magnetometrem jest ten, który wykorzystuje efekt Halla. Gdy obiekt jest położony w silnym polu magnetycznym prostopadłym do jego płaszczyzny \ref{Wizualizacja działania magnetometru opartego o czujnik Halla}, wtedy na obwodzie przyłączonym do tego obiektu wykrywane jest napięcie, zwane napięciem Halla.
        Kompas skonstruowany~na~bazie magnetometru pozwala w prosty sposób określić bezwzględną orientację obiektu względem północy.

        \singlesizedurlimage{images/mag_work.png}{Wizualizacja działania magnetometru opartego o czujnik Halla} {0.3}{\cite{mag_work_pic}}
    }
    \subsection{Enkoder}
    {
        Enkoder jest czujnikiem, który jest zamontowany na wale obrotowym koła lub innego mechanizmu. Za pomocą enkodera istnieje możliwość zmierzenia liczby impulsów w danym czasie, ~co~pozwala obliczyć przebyty dystans w zadanym czasie. 
        Enkodery inkrementalne \cite{encoders} generują serię impulsów od momentu włączenia zasilania. Zwykle wykorzystują tarczę osadzoną na wale,~w~której wykonane są otwory. Przetwarzają one obrót tarczy na dwukanałowy sygnał kwadraturowy \cite{encoders}.
    }
}