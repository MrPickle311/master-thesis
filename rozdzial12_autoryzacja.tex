\section{System autoryzacji i zarządzania użytkownikami}
\label{chap:autoryzacja}

System analizy danych w czasie rzeczywistym wykorzystuje nowoczesny mechanizm autoryzacji oparty na tokenach JWT (ang. JSON Web Token) z wykorzystaniem serwera Keycloak jako dostawcy tożsamości. Takie rozwiązanie zapewnia bezpieczny dostęp do zasobów systemu oraz umożliwia skalowalne zarządzanie użytkownikami i rolami.

\subsection{Architektura systemu autoryzacji}

System Keycloak pełni rolę centralnego serwera uwierzytelnienia, który zarządza procesami logowania, wydawania tokenów oraz kontroli dostępu. Aplikacja frontendowa komunikuje się bezpośrednio z systemem Keycloak w celu uzyskania tokenów dostępu, które następnie są wykorzystywane do autoryzacji żądań kierowanych do usług back-endowych poprzez aplikację FrontService - bramę API działającą w architekturze mikroserwisowej.

FrontService stanowi centralny punkt dostępu do całego klastra mikroserwisów, implementując jednolity system autoryzacji dla wszystkich usług back-endowych. Dzięki tej architekturze, każdy mikroserwis w klastrze może polegać na walidacji tokenów przeprowadzonej przez bramę, co upraszcza implementację autoryzacji w poszczególnych serwisach oraz zapewnia spójność mechanizmów bezpieczeństwa w całym systemie.

Back-end aplikacji został skonfigurowany do pracy w trybie hybrydowym - może działać zarówno z serwerem Keycloak jako zewnętrznym dostawcą uwierzytelnienia, jak i z własnym wewnętrznym systemem JWT (ang. JSON Web Token). Konfiguracja ta jest kontrolowana przez parametr \texttt{app.security.useKeycloak}, co zapewnia elastyczność wdrożenia aplikacji w różnych środowiskach.

\subsection{Implementacja po stronie frontendu}

Aplikacja frontendowa wykorzystuje bibliotekę \texttt{keycloak-js} do integracji z serwerem Keycloak. Konfiguracja połączenia jest realizowana przez zmienne środowiskowe określające URL serwera, nazwę grupy oraz identyfikator klienta. System automatycznie przekierowuje użytkowników do strony logowania serwera Keycloak w przypadku braku ważnego tokenu dostępu.

Po pomyślnym uwierzytelnieniu, aplikacja automatycznie dołącza token Bearer do nagłówka \texttt{Authorization} wszystkich żądań HTTP (ang. Hypertext Transfer Protocol) kierowanych do interfejsu API. Implementacja obejmuje mechanizm automatycznego odświeżania tokenów, który sprawdza ważność tokenu przed każdym żądaniem i w razie potrzeby odnawia go w sposób przezroczysty dla użytkownika.

Interface \texttt{KeycloakInterface} zapewnia zunifikowany interfejs API do zarządzania sesją użytkownika, sprawdzania ról oraz wykonywania operacji logowania i wylogowania. Implementacja ta abstrahuje szczegóły komunikacji z serwerem Keycloak, ułatwiając integrację z komponentami biblioteki React.

\subsection{Autoryzacja po stronie back-endu}

Aplikacja FrontService implementuje wielowarstwowy system autoryzacji wykorzystujący bibliotekę Spring Security WebFlux  w trybie Keycloak. System wykorzystuje mechanizm protokołu OAuth2 Resource Server do walidacji tokenów JWT wydanych przez serwer uwierzytelnienia. Tokeny są weryfikowane pod kątem poprawności podpisu, ważności czasowej oraz uprawnień użytkownika.

Ekstraktor uprawnień \texttt{KeycloakGrantedAuthoritiesConverter} analizuje strukturę tokenu JWT w celu pobrania ról zarówno z poziomu grupy jak i klienta. Umożliwia to precyzyjne mapowanie uprawnień serwera Keycloak na role używane wewnętrznie przez aplikację. System rozpoznaje hierarchię uprawnień, gdzie role grupy mają wyższy priorytet niż role specyficzne dla klienta.

\subsection{Model uprawnień}

System definiuje dwie podstawowe role użytkowników: \texttt{DATA\_ACCESSOR} oraz \texttt{ADMIN}. Użytkownicy z rolą \texttt{DATA\_ACCESSOR} mają dostęp do funkcji odczytu i analizy danych, obejmujących przeglądanie raportów, panelu sterowania oraz konfiguracji systemu. Rola \texttt{ADMIN} rozszerza te uprawnienia o możliwość zarządzania użytkownikami, modyfikacji konfiguracji systemu oraz dostępu do funkcji administracyjnych.

Kontrola dostępu jest implementowana na poziomie endpointów API poprzez adnotacje Spring Security. Ścieżki publiczne, takie jak: endpointy healthcheck, są dostępne bez uwierzytelnienia, podczas gdy zasoby biznesowe wymagają odpowiednich ról. System automatycznie zwraca kod odpowiedzi HTTP 401 dla żądań bez ważnego tokenu oraz 403 dla żądań z niewystarczającymi uprawnieniami.

\subsection{Zarządzanie użytkownikami}

Proces zarządzania kontami użytkowników odbywa się centralnie przez panel administracyjny serwera Keycloak. Administratorzy systemu mają wyłączną kontrolę nad tworzeniem nowych kont użytkowników, przypisywaniem ról oraz zarządzaniem uprawnieniami dostępu. Takie podejście zapewnia pełną kontrolę nad bezpieczeństwem systemu oraz eliminuje ryzyko związane z samorejestracją użytkowników. Administrator może definiować szczegółowe profile użytkowników, ustalać zasady haseł oraz kontrolować aktywność kont. 

\subsection{Bezpieczeństwo i zarządzanie sesjami}

Implementacja uwzględnia współczesne standardy bezpieczeństwa aplikacji webowych. Tokeny typu JWT (ang. \textit{JSON Web Token}) mają ograniczony czas życia, co minimalizuje ryzyko w przypadku ich kompromitacji. System automatycznie czyści tokeny z pamięci przeglądarki po wylogowaniu użytkownika oraz implementuje mechanizmy ochrony przed atakami typu CSRF (ang. Cross-Site Request Forgery) poprzez odpowiednią konfigurację nagłówków CORS (ang. Cross-Origin Resource Sharing).

Aplikacja FrontService implementuje dodatkowe mechanizmy bezpieczeństwa, w tym ograniczenie liczby prób logowania z określonego adresu IP za pomocą cache-a biblioteki Caffeine. Konfiguracja ta chroni system przed atakami typu brute force poprzez czasowe blokowanie adresów IP po przekroczeniu limitu nieudanych prób uwierzytelnienia.

Zarządzanie błędami autoryzacji jest zunifikowane przez \texttt{ReactiveExceptionHandler}, który zapewnia spójne komunikaty błędów oraz odpowiednie kody odpowiedzi HTTP niezależnie od źródła błędu uwierzytelnienia czy autoryzacji. 