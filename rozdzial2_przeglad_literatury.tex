\section{Przegląd litartury i stan wiedzy}
\label{sec:przeglad_literatury}

W niniejszym rozdziale przedstawiono aktualny stan wiedzy w zakresie analizy danych w czasie rzeczywistym, systemów wieloczujnikowych,
architektury mikroserwisowej, platformy Kubernetes oraz narzędzi Big Data. Omówiono również istniejące rozwiązania i ich ograniczenia.

\subsection{Analiza danych w czasie rzeczywistym}
\label{subsec:analiza_danych}

Analiza danych w czasie rzeczywistym to proces przetwarzania i analizy danych natychmiast po ich wygenerowaniu,
umożliwiający podejmowanie decyzji na podstawie aktualnych informacji \citep{realtime_analytics}. W przeciwieństwie do tradycyjnego przetwarzania
wsadowego (ang. batch processing), które operuje na dużych zbiorach historycznych danych, analiza w czasie rzeczywistym koncentruje się na przetwarzaniu
strumieni danych (ang. stream processing) z minimalnymi opóźnieniami.

\subsubsection{Modele przetwarzania danych}
\label{subsubsec:modele_przetwarzania}

W literaturze wyróżnia się trzy główne modele przetwarzania danych \citep{data_processing_models}:

\begin{itemize}
    \item \textbf{przetwarzanie wsadowe (ang. batch processing)} - dane są gromadzone w pakiety i przetwarzane periodycznie, np. raz dziennie lub raz na tydzień. Model ten charakteryzuje się wysoką przepustowością, ale dużymi opóźnieniami,
    \item \textbf{przetwarzanie mikrowsadowe (ang. micro-batch processing)} - dane są przetwarzane w małych pakietach, z częstotliwością rzędu sekund lub minut. Jest to kompromis między przetwarzaniem wsadowym a strumieniowym,
    \item \textbf{przetwarzanie strumieniowe (ang. stream processing)} - dane są przetwarzane natychmiast po ich wygenerowaniu, indywidualnie lub w bardzo małych grupach. Model ten charakteryzuje się niskimi opóźnieniami, ale często niższą przepustowością.
\end{itemize}

W kontekście analizy danych z systemów wieloczujnikowych, szczególnie istotne jest przetwarzanie strumieniowe, które umożliwia szybką reakcję na zmieniające się warunki procesu produkcyjnego.

\subsubsection{Wyzwania w analizie danych w czasie rzeczywistym}
\label{subsubsec:wyzwania_analizy_rt}

Analiza danych w czasie rzeczywistym, choć oferuje wiele korzyści, stawia przed projektantami systemów szereg istotnych wyzwań technicznych. Według \cite{geeksforgeeks_realtime}, kluczowe wyzwania obejmują:

\begin{itemize}
    \item \textbf{duża objętość i wysoka prędkość danych} - systemy czasu rzeczywistego muszą obsługiwać ogromne ilości danych napływających z wysoką prędkością. Wymaga to wydajnej infrastruktury zdolnej do przetwarzania danych z minimalnym opóźnieniem. Rozwiązaniem jest wykorzystanie systemów rozproszonych, optymalizacja algorytmów oraz efektywne zarządzanie zasobami,
    
    \item \textbf{wymóg niskich opóźnień} - aplikacje czasu rzeczywistego wymagają niemal natychmiastowego przetwarzania danych i generowania odpowiedzi. Opóźnienia mogą prowadzić do nieaktualnych wyników i nieprawidłowych decyzji. Kluczowe jest zastosowanie przetwarzania w pamięci, optymalizacji zapytań i minimalizacji operacji wejścia/wyjścia,
    
    \item \textbf{spójność i dokładność danych} - utrzymanie spójności danych podczas ich przetwarzania jest krytyczne dla poprawności analizy. Systemy muszą zapewnić dokładną semantykę przetwarzania (exactly-once processing) i obsługę nieporządku czasowego w danych,
    
    \item \textbf{odporność na awarie i niezawodność} - systemy analizy czasu rzeczywistego muszą działać nieprzerwanie, nawet w przypadku awarii poszczególnych komponentów. Wymagane są mechanizmy automatycznego odzyskiwania po awarii, replikacji danych i równoważenia obciążenia,
    
    \item \textbf{przetwarzanie złożonych zdarzeń} - wykrywanie wzorców i korelacji w strumieniach danych wymaga zaawansowanych technik przetwarzania zdarzeń. Konieczne jest zastosowanie algorytmów CEP (Complex Event Processing) zdolnych do identyfikacji istotnych wzorców w czasie rzeczywistym,
    
    \item \textbf{integracja z istniejącymi systemami} - nowe rozwiązania muszą współpracować z istniejącą infrastrukturą IT. Wymaga to wykorzystania standardowych interfejsów, brokerów wiadomości i metod transformacji danych,
    
    \item \textbf{skalowalność} - systemy powinny płynnie dostosowywać się do zmieniających się wolumenów danych i wzorców ruchu. Kluczowe jest projektowanie architektury umożliwiającej poziome skalowanie oraz elastyczne przydzielanie zasobów,
    
    \item \textbf{bezpieczeństwo i prywatność} - ochrona wrażliwych danych podczas przetwarzania w czasie rzeczywistym stanowi istotne wyzwanie. Systemy muszą implementować szyfrowanie, kontrolę dostępu oraz mechanizmy audytu, zapewniając jednocześnie zgodność z przepisami dotyczącymi ochrony danych.
\end{itemize}

Skuteczne rozwiązanie tych wyzwań wymaga starannego projektowania architektury systemu, doboru odpowiednich technologii oraz uwzględnienia specyficznych wymagań aplikacji \cite{geeksforgeeks_realtime}. W kolejnych rozdziałach omówione zostaną narzędzia i techniki, które pozwalają sprostać tym wyzwaniom.

\subsection{Systemy wieloczujnikowe w przemyśle}
\label{subsec:systemy_wieloczujnikowe}

Systemy wieloczujnikowe to zbiory urządzeń pomiarowych, które monitorują różne parametry procesów przemysłowych \citep{multisensor_systems}.
W nowoczesnych zakładach produkcyjnych, systemy te generują ogromne ilości danych, które odpowiednio przetworzone mogą
dostarczyć cennych informacji na temat stanu procesu.

\subsubsection{Rodzaje czujników przemysłowych}
\label{subsubsec:rodzaje_czujnikow}

W przemyśle wykorzystuje się różne rodzaje czujników:

\begin{itemize}
    \item \textbf{czujniki temperatury} - monitorują temperaturę w różnych punktach instalacji,
    \item \textbf{czujniki ciśnienia} - mierzą ciśnienie gazu lub cieczy w instalacji,
    \item \textbf{czujniki przepływu} - monitorują przepływ cieczy i gazów przez instalację,
    \item \textbf{czujniki składu gazu} - analizują skład mieszanin gazowych, np. zawartość wodoru, azotu, amoniaku, tlenu czy dwutlenku węgla,
    \item \textbf{czujniki drgań i hałasu} - monitorują drgania i hałas generowane przez urządzenia, co może wskazywać na potencjalne problemy,
    \item i inne.
\end{itemize}

\subsubsection{Architektura systemów wieloczujnikowych}
\label{subsubsec:architektura_systemow}

Tradycyjne systemy wieloczujnikowe opierają się na architekturze trójwarstwowej:

\begin{itemize}
    \item \textbf{warstwa czujników} - obejmuje czujniki fizyczne wraz z konwerterami analogowo-cyfrowymi,
    \item \textbf{warstwa komunikacji} - odpowiada za przesyłanie danych z czujników do warstwy przetwarzania,
    \item \textbf{warstwa przetwarzania} - przetwarza dane z czujników i udostępnia je użytkownikom lub innym systemom.
\end{itemize}

W nowoczesnych systemach, warstwa komunikacji często wykorzystuje technologie IoT, takie jak MQTT, AMQP czy OPC UA,
a warstwa przetwarzania opiera się na mechanizmach przetwarzania strumieniowego, takich jak Apache Kafka czy Apache Flink.


\subsection{Architektura mikroserwisowa}
\label{subsec:architektura_mikroserwisowa}

Architektura mikroserwisowa to podejście do tworzenia aplikacji jako zbioru luźno powiązanych, małych, autonomicznych usług,
komunikujących się ze sobą za pomocą mechanizmów, takich jak protokół HTTP \citep{microservice_architecture}.
Każdy mikroserwis realizuje określoną funkcjonalność biznesową i może być rozwijany, wdrażany i skalowany niezależnie od innych usług.

\subsubsection{Zalety architektury mikroserwisowej}
\label{subsubsec:zalety_mikroserwisow}

Architektura mikroserwisowa oferuje szereg zalet w kontekście systemów do analizy danych w czasie rzeczywistym \citep{microservice_benefits}:

\begin{itemize}
    \item \textbf{skalowalność} - możliwość niezależnego skalowania poszczególnych usług w zależności od obciążenia,
    \item \textbf{odporność na awarie} - awaria jednego mikroserwisu nie powoduje awarii całego systemu,
    \item \textbf{elastyczność technologiczna} - możliwość wykorzystania różnych technologii i języków programowania w różnych mikroserwisach,
    \item \textbf{szybsze wdrażanie} - możliwość niezależnego wdrażania poszczególnych mikroserwisów,
    \item \textbf{łatwiejsze zarządzanie kodem} - mniejsze, bardziej zrozumiałe bazy kodu dla poszczególnych usług.
\end{itemize}

\subsubsection{Wyzwania architektury mikroserwisowej}
\label{subsubsec:wyzwania_mikroserwisow}

Architektura mikroserwisowa stawia również pewne wyzwania \citep{microservice_challenges}:

\begin{itemize}
    \item \textbf{złożoność operacyjna} - zarządzanie wieloma niezależnymi usługami może być skomplikowane,
    \item \textbf{spójność danych} - trudności w utrzymaniu spójności danych między różnymi mikroserwisami,
    \item \textbf{koszty komunikacji sieciowej} - komunikacja między mikroserwisami wprowadza dodatkowe opóźnienia,
    \item \textbf{testowanie end-to-end} - trudności w testowaniu całego systemu złożonego z wielu niezależnych usług.
\end{itemize}

\subsection{Kubernetes jako platforma orkiestracji kontenerów}
\label{subsec:kubernetes}

Kubernetes to otwarta platforma do orkiestracji kontenerów, która automatyzuje wdrażanie, skalowanie i zarządzanie aplikacjami
kontenerowymi \citep{kubernetes}. Powstała jako projekt Google, obecnie rozwijana przez Cloud Native Computing Foundation (CNCF).

\subsubsection{Podstawowe koncepcje Kubernetes}
\label{subsubsec:podstawy_kubernetes}

Kubernetes operuje na kilku kluczowych koncepcjach \citep{kubernetes_concepts}:

\begin{itemize}
    \item \textbf{pod} - najmniejsza jednostka wdrożeniowa w Kubernetes, składająca się z jednego lub więcej kontenerów,
    \item \textbf{deployment} - określa pożądany stan podów, umożliwiając ich skalowanie i aktualizacje,
    \item \textbf{service} - abstrakcja, która definiuje logiczny zestaw podów i politykę dostępu do nich,
    \item \textbf{ingress} - zarządza zewnętrznym dostępem do usług w klastrze,
    \item \textbf{configMap} i \textbf{Secret} - mechanizmy do przechowywania konfiguracji i tajnych danych,
    \item \textbf{namespace} - mechanizm do izolacji zasobów w klastrze.
\end{itemize}

\subsubsection{Zalety Kubernetes w kontekście analizy danych w czasie rzeczywistym}
\label{subsubsec:zalety_kubernetes}

Kubernetes oferuje szereg zalet w kontekście rozproszonych systemów \citep{kubernetes_benefits}:

\begin{itemize}
    \item \textbf{automatyczne skalowanie} - możliwość automatycznego dostosowywania liczby replik usług w zależności od obciążenia,
    \item \textbf{samonaprawianie} - automatyczne ponowne uruchamianie podów doznających awarii,
    \item \textbf{równoważenie obciążenia} - równomierne rozłożenie ruchu między replikami usług,
    \item \textbf{aktualizacje bez przestojów} - możliwość aktualizacji usług bez przerywania ich działania,
    \item \textbf{deklaratywna konfiguracja} - definiowanie pożądanego stanu systemu, a nie kroków do jego osiągnięcia.
\end{itemize}

\subsection{Narzędzia Big Data do przetwarzania strumieniowego}
\label{subsec:narzedzia_big_data}

W kontekście analizy danych w czasie rzeczywistym, szczególnie istotne są narzędzia Big Data do przetwarzania strumieniowego.
Poniżej omówiono kilka kluczowych technologii wykorzystywanych w tym obszarze.

\subsubsection{Apache Kafka}
\label{subsubsec:apache_kafka}

Apache Kafka to rozproszona platforma do przetwarzania strumieniowego, opracowana przez LinkedIn, obecnie rozwijana jako projekt
Apache Software Foundation \citep{kafka}. Kafka oferuje następujące funkcjonalności:

\begin{itemize}
    \item \textbf{wysoka przepustowość} - możliwość obsługi milionów wiadomości na sekundę,
    \item \textbf{trwałość danych} - dane są przechowywane na dysku i replikowane między brokerami,
    \item \textbf{skalowalność} - łatwe skalowanie poziome przez dodawanie nowych brokerów,
    \item \textbf{mechanizm partycjonowania} - umożliwia równoległe przetwarzanie danych,
    \item \textbf{gwarancje dostarczania} - co najmniej raz, co najwyżej raz lub dokładnie raz.
\end{itemize}

\subsubsection{Kafka Streams}
\label{subsubsec:kafka_streams}

Kafka Streams to biblioteka przetwarzania strumieniowego, zintegrowana z Apache Kafka \citep{kafka_streams}. Oferuje następujące funkcjonalności:

\begin{itemize}
    \item \textbf{przetwarzanie rekord po rekordzie} - minimalne opóźnienia przetwarzania,
    \item \textbf{operacje stanowe i bezstanowe} - możliwość agregacji danych w czasie,
    \item \textbf{okna czasowe} - przetwarzanie danych w zdefiniowanych oknach czasowych,
    \item \textbf{łączenie strumieni} - możliwość łączenia danych z różnych strumieni,
    \item \textbf{dokładnie raz} - gwarancje przetwarzania dokładnie raz, eliminujące duplikaty i utratę danych.
\end{itemize}

\subsubsection{Apache Flink}
\label{subsubsec:apache_flink}

Apache Flink to framework przetwarzania strumieniowego, oferujący funkcjonalności podobne do Kafka Streams,
ale jako oddzielna platforma \citep{flink}. Flink charakteryzuje się:

\begin{itemize}
    \item \textbf{niskimi opóźnieniami} - przetwarzanie rekord po rekordzie z minimalnymi opóźnieniami,
    \item \textbf{wysoką przepustowością} - efektywne przetwarzanie dużych wolumenów danych,
    \item \textbf{dokładnie raz} - gwarancje przetwarzania dokładnie raz,
    \item \textbf{zaawansowanym zarządzaniem stanem} - efektywne przechowywanie i dostęp do stanu przetwarzania,
    \item \textbf{obsługą czasu zdarzeń} - możliwość przetwarzania danych na podstawie czasu, w którym zdarzenia zostały wygenerowane.
\end{itemize}

\subsubsection{Apache Spark Streaming}
\label{subsubsec:spark_streaming}

Apache Spark Streaming to moduł przetwarzania strumieniowego platformy Apache Spark \citep{spark_streaming}.
Opiera się na modelu mikrowsadowym, gdzie dane są przetwarzane w małych pakietach. Spark Streaming oferuje:

\begin{itemize}
    \item \textbf{integrację z ekosystemem Spark} - możliwość wykorzystania bibliotek Spark do analizy danych i uczenia maszynowego. Ekosystem ten obejmuje m.in. Spark SQL do przetwarzania danych strukturalnych oraz Spark ML (Machine Learning Library) do implementacji algorytmów uczenia maszynowego,
    \item \textbf{wysoką przepustowość} - efektywne przetwarzanie dużych wolumenów danych,
    \item \textbf{odporność na awarie} - automatyczne odtwarzanie stanu po awarii,
    \item \textbf{łatwe skalowanie} - możliwość łatwego skalowania przetwarzania przez dodawanie węzłów.
\end{itemize}

\subsubsection{Apache Avro i Confluent Schema Registry}
\label{subsubsec:avro_schema_registry}

W rozproszonych systemach przetwarzania danych, zwłaszcza tych wykorzystujących Apache Kafka, kluczowe znaczenie ma efektywna serializacja danych oraz zarządzanie ich schematami. Pozwala to na minimalizację narzutu komunikacyjnego oraz zapewnienie spójności i kompatybilności danych między różnymi komponentami systemu.

\textbf{Apache Avro} to format serializacji danych oparty na schematach, który zapewnia kompaktową reprezentację binarną oraz bogate możliwości ewolucji schematów \citep{avro_documentation}. Dzięki temu możliwe jest modyfikowanie struktury danych bez zakłócania pracy istniejących producentów i konsumentów. Avro jest szczególnie popularne w ekosystemie Apache Kafka ze względu na swoją wydajność i elastyczność.

\textbf{Confluent Schema Registry} to usługa działająca jako centralne repozytorium schematów (m.in. Avro, JSON Schema, Protobuf) \citep{confluent_schema_registry}. Integruje się z klientami Kafka, umożliwiając automatyczną rejestrację, walidację i pobieranie schematów podczas serializacji i deserializacji wiadomości. Schema Registry pomaga w utrzymaniu jakości danych, zapobiega problemom związanym z niekompatybilnością schematów oraz ułatwia zarządzanie zmianami w strukturze danych w dynamicznie rozwijających się systemach. W ramach opisywanego projektu, Confluent Schema Registry zostało wykorzystane do zarządzania schematami Avro dla danych przesyłanych przez Apache Kafka.

\subsection{Istniejące rozwiązania i ich ograniczenia}
\label{subsec:istniejace_rozwiazania}

W literaturze i praktyce przemysłowej istnieje szereg rozwiązań do analizy danych z systemów wieloczujnikowych. Poniżej omówiono kilka z nich, wraz z ich ograniczeniami.

\subsubsection{Tradycyjne systemy SCADA}
\label{subsubsec:scada}

Systemy SCADA (Supervisory Control and Data Acquisition) to tradycyjne rozwiązania do monitorowania i kontroli procesów przemysłowych \citep{scada}.
Mimo popularności, systemy te mają pewne ograniczenia w kontekście analizy danych w czasie rzeczywistym:

\begin{itemize}
    \item \textbf{ograniczona skalowalność} - trudności w obsłudze dużej liczby czujników i strumieni danych,
    \item \textbf{monolityczna architektura} - utrudnia elastyczne rozwijanie i modyfikowanie systemu,
    \item \textbf{ograniczone możliwości analityczne} - często koncentrują się na wizualizacji danych, a nie ich głębokiej analizie,
    \item \textbf{wysokie koszty licencji} - komercyjne systemy SCADA często wiążą się z wysokimi kosztami licencji.
\end{itemize}

\subsubsection{Platformy IoT w chmurze}
\label{subsubsec:cloud_iot}

Platformy IoT w chmurze, takie jak AWS IoT, Azure IoT Hub czy Google Cloud IoT Core, oferują zaawansowane możliwości analizy danych z
urządzeń IoT \citep{cloud_iot}. Mimo to, mają pewne ograniczenia:

\begin{itemize}
    \item \textbf{zależność od dostawcy chmury} - trudności w migracji między różnymi dostawcami,
    \item \textbf{koszty transferu danych} - wysokie koszty przy dużym wolumenie danych,
    \item \textbf{opóźnienia sieciowe} - potencjalne opóźnienia związane z przesyłaniem danych do chmury,
    \item \textbf{ograniczone możliwości dostosowania} - platformy chmurowe oferują określony zestaw usług, które mogą nie spełniać wszystkich wymagań.
\end{itemize}

\subsubsection{Open source rozwiązania do analizy danych w czasie rzeczywistym}
\label{subsubsec:open_source}

Istnieje szereg open source rozwiązań do analizy danych w czasie rzeczywistym, takich jak Apache NiFi,
Apache Druid czy InfluxDB . Mimo ich zalet, mają również pewne ograniczenia:

\begin{itemize}
    \item \textbf{złożoność wdrożenia} - konfiguracja i wdrożenie mogą być skomplikowane,
    \item \textbf{ograniczone wsparcie} - wsparcie techniczne może być ograniczone w porównaniu do rozwiązań komercyjnych,
    \item \textbf{konieczność integracji wielu narzędzi} - często wymagają integracji wielu narzędzi, co zwiększa złożoność systemu.
\end{itemize}

Podsumowując, istniejące rozwiązania do analizy danych z systemów wieloczujnikowych mają pewne ograniczenia,
które uzasadniają potrzebę opracowania nowego systemu, opartego na architekturze mikroserwisowej, klastrze Kubernetes i narzędziach Big Data.