\section{Wprowadzenie}
\label{sec:wprowadzenie}

\subsection{Kontekst i motywacja badań}
\label{subsec:kontekst}

W dobie czwartej rewolucji przemysłowej (Przemysł 4.0) i Internetu Rzeczy (IoT), systemy wieloczujnikowe stają się nieodłącznym
elementem nowoczesnych procesów produkcyjnych. Generują one ogromne ilości danych, które odpowiednio wykorzystane mogą dostarczyć
cennych informacji na temat stanu i wydajności monitorowanych procesów. Analiza tych danych w czasie rzeczywistym stanowi jednak
wyzwanie, zarówno pod względem technicznym, jak i organizacyjnym.

Tradycyjne podejścia do przetwarzania danych, oparte na scentralizowanych systemach, często nie są w stanie efektywnie obsłużyć
dużej liczby równoczesnych strumieni danych przy zachowaniu niskich opóźnień. Ponadto, skalowanie takich systemów w odpowiedzi na rosnące
obciążenie może być problematyczne i kosztowne.

Rozproszona architektura, w połączeniu z technologiami konteneryzacji i orkiestracji, takimi jak Kubernetes,
oferuje nowe możliwości w zakresie budowy systemów do analizy danych w czasie rzeczywistym. Umożliwia ona elastyczne skalowanie,
izolację usług i łatwiejsze zarządzanie złożonymi aplikacjami. Jednocześnie narzędzia Big Data, takie jak Apache Kafka czy Apache Spark zapewniają
wydajne mechanizmy do przetwarzania strumieniowego, niezbędne w analizie danych w czasie rzeczywistym.

Motywacją do podjęcia niniejszych badań jest potrzeba sprawdzenia możliwości do opracowania wydajnego i skalowalnego do
analizy danych w czasie rzeczywistym z systemów wieloczujnikowych, który mógłby być wykorzystany w różnych procesach przemysłowych.

\subsection{Cel i zakres pracy}
\label{subsec:cel}

Głównym celami niniejszej pracy są zaprojektowanie, implementacja i analiza wydajności systemu do przetwarzania danych w
czasie rzeczywistym z wielu czujników.

Szczegółowe cele pracy obejmują:
\begin{itemize}
    \item Opracowanie architektury systemu do analizy danych w czasie rzeczywistym z systemów wieloczujnikowych, opartego na klastrze Kubernetes i narzędziach Big Data
    \item Implementacja systemu zgodnie z zaprojektowaną architekturą, z wykorzystaniem mikroserwisów i technologii przetwarzania strumieniowego
    \item Opracowanie i implementacja algorytmów do analizy danych w czasie rzeczywistym, w tym algorytmów do detekcji anomalii i przewidywania awarii
    \item Przeprowadzenie badań eksperymentalnych w celu oceny wydajności i skalowalności opracowanego systemu
    \item Analiza przydatności systemu w środowisku przemysłowym, ze szczególnym uwzględnieniem procesu syntezy amoniaku
\end{itemize}

Zakres pracy obejmuje:
\begin{itemize}
    \item Analizę istniejących rozwiązań do przetwarzania danych w czasie rzeczywistym
    \item Projekt i implementację systemu przetwarzania danych opartego na klastrze Kubernetes
    \item Opracowanie mechanizmów agregacji i analizy danych z wielu czujników
    \item Implementację wizualizacji danych w czasie rzeczywistym
    \item Testowanie wydajności i skalowalności systemu
    \item Ocenę przydatności zaimplementowanego rozwiązania w środowisku produkcyjnym
\end{itemize}

\subsection{Metodologia badań}
\label{subsec:metodologia}

Badania przeprowadzone w ramach niniejszej pracy opierają się na następującej metodologii:

\begin{enumerate}
    \item pomiar obciążenia procesora i zajętości pamięci operacyjnej,
    \item testowanie poprawności wykonywania algorytmów pod zwiększonym obciążeniem,
    \item testowanie automatycznego skalowania systemu,
    \item pomiar przepustowości i opóźnień przetwarzania.
\end{enumerate}

W badaniach wykorzystano dane symulowane, generowane przez funkcje AWS Lambda.

\subsection{Struktura pracy}
\label{subsec:struktura}

Praca składa się z następujących rozdziałów:

\begin{itemize}
    \item \textbf{Rozdział 1: Wprowadzenie} - przedstawia kontekst i motywację badań, cel i zakres pracy, tezę oraz metodologię badań
    \item \textbf{Rozdział 2: Przegląd literatury i stan wiedzy} - omawia istniejące rozwiązania w zakresie analizy danych w czasie rzeczywistym, architektury mikroserwisowej, Kubernetes oraz narzędzi Big Data
    \item \textbf{Rozdział 3: Projekt systemu przetwarzania danych w czasie rzeczywistym} - przedstawia wymagania, architekturę, model danych oraz przepływ danych w projektowanym systemie
    \item \textbf{Rozdział 4: Implementacja systemu na klastrze Kubernetes} - opisuje konfigurację klastra, wdrożenie mikroserwisów, konfigurację Apache Kafka i Kafka Streams, implementację przetwarzania strumieniowego
    \item \textbf{Rozdział 5: Algorytmy analizy danych w czasie rzeczywistym} - omawia zaimplementowane algorytmy do analizy danych
    \item \textbf{Rozdział 6: Badania eksperymentalne i analiza wyników} - przedstawia metodologię testowania, scenariusze testowe, wyniki badań wydajnościowych
    \item \textbf{Rozdział 7: Zastosowania praktyczne} - omawia praktyczne zastosowania opracowanego systemu w przemyśle, korzyści ekonomiczne i operacyjne oraz wyzwania wdrożeniowe
    \item \textbf{Rozdział 8: Podsumowanie i wnioski} - zawiera podsumowanie pracy, główne osiągnięcia i wnioski, ograniczenia badań oraz kierunki przyszłych badań
\end{itemize} 