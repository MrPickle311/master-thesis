\section{WPROWADZENIE}
\label{sec:wprowadzenie}

\subsection{Kontekst i motywacja badań}
\label{subsec:kontekst}

W dobie czwartej rewolucji przemysłowej (Przemysł 4.0) i Internetu Rzeczy (IoT), systemy wieloczujnikowe stają się nieodłącznym elementem nowoczesnych procesów produkcyjnych. Generują one ogromne ilości danych, które odpowiednio wykorzystane mogą dostarczyć cennych informacji na temat stanu i wydajności monitorowanych procesów. Analiza tych danych w czasie rzeczywistym stanowi jednak wyzwanie, zarówno pod względem technicznym, jak i organizacyjnym.

Tradycyjne podejścia do przetwarzania danych, oparte na scentralizowanych systemach, często nie są w stanie efektywnie obsłużyć dużej liczby równoczesnych strumieni danych przy zachowaniu niskich opóźnień. Ponadto, skalowanie takich systemów w odpowiedzi na rosnące obciążenie może być problematyczne i kosztowne.

Architektura mikroserwisowa, w połączeniu z technologiami konteneryzacji i orkiestracji, takimi jak Kubernetes, oferuje nowe możliwości w zakresie budowy systemów do analizy danych w czasie rzeczywistym. Umożliwia ona elastyczne skalowanie, izolację usług i łatwiejsze zarządzanie złożonymi aplikacjami. Jednocześnie narzędzia Big Data, takie jak Apache Kafka, zapewniają wydajne mechanizmy do przetwarzania strumieniowego, niezbędne w analizie danych w czasie rzeczywistym.

Motywacją do podjęcia niniejszych badań jest potrzeba opracowania wydajnego, skalowalnego i odpornego na awarie systemu do analizy danych w czasie rzeczywistym z systemów wieloczujnikowych, który mógłby być wykorzystany w różnych procesach przemysłowych, ze szczególnym uwzględnieniem syntezy amoniaku (proces Habera).

\subsection{Cel i zakres pracy}
\label{subsec:cel}

Głównym celem niniejszej pracy jest projekt, implementacja i analiza wydajności systemu do przetwarzania danych w czasie rzeczywistym z wielu czujników, wykorzystując architekturę mikroserwisową opartą na klastrze Kubernetes oraz narzędzia Big Data.

Szczegółowe cele pracy obejmują:
\begin{itemize}
    \item Opracowanie architektury systemu do analizy danych w czasie rzeczywistym z systemów wieloczujnikowych, opartego na klastrze Kubernetes i narzędziach Big Data
    \item Implementacja systemu zgodnie z zaprojektowaną architekturą, z wykorzystaniem mikroserwisów i technologii przetwarzania strumieniowego
    \item Opracowanie i implementacja algorytmów do analizy danych w czasie rzeczywistym, w tym algorytmów do detekcji anomalii i przewidywania awarii
    \item Przeprowadzenie badań eksperymentalnych w celu oceny wydajności i skalowalności opracowanego systemu
    \item Analiza przydatności systemu w środowisku przemysłowym, ze szczególnym uwzględnieniem procesu syntezy amoniaku
\end{itemize}

Zakres pracy obejmuje:
\begin{itemize}
    \item Analizę istniejących rozwiązań do przetwarzania danych w czasie rzeczywistym
    \item Projekt i implementację systemu przetwarzania danych opartego na klastrze Kubernetes
    \item Opracowanie mechanizmów agregacji i analizy danych z wielu czujników
    \item Implementację wizualizacji danych w czasie rzeczywistym
    \item Testowanie wydajności i skalowalności systemu
    \item Ocenę przydatności zaimplementowanego rozwiązania w środowisku produkcyjnym
\end{itemize}

\subsection{Teza pracy}
\label{subsec:teza}

Teza niniejszej pracy brzmi: Wykorzystanie architektury mikroserwisowej opartej na klastrze Kubernetes oraz narzędzi Big Data umożliwia efektywną analizę danych w czasie rzeczywistym z systemów wieloczujnikowych, zapewniając elastyczne skalowanie, odporność na awarie oraz niskie opóźnienia przetwarzania, co jest kluczowe w zastosowaniach przemysłowych.

\subsection{Metodologia badań}
\label{subsec:metodologia}

Badania przeprowadzone w ramach niniejszej pracy opierają się na następującej metodologii:

\begin{enumerate}
    \item Analiza literatury i istniejących rozwiązań w zakresie przetwarzania danych w czasie rzeczywistym, ze szczególnym uwzględnieniem zastosowań w systemach wieloczujnikowych
    \item Projekt architektury systemu do analizy danych w czasie rzeczywistym, z wykorzystaniem klastra Kubernetes i narzędzi Big Data
    \item Implementacja systemu zgodnie z zaprojektowaną architekturą
    \item Przeprowadzenie badań eksperymentalnych, obejmujących:
    \begin{itemize}
        \item Pomiar obciążenia procesora i zajętości pamięci operacyjnej
        \item Testowanie poprawności wykonywania algorytmów pod zwiększonym obciążeniem
        \item Testowanie automatycznego skalowania systemu
        \item Pomiar przepustowości i opóźnień przetwarzania
    \end{itemize}
    \item Analiza wyników badań i formułowanie wniosków
\end{enumerate}

W badaniach wykorzystano dane symulowane, generowane przez funkcje AWS Lambda, które odzwierciedlają odczyty czujników w procesie syntezy amoniaku. Dane te obejmują pomiary temperatury, ciśnienia, przepływu oraz składu gazu (H2, N2, NH3, O2, CO2).

\subsection{Struktura pracy}
\label{subsec:struktura}

Praca składa się z następujących rozdziałów:

\begin{itemize}
    \item \textbf{Rozdział 1: Wprowadzenie} - przedstawia kontekst i motywację badań, cel i zakres pracy, tezę oraz metodologię badań
    \item \textbf{Rozdział 2: Przegląd literatury i stan wiedzy} - omawia istniejące rozwiązania w zakresie analizy danych w czasie rzeczywistym, architektury mikroserwisowej, Kubernetes oraz narzędzi Big Data
    \item \textbf{Rozdział 3: Projekt systemu przetwarzania danych w czasie rzeczywistym} - przedstawia wymagania, architekturę, model danych oraz przepływ danych w projektowanym systemie
    \item \textbf{Rozdział 4: Implementacja systemu na klastrze Kubernetes} - opisuje konfigurację klastra, wdrożenie mikroserwisów, konfigurację Apache Kafka i Kafka Streams, implementację przetwarzania strumieniowego oraz aspekty bezpieczeństwa
    \item \textbf{Rozdział 5: Algorytmy analizy danych w czasie rzeczywistym} - omawia zaimplementowane algorytmy do analizy danych, w tym agregację i statystyki, detekcję anomalii, przewidywanie awarii oraz korelację danych z wielu czujników
    \item \textbf{Rozdział 6: Badania eksperymentalne i analiza wyników} - przedstawia metodologię testowania, scenariusze testowe, wyniki badań wydajnościowych oraz porównanie z innymi rozwiązaniami
    \item \textbf{Rozdział 7: Zastosowania praktyczne} - omawia praktyczne zastosowania opracowanego systemu w przemyśle, korzyści ekonomiczne i operacyjne oraz wyzwania wdrożeniowe
    \item \textbf{Rozdział 8: Podsumowanie i wnioski} - zawiera podsumowanie pracy, główne osiągnięcia i wnioski, ograniczenia badań oraz kierunki przyszłych badań
\end{itemize} 