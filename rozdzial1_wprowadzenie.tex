\section{Wprowadzenie}
\label{sec:wprowadzenie}

\subsection{Kontekst i motywacja badań}
\label{subsec:kontekst}

W dobie czwartej rewolucji przemysłowej \cite{przemysl40_iot_ogolnie} i Internetu Rzeczy (\mbox{\textit{IoT}}, ang. \mbox{\textit{Internet of Things}}) \cite{iot_definition_aws}, definiowanego jako sieć połączonych ze sobą urządzeń fizycznych wyposażonych w czujniki i oprogramowanie umożliwiające zbieranie i wymianę informacji, systemy wieloczujnikowe stają się nieodłącznym 
elementem nowoczesnych procesów produkcyjnych. Generują one ogromne ilości danych, które odpowiednio wykorzystane mogą dostarczyć
cennych informacji na temat stanu i wydajności monitorowanych procesów. Analiza tych danych w czasie rzeczywistym stanowi jednak
wyzwanie, zarówno pod względem technicznym, jak i organizacyjnym \cite{realtime_challenges}.

Tradycyjne podejścia do przetwarzania informacji, oparte na scentralizowanych systemach, często nie są w stanie efektywnie obsłużyć
dużej liczby równoczesnych strumieni przy zachowaniu niskich opóźnień. Ponadto, skalowanie takich systemów w odpowiedzi na rosnące
obciążenie może być problematyzowane i kosztowne.

Rozproszona architektura, w połączeniu z technologiami konteneryzacji i orkiestracji, takimi jak: platforma \textit{Kubernetes} (system do zarządzania kontenerami) \cite{kubernetes_benefits},
oferuje nowe możliwości w zakresie budowy systemów do analizy danych w czasie rzeczywistym. Umożliwia ona elastyczne skalowanie,
izolację usług i łatwiejsze zarządzanie złożonymi aplikacjami. Jednocześnie narzędzia \mbox{\textit{Big Data}}, czyli technologie służące do przetwarzania i analizy dużych, złożonych zbiorów danych, takie jak: broker \mbox{\textit{Apache Kafka}} (broker do przetwarzania strumieni danych) czy silnik analityczny \mbox{\textit{Apache Spark}} (silnik do obliczeń rozproszonych) \cite{kafka, spark_streaming}, zapewniają
wydajne mechanizmy do przetwarzania strumieniowego, niezbędne w analizie informacji w czasie rzeczywistym.

Motywacją do podjęcia niniejszych badań jest potrzeba sprawdzenia możliwości do opracowania wydajnego i skalowalnego, autorskiego systemu do
analizy danych w czasie rzeczywistym z systemów wieloczujnikowych, mógłby zostać w przyszłości wykorzystany w różnych procesach przemysłowych.

\subsection{Cel i zakres pracy}
\label{subsec:cel}

Głównymi celami niniejszej pracy są: zaprojektowanie, implementacja i analiza wydajności systemu do przetwarzania danych w
czasie rzeczywistym z wielu czujników.

\vspace{0.3em}

Szczegółowe cele pracy obejmują:
\begin{itemize}
    % \setlength\itemsep{-0.3em}
    \item opracowanie architektury projektowanego systemu do analizy danych strumieniowych z systemów wieloczujnikowych, opartego na klastrze Kubernetes i narzędziach Big Data,
    \item implementacja projektowanego systemu zgodnie z zaprojektowaną architekturą, z wykorzystaniem mikroserwisów i technologii przetwarzania strumieniowego,
    \item opracowanie i implementacja algorytmów do analizy danych strumieniowych, w tym algorytmów do detekcji anomalii i przewidywania awarii,
    \item przeprowadzenie badań eksperymentalnych w celu oceny wydajności i skalowalności opracowanego systemu,
    \item analiza przydatności opracowanego systemu w środowisku przemysłowym.
\end{itemize}

\newpage

Zakres pracy obejmuje:
\begin{itemize}
    \item analizę istniejących rozwiązań do przetwarzania danych w czasie rzeczywistym,
    \item projekt i implementację autorskiego systemu przetwarzania danych opartego na klastrze \textit{Kubernetes},
    \item opracowanie mechanizmów agregacji i analizy danych z wielu czujników,
    \item implementację wizualizacji danych w czasie rzeczywistym,
    \item testowanie wydajności i skalowalności systemu,
    \item ocenę przydatności zaimplementowanego rozwiązania w środowisku produkcyjnym.
\end{itemize}

\subsection{Struktura pracy}
\label{subsec:struktura}

Praca składa się z następujących rozdziałów:

\begin{itemize}
    \item \textbf{rozdział 1: Wprowadzenie} - przedstawia kontekst i motywację badań, cel i zakres pracy oraz metodologię badań,
    \item \textbf{rozdział 2: Przegląd literatury i stan wiedzy} - omawia istniejące rozwiązania w zakresie analizy danych w czasie rzeczywistym, architektury mikroserwisowej, Kubernetes oraz narzędzi Big Data,
    \item \textbf{rozdział 3: Projekt systemu przetwarzania danych w czasie rzeczywistym} - przedstawia wymagania, architekturę, model danych oraz przepływ danych w projektowanym systemie,
    \item \textbf{rozdział 4: Implementacja systemu na klastrze Kubernetes} - opisuje konfigurację klastra, wdrożenie mikroserwisów, konfigurację Apache Kafka i Kafka Streams, implementację przetwarzania strumieniowego,
    \item \textbf{rozdział 5: Algorytmy analizy danych w czasie rzeczywistym} - omawia zaimplementowane algorytmy do analizy danych,
    \item \textbf{rozdział 6: Badania eksperymentalne i analiza wyników} - przedstawia metodologię testowania, scenariusze testowe, wyniki badań wydajnościowych,
    \item \textbf{rozdział 7: Zastosowania praktyczne} - omawia praktyczne zastosowania opracowanego systemu w przemyśle, korzyści ekonomiczne i operacyjne oraz wyzwania wdrożeniowe,
    \item \textbf{rozdział 8: Podsumowanie i wnioski} - zawiera podsumowanie pracy, główne osiągnięcia i wnioski, ograniczenia badań oraz kierunki przyszłych badań.
\end{itemize} 