\newpage
\section{Zaimplementowane składowe algorytmy autopilota}
\label{algorithms}
{
    \subsection{Algorytmy konwersji danych z czujników}
    {
        \subsubsection{Odczytywanie pozycji robota}
        {
            W skład oprogramowania autopilota wchodzi węzeł odpowiedzialny za transformację współrzędnych geograficznych na współrzędne kartezjańskie. Odczytuje on z odpowiedniego kanału (patrz rozdział \ref{soft_struct}) pozycję geograficzną robota i wykorzystuje on algorytm odwzorowania walcowego równoodległościowego, by otrzymać współrzędne kartezjańskie wyrażone w metrach. Wspomniany algorytm został opisany w rozdziale \ref{gps_to_xy_transform}. By uprościć obliczenia, to długość geograficzną referencyjną przyjęto jako długość geograficzną w środku mapy \textbf{$\phi_0$}, czyli \textbf{$\phi_1 = \phi_0$}. 
        }
        \subsubsection{Odczytywanie orientacji robota}
        {
            W skład oprogramowania autopilota również wchodzi węzeł odpowiedzialny za transformację danych z magnetometru na aktualną orientację robota. Jako podstawy użyto algorytmu konwersji przedstawionego w rozdziale \ref{mag_to_yaw_transform}. Znaczenie symboli w tym rozdziale jest takie samo jak~w~rozdziale \ref{mag_to_yaw_transform}. Postać algorytmu opisanego w rozdziale \ref{mag_to_yaw_transform} jest niepraktyczna w dalszych obliczeniach. By ujednolicić zapis użyto funkcji $atan2$. 
            
            \begin{equation}
                    \phi_R= \arctan2(H_x,H_y,) \label{eq:1}
            \end{equation}
            
            Ta funkcja zwraca wartości z zakresu ($-\pi$,$\pi$). Jednak wygodniej było by korzystać z wartości z zakresu ($0$,$2\pi$). Dzięki wykorzystaniu dzielenia z resztą przez $2\pi$, poniższa funkcja spełnia wspomniane oczekiwania. 
            
            \begin{equation}\label{eq:1}
                \phi_R=\mod(\arctan2(H_x,H_y,)+ 2\pi, 2\pi)
            \end{equation}

            Węzeł odczytuje surowe dane z magnetometru i wykorzystując zaimplementowany w Pythonie wzór \ref{eq:1} publikuje aktualną orientację robota wyrażoną w radianach.
        }
    }
    \subsection{Algorytm wykonywania misji}
    {
        Algorytm wykonywania misji wykorzystuje algorytm skrętu oraz dojazdu do punktu wraz~z~wymaganą dokładnością, by zrealizować misję przesłaną w pliku. Ostatecznie doprowadza on robota do ostatniej lokalizacji zapisanej w pliku. Na wejściu przyjmuje on lokalizację pliku z misją, odczytuje go i wykonuję tę misję.
        Na rysunkach \ref{Schemat algorytmu wykonywania misji} oraz \ref{Schemat podróży robota} znajdują się graficzne przedstawienia tego algorytmu.
        \singlesizedimage{images/main_alg.png}{Schemat algorytmu wykonywania misji}{0.6}

        \singlesizedimage{images/mission_algorithm_visualisation.png}{Schemat podróży robota}{0.75}
    }
    \subsection{Algorytm skrętu wraz z wymaganą dokładnością}
    {
        \label{rot_algorithm}
        Zadaniem tego algorytmu jest obrócenie robota, tak by jego oś \textbf{x} pokryła się~z~kierunkiem wektora przemieszczenia pomiędzy aktualnym punktem a punktem następnym z wymaganą dokładnością podaną w stopniach.

        Najpierw następuje obliczenie wspomnianego wektora przemieszczenia $v_{disp}$ z punktu aktualnego do punktu następnego. Znając aktualną orientację robota oraz $v_{disp}$ następuje obliczenie całkowitego wymaganego kąta potrzebnego do obrotu, oraz kierunku tego obrotu. Opisany algorytm opiera się na wzorach \ref{rot_eqtn_1} oraz \ref{rot_eqtn_2}.

        \singlesizedimage{images/rotate_alg.png}{Schemat algorytmu skrętu robota}{0.75}

        \begin{equation}\label{rot_eqtn_1}
            \phi_2^1=\phi_{2}^{0} - \phi_1^0
        \end{equation}
        \begin{equation}\label{rot_eqtn_2}
            \phi_{2}^{0}=\arctan2(v_{disp_y},v_{disp_x})
        \end{equation}

        Wartość $\phi_1^0$ odczytuje się z kompasu. Jednak jest obecny tutaj obrót z pewną dokładnością co do kąta. Jest to problematyczne do zrealizowania, gdyż błąd 2-3$^o$ jest dużym błędem, często niepozwalającym na prawidłowe dotarcie do celu. Rozwiązanie tego problemu przedstawiono w rozdziale \ref{compass_analysis}.

        Obrót jest sterowany parametrami i te parametry są podane w pliku misji obok współrzędnych docelowego punktu. Wszystkie parametry użyte w pliku misji zostały przedstawione w rozdziale \ref{mission_file}. 
        Obrót robotem odbywa się za pomocą wymuszenia prędkościowego. Zadaje się pewną prędkość kątową, by robot znalazł się w danej orientacji absolutnej. 
        Obrót robota składa się z trzech faz, które przedstawiono również na wykresie \ref{Schemat rozłożenia wymuszenia prędkościowego w czasie dla skrętu robota}

        \begin{itemize}
            \item stopniowego rozwijania prędkości kątowej robota,
            \item obrotu robota ze stałą prędkością kątową,
            \item stopniowej redukcji prędkości kątowej robota.
        \end{itemize}

        \singlesizedimage{images/rot_chart.png}{Schemat rozłożenia wymuszenia prędkościowego w czasie dla skrętu robota}{1.0}
        gdzie:\\
        $\phi_{traveled}$ -- moduł przebytego już kąta,\\
        $abs(\phi_2^1)$ -- moduł całkowitego brakującego kąta,\\
        $\frac{\phi_{traveled}}{abs(\phi_2^1)}$ -- postęp wykonywania się algorytmu,\\
        $\omega_{max}$ -- maksymalna prędkość kątowa,\\
        $K_2$ -- współczynnik oznaczający postęp, przy którym robot przestaje zwiększać swoją prędkość kątową,\\
        $K_3$ -- współczynnik oznaczający postęp, przy którym robot zaczyna redukować swoją prędkość kątową.\\

        Maksymalna prędkość kątowa $\omega_{max}$ jest wyrażona wzorem \ref{max_omega}.
        \begin{equation}\label{max_omega}
            \omega_{max}=K_1 \cdot abs(\phi_2^1)
        \end{equation}
        gdzie,\\
        $K_1$ -- współczynnik oznaczający ograniczający maksymalną prędkość kątową\\
        
        Sterowanie prędkością kątową w fazach redukcji i rozwijania prędkości kątowej odbywa się~za~ pomocą pętli sprzężenia zwrotnego. Ta pętla jest skwantyzowana, to znaczy, że wymuszenie nie jest nadawane w sposób ciągły, tylko dopiero po osiągnięciu danego postępu w obrocie. Wielkość ziarnistości kwantyzacji $N$ podaje się również w pliku misji. Ta liczba oznacza liczbę zmian prędkości podczas wykonywania się faz stopniowego rozwijania prędkości kątowej robota oraz stopniowej redukcji prędkości kątowej robota.
        
        Sterowanie brakującym kątem odbywa się jednak w sposób ciągły. 
        Algorytm jest zaprojektowany tak, że robot zawsze wykonuje obrót mniejszy niż $\pi$, czyli w celu obrotu do punktu wybiera bliższą trasę kątową.
    }
    \subsection{Algorytm dojazdu do punktu wraz z wymaganą dokładnością}
    {
        \label{disp_algorithm}
        Zadaniem tego algorytmu jest przemieszczenie robota, tak by jego położenie pokryło się~z~punktem docelowym z wymaganą dokładnością podaną w metrach.
        Najpierw następuje obliczenie wektora przemieszczenia $v_{disp}$ (patrz rysunek \ref{Schemat algorytmu skrętu robota}) z aktualnego punktu do następnego punktu. Jednak obecny jest tutaj dojazd do punktu z pewną dokładnością co do pewnego błędu wyrażonego~w~metrach. Jest to pewne wyzywanie, gdyż już po starcie robot jest obarczony pewnym błędem orientacji. Dodatkowo skala problemu jest zwiększana przez niedokładność odbiornika GPS. Dlatego punkty należy rozsiewać gęsto lub zmniejszać wymaganą dokładność.

        Dojazd do punktu jest sterowany pewnymi parametrami i te parametry są podane w pliku misji obok współrzędnych docelowego punktu. Wszystkie parametry w pliku misji zostały przedstawione w rozdziale \ref{mission_file}. Ruch translacyjny robota odbywa się za pomocą wymuszenia prędkościowego liniowego. Dojazd robota do punktu składa się z trzech faz, które zostały również przedstawione na wykresie \ref{Schemat rozłożenia wymuszenia prędkościowego w czasie dla algorytmu dojazdu do punktu}:

        \begin{itemize}
            \item stopniowego rozwijania prędkości liniowej robota,
            \item jazdy robota z stałą prędkością liniową,
            \item stopniowej redukcji prędkości liniowej robota.
        \end{itemize}

        \singlesizedimage{images/disp_chart.png}{Schemat rozłożenia wymuszenia prędkościowego w czasie dla algorytmu dojazdu do punktu}{1.0}
        gdzie,\\
        $v_{traveled}$ -- przebyty dystans\\
        $|v_{disp}|$ -- odległość między punktami\\
        $\frac{|v_{traveled}|}{|v_{disp}|}$ -- postęp wykonywania się algorytmu\\
        $v_{max}$ -- maksymalna prędkość liniowa\\
        $K_2$ -- współczynnik oznaczający postęp, przy którym robot przestaje zwiększać swoją prędkość liniową\\
        $K_3$ -- współczynnik oznaczający postęp, przy którym robot zaczyna redukować swoją prędkość liniową.\\

        Sterowanie prędkością liniową w fazach stopniowego rozwijania prędkości liniowej robota~i~stopniowej redukcji prędkości liniowej robota odbywa się za pomocą pętli sprzężenia zwrotnego. Ta pętla jest skwantyzowana, to znaczy, że wymuszenie nie jest nadawane w sposób ciągły, tylko dopiero po osiągnięciu danego postępu w obrocie. Wielkość ziarnistości kwantyzacji $N$ podaje się również~w~pliku misji. Ta liczba oznacza liczbę zmian prędkości podczas wykonywania się faz redukcji oraz rozwijania prędkości liniowej.
        Sterowanie brakującą odległością odbywa się jednak~w~sposób ciągły. Maksymalna prędkość liniowa $v_{max}$ jest wyrażona wzorem \ref{max_v}.
        
        \begin{equation}\label{max_v}
            v_{max}=K_1 \cdot |v_{disp}|
        \end{equation}
        gdzie:\\
        $K_1$ -- współczynnik oznaczający ograniczający maksymalną prędkość liniową\\
    }
    \newpage
    \subsection{Opis pliku misji}
    {
    \label{mission_file}
    W pliku misji znajdują się kolejne punkty, do których robot ma dotrzeć wraz z podanymi parametrami trasy takimi jak na przykład dokładność. Współrzędne są podane jako współrzędne geograficzne. Po przygotowaniu zestawu punktów, plik należy następnie załadować do autopilota, który po pomyślnym załadowaniu go zacznie wykonywać misję. Znaczenie parametrów użytych~w~pliku zostało opisane w rozdziałach~\ref{rot_algorithm}~i~\ref{disp_algorithm}

    \vspace{2mm}
    
    \begin{lstlisting}[caption=Przykładowy plik misji]
        52.4535363,21.9054322,0.1,0.2,0.8,10,0.3,1
        52.4538811,21.9039300,0.1,0.2,0.8,10,0.3,1
        52.4532222,21.9036366,0.1,0.2,0.8,10,0.3,1
        52.4588843,21.9035553,0.1,0.2,0.8,10,0.3,1
        52.4576575,21.9035444,0.1,0.2,0.8,10,0.3,1
        52.4545334,21.9022263,0.1,0.2,0.8,10,0.3,1
        52.4556423,21.9035363,0.1,0.2,0.8,10,0.3,1
        52.4536544,21.9034331,0.1,0.2,0.8,10,0.3,1
    \end{lstlisting}

    \vspace{2mm}
    
    Parametry są ułożone w następującej kolejności:
    \begin{itemize}
            \item szerokość geograficzna,
            \item długość geograficzna,
            \item współczynnik K1,
            \item współczynnik K2,
            \item współczynnik K3,
            \item stopień kwantyzacji zmian prędkości,
            \item precyzja dojazdu do punktu wyrażona w metrach,
            \item precyzja obrotu wyrażona w stopniach.
    \end{itemize}
    }
}