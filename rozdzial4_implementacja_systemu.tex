\section{Implementacja systemu przetwarzania danych w czasie rzeczywistym}
\label{sec:implementacja_systemu}

W niniejszym rozdziale przedstawiono szczegóły implementacji systemu do analizy danych w czasie rzeczywistym z systemów wieloczujnikowych,
zgodnie z projektem opisanym w poprzednim rozdziale.

\subsection{Wykorzystane technologie}
\label{subsec:technologie}

W stworzonej autorskiej implementacji zastosowane zostały przedstawione poniżej kluczowe technologie.

\subsubsection{Języki programowania}
\label{subsubsec:jezyki_programowania}

\begin{itemize}
    \item \textbf{Java 21} - główny język programowania wykorzystany do implementacji mikroserwisów (np. \texttt{FrontService}, \texttt{DataService}),
    \item \textbf{Scala} - język programowania wykorzystany do implementacji aplikacji \texttt{SparkDataProcessor} służącej do przetwarzania strumieni danych za pomocą Apache Spark,
    \item \textbf{Python 3.9} - język wykorzystany do implementacji symulatorów czujników i skryptów pomocniczych (np. do trenowania modelu ML),
    \item \textbf{TypeScript} - język wykorzystany do implementacji interfejsu użytkownika.
\end{itemize}

\subsubsection{Frameworki i biblioteki}
\label{subsubsec:frameworki}

\begin{itemize}
    \item \textbf{Spring Boot 3.4} - framework wykorzystany do implementacji mikroserwisów Java,
    \item \textbf{Apache Spark 3.x} \cite{spark_streaming} - framework wykorzystany do przetwarzania strumieniowego danych (Spark Structured Streaming) oraz do zadań uczenia maszynowego (Spark ML) w aplikacji \texttt{SparkDataProcessor},
    \item \textbf{Spring Cloud} - zestaw narzędzi do budowy aplikacji chmurowych,
    \item \textbf{React 18} - biblioteka JavaScript wykorzystana do budowy interfejsu użytkownika.
\end{itemize}

\subsubsection{Bazy danych i systemy magazynowania}
\label{subsubsec:bazy_danych}

\begin{itemize}
    \item \textbf{Apache Kafka 3.4} \cite{kafka} - platforma do przetwarzania strumieniowego,
    \item \textbf{PostgreSQL 15} - relacyjna baza danych,
    \item \textbf{Elasticsearch 8.7} - baza danych NoSQL \cite{nosql_definition} do wyszukiwania i analizy,
    \item \textbf{Redis 7.0} - baza danych in-memory wykorzystywana jako cache.
\end{itemize}

\subsubsection{Infrastruktura i orkiestracja}
\label{subsubsec:infrastruktura}

\begin{itemize}
    \item \textbf{Kubernetes 1.26} \cite{kubernetes} - platforma do orkiestracji kontenerów,
    \item \textbf{Docker} - platforma konteneryzacji,
    \item \textbf{Helm 3} - menedżer pakietów dla Kubernetes,
    \item \textbf{AWS} (Amazon Web Services) \cite{aws_definition} - chmura obliczeniowa wykorzystana do hostowania usług zewnętrznych,
    \item \textbf{Terraform} - narzędzie do zarządzania infrastrukturą jako kod (IaC, ang. Infrastructure as Code) \cite{terraform_docs}.
\end{itemize}

\subsection{Implementacja symulatora czujników}
\label{subsec:implementacja_symulatora}

\subsubsection{Architektura symulatora}
\label{subsubsec:architektura_symulatora}

Architektura symulatora składa się z kilku kluczowych komponentów:

\begin{itemize}
    \item \textbf{AWS Step Functions} - usługa, która koordynuje wykonywanie funkcji Lambda, zarządzając przepływem pracy i częstotliwością generowania danych,
    \item \textbf{AWS Lambda} \cite{aws_lambda_docs} - usługa bezserwerowa, która wykonuje kod symulatora,
    \item \textbf{AWS SNS} (Amazon Simple Notification Service) \cite{sns_docs} - usługa powiadomień, która działa jako punkt dystrybucji dla wygenerowanych danych,
    \item \textbf{AWS SQS} (Amazon Simple Queue Service) \cite{sqs_docs} - usługa kolejek, która buforuje wiadomości przed ich przetworzeniem przez system.
\end{itemize}

%Schemat architektury symulatora przedstawiono na Rysunku \ref{fig:architektura_symulatora}.
%
%\begin{figure}[h]
%    \centering
%    % Tutaj powinien być diagram architektury symulatora
%    \caption{Architektura symulatora czujników}
%    \label{fig:architektura_symulatora}
%\end{figure}

\subsubsection{Implementacja funkcji Lambda}
\label{subsubsec:implementacja_lambda}

Funkcje Lambda zostały zaimplementowane w języku Python 3.9. Każda funkcja generuje dane dla określonego typu czujnika,
symulując jego zachowanie w różnych warunkach pracy. Symulator uwzględnia specyfikę każdego typu czujnika i jego charakterystykę w kontekście procesu syntezy amoniaku.

Dla każdego typu czujnika zaimplementowano logikę generowania realistycznych danych, uwzględniającą:
\begin{itemize}
    \item normalne wahania wartości w zakresie typowym dla danego parametru,
    \item możliwość wystąpienia anomalii z określonym prawdopodobieństwem,
    \item korelacje między różnymi parametrami procesu,
    \item symulację zakłóceń i szumów pomiarowych.
\end{itemize}