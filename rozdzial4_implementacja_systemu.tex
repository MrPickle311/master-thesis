\section{IMPLEMENTACJA SYSTEMU PRZETWARZANIA DANYCH W CZASIE RZECZYWISTYM}
\label{sec:implementacja_systemu}

W niniejszym rozdziale przedstawiono szczegóły implementacji systemu do analizy danych w czasie rzeczywistym z systemów wieloczujnikowych, zgodnie z projektem opisanym w poprzednim rozdziale. Omówiono wykorzystane technologie, implementację poszczególnych komponentów oraz wdrożenie systemu na klastrze Kubernetes.

\subsection{Wykorzystane technologie}
\label{subsec:technologie}

Implementacja systemu opiera się na wielu nowoczesnych technologiach, które umożliwiają efektywne przetwarzanie danych w czasie rzeczywistym. Kluczowe technologie wykorzystane w implementacji to:

\subsubsection{Języki programowania}
\label{subsubsec:jezyki_programowania}

\begin{itemize}
    \item \textbf{Java 17} - główny język programowania wykorzystany do implementacji mikroserwisów przetwarzających dane.
    \item \textbf{Python 3.9} - język wykorzystany do implementacji symulatorów czujników i skryptów pomocniczych.
    \item \textbf{TypeScript} - język wykorzystany do implementacji interfejsu użytkownika.
\end{itemize}

\subsubsection{Frameworki i biblioteki}
\label{subsubsec:frameworki}

\begin{itemize}
    \item \textbf{Spring Boot 3.1} - framework wykorzystany do implementacji mikroserwisów.
    \item \textbf{Kafka Streams 3.4} - biblioteka wykorzystana do przetwarzania strumieniowego.
    \item \textbf{Spring Cloud} - zestaw narzędzi do budowy aplikacji chmurowych.
    \item \textbf{React 18} - biblioteka JavaScript wykorzystana do budowy interfejsu użytkownika.
    \item \textbf{D3.js} - biblioteka JavaScript wykorzystana do wizualizacji danych.
    \item \textbf{Matplotlib i Pandas} - biblioteki Pythona wykorzystane w analizie danych i generowaniu wykresów.
\end{itemize}

\subsubsection{Bazy danych i systemy magazynowania}
\label{subsubsec:bazy_danych}

\begin{itemize}
    \item \textbf{Apache Kafka 3.4} - platforma do przetwarzania strumieniowego.
    \item \textbf{PostgreSQL 15} - relacyjna baza danych.
    \item \textbf{Elasticsearch 8.7} - baza danych NoSQL do wyszukiwania i analizy.
    \item \textbf{Redis 7.0} - baza danych in-memory wykorzystywana jako cache.
\end{itemize}

\subsubsection{Infrastruktura i orkiestracja}
\label{subsubsec:infrastruktura}

\begin{itemize}
    \item \textbf{Kubernetes 1.26} - platforma do orkiestracji kontenerów.
    \item \textbf{Docker} - platforma konteneryzacji.
    \item \textbf{Helm 3} - menedżer pakietów dla Kubernetes.
    \item \textbf{AWS} - chmura obliczeniowa wykorzystana do hostowania usług zewnętrznych.
    \item \textbf{Terraform} - narzędzie do zarządzania infrastrukturą jako kod.
\end{itemize}

\subsubsection{Narzędzia monitorowania i logowania}
\label{subsubsec:monitorowanie}

\begin{itemize}
    \item \textbf{Prometheus} - system monitorowania.
    \item \textbf{Grafana} - platforma do wizualizacji danych monitorowania.
    \item \textbf{ELK Stack} - zestaw narzędzi do zbierania, analizy i wizualizacji logów.
    \item \textbf{Jaeger} - system do śledzenia rozproszonego.
\end{itemize}

\subsection{Implementacja symulatora czujników}
\label{subsec:implementacja_symulatora}

Symulator czujników został zaimplementowany jako zestaw funkcji AWS Lambda, które generują dane symulujące odczyty z różnych typów czujników. Funkcje te są wyzwalane periodycznie przez AWS Step Functions, co umożliwia kontrolowane generowanie danych z określoną częstotliwością.

\subsubsection{Architektura symulatora}
\label{subsubsec:architektura_symulatora}

Architektura symulatora składa się z kilku kluczowych komponentów:

\begin{itemize}
    \item \textbf{AWS Step Functions} - usługa, która koordynuje wykonywanie funkcji Lambda, zarządzając przepływem pracy i częstotliwością generowania danych.
    \item \textbf{AWS Lambda} - usługa bezserwerowa, która wykonuje kod symulatora.
    \item \textbf{AWS SNS} - usługa powiadomień, która działa jako punkt dystrybucji dla wygenerowanych danych.
    \item \textbf{AWS SQS} - usługa kolejek, która buforuje wiadomości przed ich przetworzeniem przez system.
\end{itemize}

Schemat architektury symulatora przedstawiono na Rysunku \ref{fig:architektura_symulatora}.

\begin{figure}[h]
    \centering
    % Tutaj powinien być diagram architektury symulatora
    \caption{Architektura symulatora czujników}
    \label{fig:architektura_symulatora}
\end{figure}

\subsubsection{Implementacja funkcji Lambda}
\label{subsubsec:implementacja_lambda}

Funkcje Lambda zostały zaimplementowane w języku Python 3.9. Każda funkcja generuje dane dla określonego typu czujnika, symulując jego zachowanie w różnych warunkach pracy. Symulator uwzględnia specyfikę każdego typu czujnika i jego charakterystykę w kontekście procesu syntezy amoniaku.

Dla każdego typu czujnika zaimplementowano logikę generowania realistycznych danych, uwzględniającą:
\begin{itemize}
    \item Normalne wahania wartości w zakresie typowym dla danego parametru
    \item Możliwość wystąpienia anomalii z określonym prawdopodobieństwem
    \item Korelacje między różnymi parametrami procesu
    \item Symulację zakłóceń i szumów pomiarowych
    \item Różne scenariusze awarii i nieprawidłowości
\end{itemize}

\subsubsection{Konfiguracja AWS Step Functions}
\label{subsubsec:konfiguracja_step_functions}

AWS Step Functions zostało skonfigurowane do koordynacji wykonywania funkcji Lambda dla różnych typów czujników. Konfiguracja obejmuje:
\begin{itemize}
    \item Definicję przepływu pracy dla każdego typu czujnika
    \item Harmonogram wykonywania funkcji
    \item Obsługę błędów i ponownych prób
    \item Równoległe wykonywanie symulatorów dla różnych czujników
    \item Monitorowanie i raportowanie stanu wykonania
\end{itemize}