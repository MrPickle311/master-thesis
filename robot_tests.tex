\newpage
\section{Testy oprogramowania}\label{tests}
{
    \subsection{Przeprowadzone testy}\label{pre_tests}
    {
        W celu przetestowania i zweryfikowania poprawności działania oprogramowania przygotowano dwa testy. Elementy oprogramowania są dobrane wg opisów w rozdziałach \ref{soft_choice} oraz \ref{soft_struct}. Testy przeprowadzono w środowisku symulacyjnym. W pierwszym teście sprawdzono ogólną poprawność działania oprogramowania. W drugim teście natomiast sprawdzono dokładność dojazdu.

        % wynik: czas: 05:16 minut.
        % x: 0.29 m
        % y: -0.175 m
        
        Podczas pierwszego testu sprawdzono poprawność działania oprogramowania. Trasę pierwszego testu można zobaczyć na rysunku \ref{Mapa przedstawiajaca pierwszą trasę}. Robot ma za zadanie pokonać wyznaczoną trasę i wrócić na miejsce startowe. Robot podczas symulacji pokonał wyznaczoną trasę. Zakończył on misję~w~odległości 0,339 m od punktu startowego. 
        Długość i szerokość robota wynoszą 0,5 m, zatem błąd jest mniejszy niż jego rozmiar.
        Pierwszy test potwierdził poprawność wykonywania misji.
        
        Podczas drugiego testu sprawdzono dokładność dojazdu robota do poszczególnych punktów. Drugą trasę pokazano na rysunku \ref{Mapa przedstawiajaca drugą trasę}. Robot również pokonał tę trasę i wrócił na miejsce startowe. W tabeli \ref{second_mission_table} pokazano wyniki tego testu. Obliczone współrzędne oznaczono jako (x, y), natomiast współrzędne, na których robot się zatrzymał, oznaczono jako (x', y'). W ostatniej kolumnie zaprezentowany został błąd pozycji. Jest to odległość między docelowym obliczonym punktem~a~punktem, do którego robot dojechał.

        % podczas testów przyjęto, że robot za każdym razem będzie startować z 
        % base_latt: 52.220395
        % base_long: 21.010866

        % JAKIE PARAMETRY?

        \begin{table}[h!]
            \centering
            \begin{tabular}{|l|l|l|l|l|l|}
                \hline
                l.p & x {[}m{]} & x' {[}m{]} & y {[}m{]} & y' {[}m{]} & Błąd {[}m{]} \\ \hline
                1   & -13.358   & -13.055    & 7.024     & -7.007     & 0.303 \\ \hline
                2   & -37.626   & -37.279    & -6.411    & -8.789     & 2.403 \\ \hline
                3   & -68.461   & -68.607    & 8.661     & 7.785      & 0.888 \\ \hline
                4   & -67.460   & -67.935    & 30.007    & 29.746     & 0.542 \\ \hline
                5   & -12.245   & -12.661    & 28.370    & 25.422     & 2.977 \\ \hline
                6   & -0.334    & -0.056     & 14.253    & 14.997     & 0.794 \\ \hline
                7   &  0.000       & 0.519    &  0.000   & 0.360      & 0.632 \\ \hline
            \end{tabular}
            \caption{Tabelka prezentująca wyniki z pierwszej misji}
            \label{second_mission_table}
        \end{table}

        Patrząc na wiersze 2 oraz 5 w tabeli \ref{second_mission_table} można z zauważyć, że robot po zakończeniu dojazdu był oddalony odpowiednio około 2.5 m oraz 3 m. Takie zachowanie należy skorygować. Można to rozwiązać, modyfikując algorytm wykonywania misji w taki sposób, by po dojeździe do zadanego punktu sprawdzał on swoją odległość od obliczonego punktu. Jeśli będzie ona większa niż z góry zadana (ustalana przez użytkownika), to powinien on jeszcze raz wykonać podróż do tego samego punktu.

        \begin{table}[h!]
            \centering
            \begin{tabular}{|l|l|l|l|l|l|}
                \hline
                l.p & x {[}m{]} & x' {[}m{]} & y {[}m{]} & y' {[}m{]} & Błąd {[}m{]} \\ \hline
                1   & -13.358   & -13.055    & 7.024     & -7.080    & 0.368 \\ \hline
                2   & -37.626   & -37.336    & -6.411    & -7.425     & 0.300 \\ \hline
                3   & -68.461   & -68.332    & 8.661     & 8.937      & 0.305 \\ \hline
                4   & -67.460   & -67.718    & 30.007    & 29.721     & 0.385 \\ \hline
                5   & -12.245   & -12.159    & 28.370    & 28.059    & 0.323 \\ \hline
                6   & -0.334    & 0.118      & 14.253    & 15.062     & 0.926 \\ \hline
                7   & 0.000     & 0.426      & 0.000     & 15.062      & 0.521 \\ \hline
            \end{tabular}
            \caption{Tabelka prezentująca wyniki z drugiej misji}
            \label{second_mission_table_better}
        \end{table}

        \newpage
        W tabeli \ref{second_mission_table_better} zaprezentowano wyniki testu poprawionego algorytmu. W poprawce do algorytmu przyjęto założenie, że jeśli robot jest dalej niż 1 m od obliczonego punktu, to podróż do niego należy powtórzyć. Wyniki są zadowalające. W większości przypadków otrzymano błąd około 0,3 m, co jest zadowalające.

        \singlesizedimage{images/path1.png}{Mapa przedstawiajaca pierwszą trasę}{0.7}
        \singlesizedimage{images/mission2.png}{Mapa przedstawiajaca drugą trasę}{0.5}
    }
    
    \newpage
    \subsection{Wynik eksperymentów}
    {
        Wyniki eksperymentów potwierdziły wypełnianie założeń dot. oprogramowania autopilota opisanych w rozdziale \ref{introduction}. Posługując się odczytami z czujników robot w sposób prawidłowy obliczył odpowiednie wektory przemieszczenia, kąty obrotu i podążał do odpowiednich punktów. Węzeł dokonujący pomiarów z kompasu w sposób prawidłowy określał orientację robota, wtedy gdy była ona wymagana. Węzeł dokonujący pomiarów pozycji również odczytał pozycję robota w sposób prawidłowy z zadaną dokładnością. Również komunikacja pomiędzy węzłami przebiegła w sposób zgodny z oczekiwaniami. Dobrane filtry również spełniły swoje zadanie w wystarczający sposób. Robot obecny w symulacji przebył całą wytyczoną trasę docierając do punktów z wymaganą dokładnością.
    }
}