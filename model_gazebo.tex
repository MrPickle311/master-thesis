\newpage
\section{Model robota}\label{robot_model}
{
    \subsection{Opis układów współrzędnych}\label{coordinates}
    {
        Cały system jest wyposażony w 4 układy odniesienia. To według nich dokonywane są obliczenia podczas wykonywania się algorytmów.
        \begin{itemize}
            \item Układ globalny zerowy znajdujący się na przecięciu równika oraz południka zerowego.
            \item Układ lokalny zerowy znajdujący się w miejscu rozpoczęcia misji.
            \item Układ robota.
            \item Układ punktu, do którego robot podąża.
        \end{itemize}
        
        \singlesizedimage{images/coordinate systems.png}{Układy współrzędnych w systemie}{0.7}
        
        \newpage
        
        Wszystkie układy są układami prawoskrętnymi. Oś Z nie była prezentowana na schemacie, by utrzymać jego prostotę i klarowność.
        Układ zerowy jest układem zerowym globalnym. Jest to jedyny układ, który jest nieruchomy przez cały czas.
        Jego obie współrzędne geograficzne są równe zeru.
        
        Układ lokalny zerowy jest układem, którego współrzędne geograficzne są równe współrzędnym, pod jakimi robot znajdzie się podczas rozpoczynania misji. Wektor wodzący od układu początku globalnego, do początku tego układu wynosi $v_0$. Jest on obrócony względem układu globalnego~o~90$^o$. Zapewnia to zgodność orientacji symulacji oraz prawdziwego robota.
        
        Układ robota jest związany z modelem robota. Robot zawsze skierowany wzdłuż swojej osi x. Wektor wodzący od układu początku lokalnego zerowego, do początku tego układu wynosi $v_1$.
        Układ docelowego punktu jest związany z aktualnym celem robota. Początek tego układu znajduje się tam, gdzie są współrzędne geograficzne docelowego punktu. Jego oś X leży na osi X układu robota, kąt między nimi wynosi 0. Wektor wodzący od układu początku układu robota, do początku tego układu określony jest jako $v_2$.
        Jeśli robot porusza się do punktu, to równanie przedstawiające podróż do punktu przedstawia się następująco.
        
        \begin{equation}
            v_0^{(world)}=[x_{0} \;\; y_{0} \;\; 0]^T
        \end{equation}
        
        \begin{equation}
            v_1^{(1)}=v_{robot}^{world} - v_0^{(world)}
        \end{equation}
        
        Wektor $v_2$ jest wektorem, który jest różnicą trasy pomiędzy układem robota oraz punktu
        
        \begin{equation}
            v_2^{(robot)}=r_{punkt}^{world} - r_{robot}^{world}
        \end{equation}
        
        Układy robota i zerowe są obrócone między sobą o stały kąt $\phi_{0,robot}$
        
        \begin{equation}
            R^{0}_{robot}(\phi_{0,robot}) = const
        \end{equation}
        
        Układy robota oraz punktu mają tę samą orientację.
        
        \begin{equation}
            R^{robot}_{punkt}(t) = I_{3x3}
        \end{equation}
        
        Układy globalny zerowy oraz lokalny zerowy są obrócone względem siebie o 90$^o$
        
        \begin{equation}
            R^{robot}_{punkt}(\frac{\pi}{2}) = const
        \end{equation}
        
        Współrzędne $x_{0}$, $y_{0}$, współrzędne robota $r_{robot}^{world}$  oraz punktu docelowego $r_{punkt}^{world}$ są odczytywane z GPS i znane na początku obliczeń.
    }
    \subsection{Opis modelu robota wykorzystanego w symulacji autopilota}
    {
        W symulacji użyto modelu czterokołowego robota mobilnego. Model składa się z 5 elementów: korpusu głównego oraz czterech kół. Koła obracają się wokół swoich własnych osi \textbf{y}. Jest to prosty model,  wystarcza do przeprowadzenia testów oprogramowania autopilota. Współczynnik tarcia pomiędzy kołami a nawierzchnią ustawiono na 0,8. Maksymalna prędkość liniowa to $1 \frac{m}{s}$, a prędkość kątowa $1 \frac{rad}{s}$. Masa całego robota w symulacji wynosi 15 kg.
    }
    % \newpage
    % \subsection{Fizyczne aspekty modelu robota wykorzystanego w symulacji}
    % {
    % WZORY JAK WYLICZAM TO
    % MASA, MOMENTY BEZWLADNOSCI itd...
    % }
}