\section{Zastosowania praktyczne bazujące na opracowanym systemie}
\label{sec:zastosowania_praktyczne}

W niniejszym rozdziale przedstawiono praktyczne zastosowania opracowanego w ramach tej pracy rozwiązania analitycznego z
systemów wieloczujnikowych w różnych sektorach przemysłu. Omówiono korzyści ekonomiczne i operacyjne wynikające z wdrożenia oprogramowania
oraz wyzwania napotkane podczas implementacji w rzeczywistych środowiskach produkcyjnych. Przedstawiony w pracy system można rozszerzyć o dodatkowe funkcje, które pozwolą na zastosowanie go w różnych sektorach przemysłu.

\subsection{Przypadki użycia w przemyśle}
\label{subsec:przypadki_uzycia}

Opracowany w ramach niniejszej pracy system może znaleźć zastosowanie w kilku gałęziach przemysłu, gdzie analiza danych strumieniowych przynosi wymierne korzyści.

\subsubsection{Przemysł motoryzacyjny}
\label{subsubsec:przemysl_motoryzacyjny}

W fabryce produkującej komponenty automotive opracowany system mógłby zostać wdrożony do monitorowania i optymalizacji linii produkcyjnej, w szczególności maszyn zawierających elementy takie jak: pompy, sprężarki czy silniki. 

\vspace{0.3em}

Oprogramowanie mogłoby być wykorzystywane do takich zadań jak:

\begin{itemize}
    \item \textbf{monitorowanie stanu maszyn} - system zbierałby dane z czujników zainstalowanych na kluczowych maszynach (np. prasy hydrauliczne, roboty spawalnicze, systemy transportu bliskiego),
    analizując w czasie rzeczywistym parametry takie jak: temperatura, wibracje, ciśnienie i wilgotność, analogicznie do danych generowanych w systemie opisanym w rozdziale \ref{sec:implementacja_generowania},
    \item \textbf{klasyfikacja stanu technicznego} - na podstawie historycznych danych oraz wzorców identyfikowanych przez modele uczenia maszynowego (np. \textit{RandomForestClassifier}, jak opisano w rozdziale \ref{sec:algorytmy_analizy}), oprogramowanie przewidywałoby zmiany stanu technicznego maszyn,
    umożliwiając zaplanowanie konserwacji predykcyjnej i minimalizację nieplanowanych przestojów,
    \item \textbf{kontrola jakości} - analiza danych z czujników w procesie produkcyjnym pozwalałaby na wczesne wykrywanie odchyleń parametrów, mogących prowadzić do defektów
    produktów, system alarmowałby operatorów, gdy parametry procesu zbliżają się do wartości krytycznych.
\end{itemize}

\subsubsection{Przemysł energetyczny}
\label{subsubsec:przemysl_energetyczny}

W elektrowniach, zarówno konwencjonalnych, jak i wykorzystujących odnawialne źródła energii, opracowany system znalazłby zastosowanie do monitorowania kluczowych komponentów, takich jak: pompy, turbiny czy generatory. 

\vspace{0.3em}

W tym przypadku oprogramowanie mogłoby być wykorzystywane do takich zadań jak:

\begin{itemize}
    \item \textbf{optymalizacja pracy urządzeń} - analiza danych z czujników (temperatura łożysk, wibracje, ciśnienie w układach hydraulicznych) pozwalałaby na dynamiczne dostosowanie parametrów pracy urządzeń w celu maksymalizacji ich efektywności i żywotności,
    \item \textbf{predykcja zapotrzebowania na konserwację} - na podstawie danych historycznych oraz prognoz stanu technicznego komponentów (np. z wykorzystaniem modeli ML), system wspierałby planowanie działań konserwacyjnych, minimalizując ryzyko awarii,
    \item \textbf{wczesne wykrywanie usterek} - system monitorowałby parametry pracy każdego kluczowego urządzenia, wykrywając nietypowe wzorce i anomalie wskazujące na rozwijające się usterki (np. zużycie łożysk w pompach czy problemy z uszczelnieniami w turbinach).
\end{itemize}

\subsubsection{Przemysł spożywczy}
\label{subsubsec:przemysl_spozywczy}

W zakładzie przetwórstwa spożywczego system mógłby być wykorzystywany do:

\begin{itemize}
    \item \textbf{monitorowania krytycznych urządzeń procesowych} - czujniki temperatury, ciśnienia i wibracji na pompach, mieszadłach, systemach transportujących surowce dostarczałyby danych do analizy, zapewniając stabilność procesów,
    \item \textbf{kontroli parametrów procesów produkcyjnych} - w procesach takich jak: pasteryzacja, fermentacja czy homogenizacja system monitorowałby parametry kluczowe dla jakości i bezpieczeństwa produktów, natychmiast alarmując o odchyleniach,
    \item \textbf{optymalizacji zużycia mediów} - analiza korelacji między zużyciem energii, wody a parametrami pracy urządzeń (np. pomp, sprężarek w systemach chłodzenia) pozwoliłaby na identyfikację nieefektywności.
\end{itemize}

\subsection{Korzyści ekonomiczne i operacyjne}
\label{subsec:korzysci}

Wdrożenie systemu analizy danych strumieniowych, takiego jak opisany w niniejszej pracy, może przynieść szereg wymiernych korzyści ekonomicznych i operacyjnych w różnych sektorach przemysłu.

\subsubsection{Korzyści ekonomiczne}
\label{subsubsec:korzysci_ekonomiczne}

Zaimplementowany system pozwala na szereg korzyści ekonomicznych, takich jak:

\begin{itemize}
    \item \textbf{zmniejszenie liczby nieplanowanych przestojów} - wczesne wykrywanie symptomów potencjalnych awarii urządzeń (np. pomp, sprężarek, turbin) pozwala na zaplanowanie działań naprawczych przed wystąpieniem krytycznej usterki, co bezpośrednio przekłada się na zwiększenie dostępności i produktywności maszyn,
    \item \textbf{optymalizacja procesów produkcyjnych} - analiza danych strumieniowych o parametrach pracy urządzeń i przebiegu procesów może pozwolić na identyfikację nieefektywności i obszarów do optymalizacji, prowadząc do oszczędności surowców i energii,
    \item \textbf{wydłużenie żywotności urządzeń} - dzięki monitorowaniu stanu technicznego i wczesnemu reagowaniu na nieprawidłowości, możliwe jest zapobieganie poważniejszym uszkodzeniom, co może wydłużyć średni czas życia kluczowych aktywów produkcyjnych,
    \item \textbf{poprawa jakości produktów} - stabilniejsze i lepiej kontrolowane procesy produkcyjne, wynikające z ciągłego monitoringu, mogą prowadzić do zmniejszenia liczby wadliwych produktów.
\end{itemize}

\subsubsection{Korzyści operacyjne}
\label{subsubsec:korzysci_operacyjne}

Wdrożenie systemu w przemyśle wymaga również rozwiązywania określonych wyzwań organizacyjnych, takich jak:

\begin{itemize}
    \item \textbf{zwiększenie świadomości sytuacyjnej} - system dostarcza szczegółowych informacji o stanie monitorowanych urządzeń i procesów w czasie rzeczywistym, co umożliwia operatorom i kadrze zarządzającej podejmowanie decyzji w oparciu o dane,
    \item \textbf{wsparcie dla operatorów i personelu utrzymania ruchu} - automatyczne alarmy, diagnostyka i prognozy stanu technicznego odciążają personel, pozwalając skupić się na strategicznych zadaniach,
    \item \textbf{automatyzacja detekcji problemów} - algorytmy analizy danych mogą automatycznie identyfikować anomalie i odchylenia od normy, mogące zostać przeoczone przez człowieka,
    \item \textbf{wsparcie dla ciągłego doskonalenia} - zgromadzone dane i wyniki analiz stanowią cenną bazę wiedzy dla inicjatyw ciągłego doskonalenia procesów produkcyjnych i strategii utrzymania ruchu.
\end{itemize}

\subsection{Analiza ekonomiczna systemów}
\label{subsec:analiza_ekonomiczna}

\subsubsection{Struktura kosztów wdrożenia}
\label{subsubsec:struktura_kosztow}

Typowe koszty wdrożenia systemów takich jak ten opisany w niniejszej pracy mogą obejmować:

\begin{itemize}
    \item \textbf{infrastrukturę sprzętową i programową} - serwery (w przypadku wdrożeń on-premise) lub koszty usług chmurowych (np. dla platformy Kubernetes, baz danych), czujniki, urządzenia brzegowe (edge devices),
    \item \textbf{oprogramowanie} - licencje na komercyjne komponenty platformy,
    \item \textbf{wdrożenie i konfigurację} - instalacja infrastruktury, konfiguracja oprogramowania (np. klastra Kubernetes, usług danych), integracja z istniejącymi systemami (np. SCADA, MES, ERP),
    \item \textbf{rozwój i dostosowanie modeli analitycznych} - tworzenie i trenowanie modeli uczenia maszynowego, implementacja algorytmów detekcji anomalii, dostosowanie do specyficznych typów danych i urządzeń,
    \item \textbf{szkolenia i wsparcie} - przygotowanie personelu do obsługi i wykorzystania systemu, wsparcie techniczne.
\end{itemize}

\subsubsection{Czynniki wpływające na zwrot z inwestycji}
\label{subsubsec:analiza_roi}

Potencjalny zwrot z inwestycji w systemy analizy danych przemysłowych jest zależny od wielu czynników takich jak:

\begin{itemize}
    \item \textbf{skala wdrożenia} - liczba monitorowanych maszyn, punktów pomiarowych i złożoność procesów,
    \item \textbf{krytyczność monitorowanych aktywów} - im wyższe koszty przestojów lub awarii, tym większe potencjalne oszczędności,
    \item \textbf{koszty operacyjne systemu} - utrzymanie infrastruktury, aktualizacja modeli, zarządzanie platformą,
    \item \textbf{stopień integracji z procesami biznesowymi} - zdolność organizacji do wykorzystania informacji z systemu do podejmowania decyzji i modyfikacji działań operacyjnych,
    \item \textbf{istniejący poziom dojrzałości cyfrowej} - firmy z już istniejącą infrastrukturą do zbierania danych mogą osiągnąć ROI (ang. Return on Investment) szybciej.
\end{itemize}

\newpage

\subsection{Wyzwania wdrożeniowe}
\label{subsec:wyzwania_wdrozeniowe}

\subsubsection{Wyzwania techniczne}
\label{subsubsec:wyzwania_techniczne}

Wdrożenie oprogramowania w przemyśle wymaga również rozwiązywania określonych wyzwań technicznych.

\begin{itemize}
    \item \textbf{Integracja z istniejącymi systemami OT/IT} - przedsiębiorstwa często posiadają złożone środowiska z systemami takimi jak: system SCADA, MES (ang. Manufacturing Execution System), ERP (ang. Enterprise Resource Planning). Wykorzystanie platformy opartej na mikrousługach i kolejkach komunikatów (np. broker Kafka) może ułatwić ten proces,
    \item \textbf{Jakość, dostępność i heterogeniczność danych} - fundamentalnymi w tym względzie wyzwaniami są:  zapewnienie odpowiedniej jakości danych z czujników, ich kompletności oraz obsługa różnych formatów i protokołów komunikacyjnych. Konieczne może być wdrożenie procesów czyszczenia i transformacji danych,
    \item \textbf{Synchronizacja i kontekstualizacja danych} - celem zapewnienia poprawnej analizy dane z różnych źródeł powinny być odpowiednio zsynchronizowane czasowo i wzbogacone o kontekst (np. informacje o typie maszyny, trybie pracy),
    \item \textbf{Skalowalność i wydajność platformy} - rosnąca liczba czujników i wolumen danych wymagają skalowalnej infrastruktury. Zastosowanie technologii takich jak: klaster Kubernetes i silnik analityczny Apache Spark pozwala rozwiązać wcześnie wymienione wyzwania, ale wymaga odpowiedniej konfiguracji i zarządzania klastrem Kubernetes,
    \item \textbf{Bezpieczeństwo systemów} - systemy Przemysłowego Internetu Rzeczy (IIoT ang. \textit{Industrial Internet of Things}) i analityki danych są potencjalnym celem ataków. Zabezpieczenie komunikacji, danych oraz dostępu do systemu (np. z wykorzystaniem serwera autoryzacji Keycloak) w takiej sytuacji staje się krytyczne,
    \item \textbf{Rozwój i utrzymanie modeli ML} - modele uczenia maszynowego mogą wymagać regularnego retrenowania i walidacji, celem utrzymania w zmieniających się warunkach operacyjnych.
\end{itemize}

\subsubsection{Wyzwania organizacyjne}
\label{subsubsec:wyzwania_organizacyjne}

Wdrożenie oprogramowania w przemyśle wymaga również rozwiązywania określonych wyzwań organizacyjnych.

\begin{itemize}
    \item \textbf{Silosy organizacyjne} - efektywne wdrożenie i wykorzystanie systemu analitycznego często wymaga współpracy między działami IT, produkcji, utrzymania ruchu i zarządzania jakością.
    \item \textbf{Kompetencje i umiejętności} - organizacje potrzebują pracowników z umiejętnościami w zakresie analizy danych, uczenia maszynowego oraz znajomością specyfiki procesów przemysłowych. Wskazany jest wówczas rozwój tych kompetencji lub pozyskanie specjalistów w tej dziedzinie.
    \item \textbf{Opór przed zmianą} - wprowadzenie nowych technologii i podejść opartych na danych może napotkać opór ze strony pracowników przyzwyczajonych do tradycyjnych metod pracy. W takich okolicznościach istotne jest zarządzanie zmianą i komunikacja korzyści.
    \item \textbf{Definiowanie jasnych przypadków użycia i mierzalnych celów} - sukces wdrożenia zależy od dobrze zdefiniowanych problemów, rozwiązywanych przez oprogramowanie oraz od możliwości zmierzenia jego wpływu na operacje będące w toku w danym obszarze działalności.
    \newpage
    \item \textbf{Zarządzanie zmianą i adaptacja procesów} - wdrożenie systemu analitycznego to nie tylko kwestia technologii, ale również adaptacji istniejących procesów biznesowych i operacyjnych, celem pełnego wykorzystania potencjału płynącego z analizy danych.
\end{itemize}

\subsection{Perspektywy rozwoju zastosowań praktycznych}
\label{subsec:perspektywy_rozwoju}

Dotychczasowe doświadczenia zdobyte podczas wdrażania oprogramowania, będącego przedmiotem niniejszej pracy oraz analiza trendów technologicznych i biznesowych pozwalają na zidentyfikowanie kluczowych kierunków rozwoju zastosowań praktycznych w przyszłości.

\subsubsection{Kierunki rozwoju technologicznego}
\label{subsubsec:kierunki_rozwoju_tech}

Zaprezentowane w pracy rozwiązanie nie jest pozbawione mankamentów. Jego zastosowanie w przemyśle wymaga dalszego udoskonalenia, w tym także dalszych badań i testów, w celu zapewnienia skuteczności i skalowalności w różnych środowiskach produkcyjnych. Dotyczy ono następujących aspektów:

\begin{itemize}
    \item \textbf{metodologia edge computing} - przetwarzanie danych bliżej źródła ich powstawania, co zmniejsza opóźnienia i obciążenie sieci, szczególnie istotne dla aplikacji wymagających bardzo szybkiej reakcji,
    \item \textbf{zaawansowane algorytmy AI/ML} - rozwój i implementacja bardziej zaawansowanych technik, takich jak: głębokie uczenie (np. LSTM (ang. Long Short-Term Memory) do analizy szeregów czasowych), systemy uczące się w trybie online, czy bardziej zaawansowane metody detekcji anomalii,
    \item \textbf{Digital Twin (Cyfrowy Bliźniak)} - integracja systemu analitycznego z cyfrowymi reprezentacjami fizycznych aktywów i procesów, umożliwiająca symulacje, testowanie scenariuszy "what-if" i optymalizację w środowisku wirtualnym,
    \item \textbf{integracja z siecią 5G i rozszerzonym IoT} - wykorzystanie nowych standardów komunikacji do obsługi jeszcze większej liczby rozproszonych czujników i urządzeń mobilnych.
\end{itemize}