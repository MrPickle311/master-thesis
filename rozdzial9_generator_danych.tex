\chapter{Generator danych w chmurze}
\label{chap:generator_danych}

\section{Wprowadzenie}
\label{sec:generator_wprowadzenie}

W celu przeprowadzenia kompleksowych testów systemu analizy danych w czasie rzeczywistym oraz zapewnienia ciągłego strumienia danych testowych, opracowano dedykowany generator danych w chmurze. Generator ten symuluje pracę rzeczywistych urządzeń przemysłowych, generując realistyczne dane sensoryczne z uwzględnieniem korelacji między parametrami oraz ewolucji stanów technicznych urządzeń.

System generatora został zbudowany w oparciu o architekturę bezserwerową (serverless) w chmurze AWS z wykorzystaniem AWS Lambda. Generator produkuje dane dla trzech typów urządzeń przemysłowych: pomp, sprężarek i turbin, symulując cztery rodzaje parametrów sensorycznych: temperaturę, ciśnienie, wibracje oraz wilgotność.

Wartości symulowanych danych są generowane w oparciu o dostarczony wcześniej dataset z danymi z prawdziwych urządzeń przemysłowych. Na tej podstawie generowane są przebiegi czasowe dla każdego z czterech parametrów sensorycznych.


\section{Implementacja generowania danych}
\label{sec:implementacja_generowania}

\subsection{Przygotowanie danych wstępnych do generatora}

Proces przygotowania danych do generatora rozpoczyna się od analizy rzeczywistego datasetu przemysłowego za pomocą skryptu opartego na Apache Spark. Dataset wejściowy zawiera dane sensoryczne z trzech typów urządzeń przemysłowych (pomp, sprężarek i turbin) rozmieszczonych w różnych lokalizacjach fabrycznych.

\subsubsection{Struktura danych wejściowych}

Dane wejściowe składają się z następujących kolumn:
\begin{itemize}
    \item \textbf{temperature} - temperatura urządzenia w stopniach Celsjusza
    \item \textbf{pressure} - ciśnienie w barach
    \item \textbf{vibration} - poziom wibracji w jednostkach przyspieszenia
    \item \textbf{humidity} - wilgotność względna w procentach
    \item \textbf{equipment} - typ urządzenia (Pump, Compressor, Turbine)
    \item \textbf{location} - lokalizacja fabryczna (Atlanta, Chicago, San Francisco, New York, Houston)
    \item \textbf{faulty} - wskaźnik stanu awaryjnego (0.0 = normalny, 1.0 = awaryjny)
\end{itemize}

\subsubsection{Proces analizy statystycznej}

Skrypt wykonuje wieloetapową analizę danych w następujących krokach:

\paragraph{Analiza podstawowych statystyk}
Obliczane są podstawowe metryki statystyczne dla każdego typu urządzenia i stanu technicznego:
\begin{itemize}
    \item Średnie wartości
    \item Odchylenia standardowe
    \item Wartości minimalne i maksymalne
    \item Liczebności obserwacji
\end{itemize}

Wyniki są grupowane według kombinacji typu urządzenia i stanu awaryjnego, co pozwala na wyznaczenie charakterystycznych zakresów parametrów dla stanów normalnych i awaryjnych każdego typu urządzenia.

\paragraph{Analiza korelacji między sensorami}
Skrypt generuje macierze korelacji Pearsona dla różnych kombinacji:
\begin{itemize}
    \item \textbf{Korelacja ogólna} - dla wszystkich danych
    \item \textbf{Korelacja specyficzna dla urządzeń} - oddzielnie dla każdego typu urządzenia
    \item \textbf{Korelacja specyficzna dla stanów} - oddzielnie dla stanów normalnych i awaryjnych
    \item \textbf{Korelacja kombinowana} - dla każdej kombinacji typu urządzenia i stanu
\end{itemize}

Dodatkowo obliczana jest macierz zmian korelacji:
$$\Delta\mathbf{R} = \mathbf{R}_{faulty} - \mathbf{R}_{normal}$$

która pokazuje, jak korelacje między parametrami zmieniają się podczas przejścia ze stanu normalnego do awaryjnego.

\paragraph{Analiza głównych składowych (PCA)}
Wykonywana jest analiza PCA w celu:
\begin{itemize}
    \item Identyfikacji najważniejszych czynników wpływających na stan urządzeń
    \item Określenia wkładu każdego parametru sensorycznego w główne składowe
    \item Redukcji wymiarowości danych przy zachowaniu maksymalnej wariancji
    \item Wyznaczenia ważności cech (\textit{feature importance}) na podstawie ładunków składowych
\end{itemize}

\subsubsection{Generowane pliki konfiguracyjne}

Wyniki analizy są zapisywane w katalogu \texttt{analysis\_output} w postaci plików wykorzystywanych przez generator:

\paragraph{Pliki statystyk}
\begin{itemize}
    \item Statystyki podstawowe dla każdego typu urządzenia
    \item Statystyki dla stanów normalnych i awaryjnych
    \item Statystyki dla kombinacji urządzenie-stan
\end{itemize}

\paragraph{Pliki korelacji}
\begin{itemize}
    \item Macierz korelacji ogólnej
    \item Macierze specyficzne dla urządzeń i stanów
    \item Macierz zmian korelacji
    \item Słownik korelacji w formacie JSON
\end{itemize}

Te pliki konfiguracyjne stanowią fundament dla generatora danych, zapewniając, że symulowane dane zachowują realistyczne charakterystyki statystyczne i korelacyjne obserwowane w rzeczywistych danych przemysłowych.

\subsection{Stany urządzeń}
\label{subsec:symulator_urzadzen}

Drugi komponent to zaawansowany symulator urządzeń przemysłowych, który implementuje model stanów dla każdego urządzenia:

\begin{itemize}
    \item \textbf{Stan normalny} - prawidłowa praca urządzenia
    \item \textbf{Wczesne zużycie} - początkowe oznaki degradacji
    \item \textbf{Stan podkrytyczny} - znaczące pogorszenie parametrów
    \item \textbf{Stan krytyczny} - stan przed awarią
    \item \textbf{Naprawa} - okres nieaktywności urządzenia
\end{itemize}

Symulator wykorzystuje wielowymiarowe rozkłady normalne do generowania skorelowanych danych sensorycznych, zachowując realistyczne relacje między parametrami. Przejścia między stanami są modelowane probabilistycznie z określonymi czasami trwania dla każdego stanu.


\subsection{Model matematyczny korelacji}
\label{subsec:model_korelacji}

Generator wykorzystuje macierze korelacji wyznaczone na podstawie analizy danych historycznych. Dla każdego typu urządzenia $i$ i stanu $s$ definiowana jest macierz korelacji $\mathbf{R}_{i,s}$:

$$\mathbf{R}_{i,s} = \begin{pmatrix}
1 & r_{12} & r_{13} & r_{14} \\
r_{12} & 1 & r_{23} & r_{24} \\
r_{13} & r_{23} & 1 & r_{34} \\
r_{14} & r_{24} & r_{34} & 1
\end{pmatrix}$$

gdzie $r_{jk}$ oznacza współczynnik korelacji między parametrami $j$ i $k$.

\subsection{Generowanie skorelowanych wartości}
\label{subsec:generowanie_skorelowanych}

Skorelowane wartości sensoryczne są generowane przy użyciu wielowymiarowego rozkładu normalnego:

$$\mathbf{x} \sim \mathcal{N}(\boldsymbol{\mu}, \boldsymbol{\Sigma})$$

gdzie:
\begin{itemize}
    \item $\boldsymbol{\mu}$ - wektor średnich wartości dla danego urządzenia i stanu
    \item $\boldsymbol{\Sigma}$ - macierz kowariancji obliczona z macierzy korelacji i odchyleń standardowych
\end{itemize}

Macierz kowariancji obliczana jest jako:
$$\boldsymbol{\Sigma} = \mathbf{D} \mathbf{R} \mathbf{D}$$

gdzie $\mathbf{D}$ to macierz diagonalna odchyleń standardowych.

\subsection{Wygładzanie czasowe}
\label{subsec:wygladzanie_czasowe}

Aby zapewnić realistyczne przejścia między kolejnymi odczytami, implementowane jest wygładzanie wykładnicze:

$$x_{t+1} = \alpha \cdot x_t + (1-\alpha) \cdot x_{target}$$

gdzie:
\begin{itemize}
    \item $x_t$ - poprzednia wartość parametru
    \item $x_{target}$ - docelowa wartość z rozkładu
    \item $\alpha$ - współczynnik wygładzania (0.8)
\end{itemize}

\section{Integracja z chmurą AWS}
\label{sec:integracja_aws}

\subsection{Funkcja Lambda w chmurze AWS}
\label{subsec:lambda_funkcja}

Trzeci komponent to funkcja AWS Lambda, która:

\begin{itemize}
    \item \textbf{Generuje dane w czasie rzeczywistym} - uruchamiana co minutę, generuje nowe odczyty sensoryczne
    \item \textbf{Publikuje dane} - wysyła wygenerowane dane do odpowiednich tematów SNS dla każdego typu sensora
\end{itemize}

\subsection{Amazon SNS}
\label{subsec:amazon_sns}

Generator wykorzystuje Amazon Simple Notification Service (SNS) do publikowania danych. Dla każdego typu sensora utworzony jest oddzielny temat SNS:

\begin{itemize}
    \item \texttt{sensor\_data\_temperature} - dane temperaturowe
    \item \texttt{sensor\_data\_pressure} - dane ciśnieniowe  
    \item \texttt{sensor\_data\_vibration} - dane wibracyjne
    \item \texttt{sensor\_data\_humidity} - dane wilgotnościowe
\end{itemize}

\subsection{Modele danych}
\label{subsec:modele_danych}

Dane są strukturyzowane przy użyciu biblioteki Pydantic, zapewniając walidację i spójność formatów:

\begin{verbatim}
{
  "label": "pump",
  "timestamp": 1672531200,
  "event_key": "pump-1672531200",
  "temperature": {
    "temperature": 72.5
  }
}
\end{verbatim}

\subsection{Konfiguracja i parametryzacja}
\label{sec:konfiguracja}

\subsubsection{Parametry stanów urządzeń}
\label{subsec:parametry_stanow}

Każdy stan urządzenia charakteryzuje się określonymi parametrami czasowymi:

\begin{table}[h]
\centering
\begin{tabular}{|l|c|c|l|}
\hline
\textbf{Stan} & \textbf{Min. czas} & \textbf{Max. czas} & \textbf{Następny stan} \\
\hline
Normalny & 24h & 7 dni & Wczesne zużycie \\
Wczesne zużycie & 12h & 48h & Podkrytyczny \\
Podkrytyczny & 6h & 24h & Krytyczny \\
Krytyczny & 1h & 6h & Naprawa \\
Naprawa & 2h & 4h & Normalny \\
\hline
\end{tabular}
\caption{Parametry czasowe stanów urządzeń}
\label{tab:parametry_stanow}
\end{table}

\subsubsection{Zmienne środowiskowe}
\label{subsec:zmienne_srodowiskowe}

Funkcja Lambda konfigurowana jest poprzez zmienne środowiskowe:

\begin{itemize}
    \item \texttt{TOPIC\_PREFIX} - prefiks tematów SNS
    \item \texttt{AWS\_REGION} - region AWS
    \item \texttt{SEED} - ziarno generatora liczb losowych dla deterministycznych wyników
\end{itemize}

\subsubsection{Zastosowania i korzyści}
\label{sec:zastosowania_korzysci}

Generator danych w chmurze zapewnia:

\begin{itemize}
    \item \textbf{Środowisko testowe} - umożliwia testowanie systemu analizy bez dostępu do rzeczywistych urządzeń przemysłowych
    \item \textbf{Scenariusze awaryjne} - generuje kontrolowane scenariusze degradacji i awarii urządzeń
    \item \textbf{Walidację algorytmów} - pozwala na weryfikację skuteczności algorytmów detekcji anomalii
    \item \textbf{Obciążenia testowe} - umożliwia generowanie dużych wolumenów danych do testów skalowalności
    \item \textbf{Ciągłość działania} - zapewnia nieprzerwany strumień danych do celów demonstracyjnych i szkoleniowych
\end{itemize}

System ten stanowi kluczowy element infrastruktury testowej, umożliwiając kompleksną walidację opracowanego systemu analizy danych w czasie rzeczywistym bez konieczności dostępu do rzeczywistych instalacji przemysłowych. 