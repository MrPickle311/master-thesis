\newpage
\section{Odwzorowanie rzeczywistości w symulacji}
{

    \subsection{Założenia i uproszczenia}
    {
        Algorytm działa, ale z pewnymi założeniami dotyczącymi samego robota oraz jego otoczenia. Pierwszym założeniem jest to, że teren wokół robota jest idealnie płaski oraz że współczynnik tarcia między kołami robota a nawierzchnią jest stały podczas jazdy. Drugim założeniem jest to, że na trajektorii podróży robota nie ma żadnych przeszkód. Trzecim założeniem jest to, że silniki sterujące napędem robota mają możliwość szybkiego hamowania oraz przyspieszania. Dodatkowym założeniem jest to, że robot ma równomiernie rozłożony ciężar w całej swojej objętości, inaczej mówiąc, w całym korpusie robota jego gęstość jest stała.
    }
    \subsection{Wymagania względem układu elektronicznego}
    {
         Robot musi mieć zaprogramowany sterownik napędu, który steruje wymuszeniem prędkościowym. Wymagany jest tutaj napęd różnicowy. Robot musi być wyposażony~w~odbiornik GPS. By algorytm obliczania orientacji mógł zadziałać, to robot musi być wyposażony w magnetometr~oraz~w~IMU. Komputer pokładowy robota musi mieć możliwość zainstalowania na nim systemu operacyjnego z rodziny Debian oraz mieć wystarczającą ilość pamięci operacyjnej, oraz dyskowej, by móc zainstalować na nim wspomniane środowisko ROS.
    }
    \subsection{Wymagania względem otoczenia robota}
    {
        W aktualnej implementacji autopilota robot jest w stanie wykonywać proste misje. Robot nie powinien być umieszczony w budynkach oraz w miejscach znajdujących się pod ziemią. W~aktualnej implementacji robot nie będzie w stanie podróżować między gęstymi zabudowaniami ze względu na słaby sygnał GPS, którego błąd odczytu pozycji może przekroczyć odległość między budynkami. Teren taki również nie powinien być zbytnio zalesiony, gdyż w takim otoczeniu również precyzja wskazań odbiornika GPS zmniejszy się do tego poziomu, że robot nie będzie w stanie wykonać misji ze względu na błędy odczytu swojej pozycji. Sam teren powinien być sprawdzony pod tym kątem, czy napęd robota jest w stanie go pokonać. W ogólności należy unikać miejsc, które ograniczają widoczność satelitów. Teren również powinien być płaski oraz nie zawierać przeszkód, które mogłyby się znaleźć na trajektorii robota. Ta sprawa tyczy się również ruchu ulicznego, patrolowany obszar nie powinien mieć kontaktu z lokalnymi drogami oraz zabudowaniami. 
    }
}