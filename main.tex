\documentclass[12pt]{przejsciowka}
\setlength{\headheight}{15pt}

% \usepackage[authoryear]{natbib}
\bibliographystyle{plainnat}

\class{Praca dyplomowa magisterska}
\title{Analiza danych w czasie rzeczywistym z systemów wieloczujnikowych z wykorzystaniem klastra Kubernetes oraz narzędzi Big Data}
\author{Damian Wójcik}
\instructor{dr. inż. Łukasz Oskwarek}%

% Ensure all chapter files are properly closed
\newcommand{\inputcleanchapter}[1]{%
    \begingroup
    \input{#1}
    \endgroup
}

\newcommand\enablesectionformat
{
    \titleformat{\section}[block]
    {
        \Large
        \bfseries
        \filcenter
    }{\thesection.}{10pt}{}
}

\newcommand\singlebigimage[2]
{
    \begin{figure}[h]
        \centering
        \includegraphics[width=1.0\textwidth]{#1}
        \caption{#2}
        \label{#2}
    \end{figure}
}

\newcommand\singlesizedimage[3]
{
    \begin{figure}[h!]
        \centering
        \includegraphics[width=#3\textwidth]{#1}
        \caption{#2}
        \label{#2}
    \end{figure}
}

\newcommand\singlesizedimageforced[3]
{
    \begin{figure}[H]
        \centering
        \includegraphics[width=#3\textwidth]{#1}
        \caption{#2}
        \label{#2}
    \end{figure}
}

\newcommand*{\captionsource}[2]{%
  \caption[{#1}]{%
  \centering
    #1%
    \\\hspace{\linewidth}%
    \textbf{Źródło: } #2%
  }%
}

\newcommand\singlesizedurlimage[4]
{
    \begin{figure}[h!]
        \centering
        \includegraphics[width=#3\textwidth]{#1}
        \caption{#2 #4}
        % \captionsource{#2}{#4}
        \label{#2}
    \end{figure}
}

\newcommand\sectionref[1]
{
    \textit{\numberstringnum{\getrefnumber{#1}}}
}

% Redefine \ref to display section/subsection numbers within square brackets
\AtBeginDocument{
    \let\oldref\ref
    \renewcommand{\ref}[1]{[\oldref{#1}]}
}



\begin{document}

    \maketitle
    \tableofcontents
    \clearpage

    \inputcleanchapter{abstract_pl}
    \clearpage

    \inputcleanchapter{abstract_en}
    \clearpage


    \inputcleanchapter{rozdzial1_wprowadzenie}
    \clearpage


    \inputcleanchapter{rozdzial2_przeglad_literatury}
    \clearpage


    \inputcleanchapter{rozdzial3_projekt_systemu}
    \clearpage

    \inputcleanchapter{rozdzial10_funkcje_biznesowe}
    \clearpage

    \inputcleanchapter{rozdzial4_implementacja_systemu}
    \clearpage
    
    \inputcleanchapter{rozdzial5_algorytmy_analizy}
    \clearpage
    
    \inputcleanchapter{rozdzial9_generator_danych}
    \clearpage

    \inputcleanchapter{rozdzial11_konfiguracja_kubernetes}
    \clearpage

    \inputcleanchapter{rozdzial12_autoryzacja}
    \clearpage

    \inputcleanchapter{rozdzial13_ewolucja_rozwiazania}
    \clearpage

    \inputcleanchapter{rozdzial7_zastosowania}
    \clearpage
 
    \inputcleanchapter{rozdzial8_podsumowanie}
    \clearpage

    \listoffigures
    \clearpage

    \listoftables
    \clearpage

    \clearpage

    \begingroup
    \section*{Spis listingów}

    \vspace{10pt}
    \noindent\textbf{Rozdział 3}
    \begin{itemize}
        \item[3.4.1] Model raportów w bazie danych Elasticsearch \dotfill \pageref{lst:model_raportow}
        \item[3.4.2] Analizatory i normalizatory w bazie danych Elasticsearch \dotfill \pageref{lst:analizatory_i_normalizatory}
    \end{itemize}

    \vspace{10pt}
    \noindent\textbf{Rozdział 7}
    \begin{itemize}
        \item[7.6.1] Plik metadanych \dotfill \pageref{lst:plik_metadanych}
        \item[7.6.2] Klasy modeli danych \dotfill \pageref{lst:klasy_modeli_danych}
        \item[7.6.3] Algorytm wyszukiwania najbliższego rekordu \dotfill \pageref{lst:algorytm_wyszukiwania_rekordu}
        \item[7.6.4] Format publikowanych wiadomości \dotfill \pageref{lst:format_publikowanych_wiadomosci}
    \end{itemize}
    \endgroup
    \clearpage

    \begin{thebibliography}{99}
    \bibitem[Analytics(2022)]{realtime_analytics}
    Analytics (2022).
    \newblock {Real-time Analytics: Definition, Examples, and Use Cases}.
    \newblock \url{https://www.databricks.com/glossary/real-time-analytics}
    \newblock [Accessed: 01/04/2025].

    \bibitem[Przemysł 4.0(2017)]{przemysl40_iot_ogolnie}
    Przemysł 4.0 (2017).
    \newblock {Czym jest Przemysł 4.0?}.
    \newblock \url{https://przemysl-40.pl/index.php/2017/03/22/czym-jest-przemysl-4-0/}
    \newblock [Accessed: 01/04/2025].

    \bibitem[Avro(2022)]{avro_documentation}
    Avro (2022).
    \newblock {Apache Avro Documentation}.
    \newblock \url{https://avro.apache.org/docs/}
    \newblock [Accessed: 01/04/2025].

    \bibitem[Confluent Schema Registry(2022)]{confluent_schema_registry}
    Confluent Schema Registry (2022).
    \newblock {Confluent Schema Registry Documentation}.
    \newblock \url{https://docs.confluent.io/platform/current/schema-registry/index.html}
    \newblock [Accessed: 01/04/2025].

    \bibitem[Processing(2022)]{data_processing_models}
    Processing (2022).
    \newblock {Data Processing Models: Batch vs. Stream Processing}.
    \newblock \url{https://www.confluent.io/learn/batch-vs-real-time-data-processing/}
    \newblock [Accessed: 01/04/2025].

    \bibitem[Challenges(2022)]{realtime_challenges}
    Challenges (2022).
    \newblock {Challenges in Real-time Data Processing}.
    \newblock \url{https://www.infoq.com/articles/real-time-data-processing-challenges/}
    \newblock [Accessed: 01/04/2025].

    \bibitem[Systems(2022)]{multisensor_systems}
    Systems (2022).
    \newblock {Multi-sensor Systems in Industrial Applications}.
    \newblock \url{https://www.sciencedirect.com/science/article/pii/S2405896318332421}
    \newblock [Accessed: 01/04/2025].

    \bibitem[Sensors(2022)]{industrial_sensors}
    Sensors (2022).
    \newblock {Industrial Sensors and Their Applications}.
    \newblock \url{https://www.omega.com/en-us/resources/industrial-sensors}
    \newblock [Accessed: 01/04/2025].

    \bibitem[Architecture(2022)]{multisensor_architecture}
    Architecture (2022).
    \newblock {Multi-sensor System Architecture and Design}.
    \newblock \url{https://www.mdpi.com/1424-8220/22/1/31}
    \newblock [Accessed: 01/04/2025].

    \bibitem[Microservices(2022)]{microservice_architecture}
    Microservices (2022).
    \newblock {Pattern: Microservice Architecture}.
    \newblock \url{https://microservices.io/patterns/microservices.html}
    \newblock [Accessed: 01/04/2025].

    \bibitem[Benefits(2022)]{microservice_benefits}
    Benefits (2022).
    \newblock {Benefits of Microservices Architecture}.
    \newblock \url{https://www.nginx.com/blog/introduction-to-microservices/}
    \newblock [Accessed: 01/04/2025].

    \bibitem[Challenges(2022)]{microservice_challenges}
    Challenges (2022).
    \newblock {Challenges in Microservices Architecture}.
    \newblock \url{https://www.nginx.com/blog/microservices-at-netflix-architectural-best-practices/}
    \newblock [Accessed: 01/04/2025].

    \bibitem[Kubernetes(2022)]{kubernetes}
    Kubernetes (2022).
    \newblock {Kubernetes Documentation}.
    \newblock \url{https://kubernetes.io/docs/home/}
    \newblock [Accessed: 01/04/2025].

    \bibitem[Concepts(2022)]{kubernetes_concepts}
    Concepts (2022).
    \newblock {Kubernetes Core Concepts}.
    \newblock \url{https://kubernetes.io/docs/concepts/}
    \newblock [Accessed: 01/04/2025].

    \bibitem[Benefits(2022)]{kubernetes_benefits}
    Benefits (2022).
    \newblock {Benefits of Using Kubernetes}.
    \newblock \url{https://kubernetes.io/docs/concepts/overview/what-is-kubernetes/}
    \newblock [Accessed: 01/04/2025].

    \bibitem[Kafka(2022)]{kafka}
    Kafka (2022).
    \newblock {Apache Kafka Documentation}.
    \newblock \url{https://kafka.apache.org/documentation/}
    \newblock [Accessed: 01/04/2025].

    \bibitem[Streams(2022)]{kafka_streams}
    Streams (2022).
    \newblock {Kafka Streams Documentation}.
    \newblock \url{https://kafka.apache.org/documentation/streams/}
    \newblock [Accessed: 01/04/2025].

    \bibitem[Flink(2022)]{flink}
    Flink (2022).
    \newblock {Apache Flink Documentation}.
    \newblock \url{https://flink.apache.org/docs/stable/}
    \newblock [Accessed: 01/04/2025].

    \bibitem[Streaming(2022)]{spark_streaming}
    Streaming (2022).
    \newblock {Spark Streaming Documentation}.
    \newblock \url{https://spark.apache.org/docs/latest/streaming-programming-guide.html}
    \newblock [Accessed: 01/04/2025].

    \bibitem[SCADA(2022)]{scada}
    SCADA (2022).
    \newblock {SCADA Systems: What They Are and How They Work}.
    \newblock \url{https://www.inductiveautomation.com/resources/article/what-is-scada}
    \newblock [Accessed: 01/04/2025].

    \bibitem[IoT(2022)]{cloud_iot}
    IoT (2022).
    \newblock {Cloud IoT Platforms Comparison}.
    \newblock \url{https://www.postscapes.com/internet-of-things-platforms/}
    \newblock [Accessed: 01/04/2025].

    \bibitem[Edge Computing(2022)]{edge_computing}
    Edge Computing in Industrial IoT (2022).
    \newblock \url{https://www.iiconsortium.org/pdf/IIC_Edge_Computing_in_IIoT_2019.pdf}
    \newblock [Accessed: 01/04/2025].

    \bibitem[Open Source(2022)]{open_source_realtime}
    Open Source Real-time Analytics Tools (2022).
    \newblock \url{https://www.influxdata.com/blog/open-source-time-series-databases/}
    \newblock [Accessed: 01/04/2025].

    \bibitem[{Microservices}(2022)]{microservices}
    {Pattern: Microservice Architecture} (2022).
    \newblock \url{https://microservices.io/patterns/microservices.html}
    \newblock [Accessed: 27/12/2022].

    \bibitem[Amazon Web Services(2024a)]{iot_definition_aws}
    Amazon Web Services (2024).
    \newblock {What is IoT? - Internet of Things Explained}.
    \newblock \url{https://aws.amazon.com/what-is/iot/}
    \newblock [Accessed: 01/04/2025].

    \bibitem[Amazon Web Services(2024b)]{aws_lambda_docs}
    Amazon Web Services (2024).
    \newblock {What is AWS Lambda?}.
    \newblock \url{https://docs.aws.amazon.com/lambda/latest/dg/welcome.html}
    \newblock [Accessed: 01/04/2025].

    \bibitem[Luckham(2002)]{cep_definition}
    Luckham, D. C. (2002).
    \newblock {The Power of Events: An Introduction to Complex Event Processing in Distributed Enterprise Systems}.
    \newblock Addison-Wesley.

    \bibitem[OPC Foundation(2024)]{opc_ua_definition}
    OPC Foundation (2024).
    \newblock {OPC Unified Architecture}.
    \newblock \url{https://opcfoundation.org/about/opc-technologies/opc-ua/}
    \newblock [Accessed: 01/04/2025].

    \bibitem[Cloud Native Computing Foundation(2024)]{cncf_website}
    Cloud Native Computing Foundation (2024).
    \newblock {CNCF Cloud Native Interactive Landscape}.
    \newblock \url{https://landscape.cncf.io/}
    \newblock [Accessed: 01/04/2025].

    \bibitem[Apache Spark(2024)]{spark_sql_docs}
    Apache Spark (2024).
    \newblock {Spark SQL, DataFrames and Datasets Guide}.
    \newblock \url{https://spark.apache.org/docs/latest/sql-programming-guide.html}
    \newblock [Accessed: 01/04/2025].

    \bibitem[JSON Schema(2024)]{json_schema_org}
    JSON Schema (2024).
    \newblock {JSON Schema}.
    \newblock \url{https://json-schema.org/}
    \newblock [Accessed: 01/04/2025].

    \bibitem[Google(2024)]{protobuf_docs}
    Google (2024).
    \newblock {Protocol Buffers Documentation}.
    \newblock \url{https://protobuf.dev/overview/}
    \newblock [Accessed: 01/04/2025].

    \bibitem[Lightbend(2024)]{hocon_spec}
    Lightbend (2024).
    \newblock {HOCON: Human-Optimized Config Object Notation}.
    \newblock \url{https://github.com/lightbend/config/blob/main/HOCON.md}
    \newblock [Accessed: 01/04/2025].

    \bibitem[PureConfig(2024)]{pureconfig_docs}
    PureConfig (2024).
    \newblock {PureConfig - A boilerplate-free library for loading configuration files}.
    \newblock \url{https://pureconfig.github.io/}
    \newblock [Accessed: 01/04/2025].

    \bibitem[Apache Spark(2024c)]{spark_mllib_reference}
    Apache Spark (2024).
    \newblock {MLlib: Main Guide - Spark 3.5.1 Documentation}.
    \newblock \url{https://spark.apache.org/docs/latest/ml-guide.html}
    \newblock [Accessed: 01/04/2025].

    \bibitem[AWS SNS(2024)]{sns_docs}
    AWS SNS (2024).
    \newblock {Amazon Simple Notification Service}.
    \newblock \url{https://docs.aws.amazon.com/sns/latest/dg/welcome.html}
    \newblock [Accessed: 01/04/2025].

    \bibitem[NoSQL(2024)]{nosql_definition}
    NoSQL (2024).
    \newblock {NoSQL Database}.
    \newblock \url{https://en.wikipedia.org/wiki/NoSQL}
    \newblock [Accessed: 01/04/2025].

    \bibitem[API Gateway(2024)]{api_gateway_definition}
    API Gateway (2024).
    \newblock {API Gateway}.
    \newblock \url{https://aws.amazon.com/api-gateway/}
    \newblock [Accessed: 01/04/2025].

    \bibitem[AWS Step Functions(2024)]{aws_step_functions_docs}
    AWS Step Functions (2024).
    \newblock {AWS Step Functions}.
    \newblock \url{https://aws.amazon.com/step-functions/}
    \newblock [Accessed: 01/04/2025].

    \bibitem[AWS SQS(2024)]{sqs_docs}
    AWS SQS (2024).
    \newblock {Amazon Simple Queue Service}.
    \newblock \url{https://aws.amazon.com/sqs/}
    \newblock [Accessed: 01/04/2025].

    \bibitem[Amazon Web Services(2024)]{aws_definition}
    Amazon Web Services (2024).
    \newblock {Amazon Web Services}.
    \newblock \url{https://aws.amazon.com/}
    \newblock [Accessed: 01/04/2025].

    \bibitem[Terraform(2024)]{terraform_docs}
    Terraform (2024).
    \newblock {Terraform Documentation}.
    \newblock \url{https://www.terraform.io/docs/index.html}
    \newblock [Accessed: 01/04/2025].


    \bibitem[Watermarking(2024)]{watermarking}
    Watermarking (2024).
    \newblock {Watermarking in Apache Spark}.
    \newblock \url{https://www.databricks.com/blog/feature-deep-dive-watermarking-apache-spark-structured-streaming}
    \newblock [Accessed: 01/04/2025].

    \bibitem[PySpark(2024)]{pyspark_docs}
    PySpark (2024).
    \newblock {PySpark Documentation}.
    \newblock \url{https://spark.apache.org/docs/latest/api/python/}
    \newblock [Accessed: 01/04/2025].

    \bibitem[Spark DataFrame(2024)]{spark_dataframe}
    Spark DataFrame (2024).
    \newblock {Spark DataFrame Documentation}.
    \newblock \url{https://spark.apache.org/docs/latest/sql-programming-guide.html#dataframes-and-datasets}
    \newblock [Accessed: 01/04/2025].

    \bibitem[Spark StringIndexer(2024)]{spark_string_indexer}
    Spark StringIndexer (2024).
    \newblock {Spark StringIndexer Documentation}.
    \newblock \url{https://spark.apache.org/docs/latest/ml-features.html#stringindexer}
    \newblock [Accessed: 01/04/2025].

    \bibitem[Spark VectorAssembler(2024)]{spark_vector_assember}
    Spark VectorAssembler (2024).
    \newblock {Spark VectorAssembler Documentation}.
    \newblock \url{https://spark.apache.org/docs/latest/ml-features.html#vectorassembler}
    \newblock [Accessed: 01/04/2025].

    \bibitem[Spark UDF(2024)]{spark_udf}
    Spark UDF (2024).
    \newblock {Spark UDF Documentation}.
    \newblock \url{https://spark.apache.org/docs/latest/sql-ref-functions-udf-scalar.html}
    \newblock [Accessed: 01/04/2025].

    \bibitem[Apache Spark(2024)]{spark_documentation}
    Apache Spark (2024).
    \newblock {Apache Spark Documentation}.
    \newblock \url{https://spark.apache.org/docs/latest/}
    \newblock [Accessed: 01/04/2025].

    \bibitem[Apache NiFi(2024)]{apache_nifi}
    Apache NiFi (2024).
    \newblock {Apache NiFi Documentation}.
    \newblock \url{https://nifi.apache.org/docs/nifi-docs/}
    \newblock [Accessed: 01/04/2025].

    \bibitem[Apache Druid(2024)]{apache_druid}
    Apache Druid (2024).
    \newblock {Apache Druid Documentation}.
    \newblock \url{https://druid.apache.org/docs/latest/}
    \newblock [Accessed: 01/04/2025].

    \bibitem[InfluxDB(2024)]{influxdb}
    InfluxDB (2024).
    \newblock {InfluxDB Documentation}.
    \newblock \url{https://docs.influxdata.com/influxdb/latest/}
    \newblock [Accessed: 01/04/2025].

    \bibitem[Jones et al.(2015)]{jwt_rfc}
    Jones, M., Bradley, J., \& Sakimura, N. (2015).
    \newblock {JSON Web Token (JWT)}.
    \newblock RFC 7519. Internet Engineering Task Force (IETF).
    \newblock \url{https://www.rfc-editor.org/rfc/rfc7519}
    \newblock [Accessed: 18/06/2024].

    \bibitem[Keycloak Team(2024)]{keycloak_docs}
    Keycloak Team (2024).
    \newblock {Keycloak Documentation}.
    \newblock \url{https://www.keycloak.org/documentation}
    \newblock [Accessed: 18/06/2024].

    \bibitem[Hardt(2012)]{oauth2_rfc}
    Hardt, D. (2012).
    \newblock {The OAuth 2.0 Authorization Framework}.
    \newblock RFC 6749. Internet Engineering Task Force (IETF).
    \newblock \url{https://www.rfc-editor.org/rfc/rfc6749}
    \newblock [Accessed: 18/06/2024].

    \bibitem[Sermersheim(2006)]{ldap_rfc}
    Sermersheim, J. (2006).
    \newblock {Lightweight Directory Access Protocol (LDAP) v3}.
    \newblock RFC 4511. Internet Engineering Task Force (IETF).
    \newblock \url{https://www.rfc-editor.org/rfc/rfc4511}
    \newblock [Accessed: 18/06/2024].

    \bibitem[Amazon Web Services(2024c)]{aws_docs}
    Amazon Web Services (2024).
    \newblock {What is AWS?}.
    \newblock \url{https://aws.amazon.com/what-is-aws/}
    \newblock [Accessed: 18/06/2024].

    \bibitem[Amazon Web Services(2024d)]{eks_docs}
    Amazon Web Services (2024).
    \newblock {What is Amazon EKS?}.
    \newblock \url{https://docs.aws.amazon.com/eks/latest/userguide/what-is-eks.html}
    \newblock [Accessed: 18/06/2024].

    \bibitem[Amazon Web Services(2024e)]{ec2_docs}
    Amazon Web Services (2024).
    \newblock {What is Amazon EC2?}.
    \newblock \url{https://docs.aws.amazon.com/AWSEC2/latest/UserGuide/concepts.html}
    \newblock [Accessed: 18/06/2024].

    \bibitem[Amazon Web Services(2024f)]{cloudformation_docs}
    Amazon Web Services (2024).
    \newblock {What is AWS CloudFormation?}.
    \newblock \url{https://docs.aws.amazon.com/AWSCloudFormation/latest/UserGuide/what-is-cfn.html}
    \newblock [Accessed: 18/06/2024].

    \bibitem[The Kubernetes Authors(2024a)]{kubeadm_docs}
    The Kubernetes Authors (2024).
    \newblock {kubeadm}.
    \newblock \url{https://kubernetes.io/docs/reference/setup-tools/kubeadm/}
    \newblock [Accessed: 18/06/2024].

    \bibitem[Ben-Kiki et al.(2021)]{yaml_spec}
    Ben-Kiki, O., Evans, C., \& döt Net, I. (2021).
    \newblock {YAML Ain't Markup Language (YAML™) Version 1.2}.
    \newblock \url{https://yaml.org/spec/1.2.2/}
    \newblock [Accessed: 18/06/2024].
    
    \bibitem[The Helm Authors(2024)]{helm_docs}
    The Helm Authors (2024).
    \newblock {Helm Documentation}.
    \newblock \url{https://helm.sh/docs/}
    \newblock [Accessed: 18/06/2024].

    \bibitem[Docker Inc.(2024)]{docker_docs}
    Docker Inc. (2024).
    \newblock {Docker Documentation}.
    \newblock \url{https://docs.docker.com/}
    \newblock [Accessed: 18/06/2024].

    \bibitem[containerd(2024)]{containerd_docs}
    containerd (2024).
    \newblock {containerd: An industry-standard container runtime}.
    \newblock \url{https://containerd.io/}
    \newblock [Accessed: 18/06/2024].

    \bibitem[The Kubernetes Authors(2024b)]{nodeport_docs}
    The Kubernetes Authors (2024).
    \newblock {Publish Services (ServiceTypes)}.
    \newblock \url{https://kubernetes.io/docs/concepts/services-networking/service/#publishing-services-servicetypes}
    \newblock [Accessed: 18/06/2024].

    \bibitem[Amazon Web Services(2024g)]{alb_docs}
    Amazon Web Services (2024).
    \newblock {What is an Application Load Balancer?}.
    \newblock \url{https://docs.aws.amazon.com/elasticloadbalancing/latest/application/introduction.html}
    \newblock [Accessed: 18/06/2024].

    \bibitem[The MetalLB Authors(2024)]{metallb_docs}
    The MetalLB Authors (2024).
    \newblock {MetalLB, bare-metal load-balancer for Kubernetes}.
    \newblock \url{https://metallb.universe.tf/}
    \newblock [Accessed: 18/06/2024].

    \bibitem[F5, Inc.(2024)]{nginx_ingress_docs}
    F5, Inc. (2024).
    \newblock {NGINX Ingress Controller}.
    \newblock \url{https://www.nginx.com/products/nginx-ingress-controller/}
    \newblock [Accessed: 18/06/2024].

    \bibitem[The cert-manager Authors(2024)]{cert_manager_docs}
    The cert-manager Authors (2024).
    \newblock {cert-manager}.
    \newblock \url{https://cert-manager.io/docs/}
    \newblock [Accessed: 18/06/2024].

    \bibitem[Okta, Inc.(2024)]{jjwt_docs}
    Okta, Inc. (2024).
    \newblock {JJWT: Java JWT: JSON Web Token for Java and Android}.
    \newblock \url{https://github.com/jwtk/jjwt}
    \newblock [Accessed: 18/06/2024].


    \bibitem[Pearson(1895)]{pearson_correlation}
    Pearson, K. (1895).
    \newblock {Notes on regression and inheritance in the case of two parents}.
    \newblock {Proceedings of the Royal Society of London}.
    \newblock \url{https://www.jstor.org/stable/108555}
    \newblock [Accessed: 18/06/2024].

    \bibitem[Jolliffe(2002)]{jolliffe_pca}
    Jolliffe, I. T. (2002).
    \newblock {Principal component analysis}.
    \newblock Springer.

    \bibitem[Anderson(2003)]{anderson_mva}
    Anderson, T. W. (2003).
    \newblock {An introduction to multivariate statistical analysis}.
    \newblock Wiley.

    \bibitem[Hyndman(2018)]{hyndman_forecasting}
    Hyndman, R. J. (2018).
    \newblock {Forecasting: principles and practice}.
    \newblock OTexts.

    \bibitem[Pydantic(2024)]{pydantic_docs}
    Pydantic (2024).
    \newblock {Pydantic Documentation}.
    \newblock \url{https://docs.pydantic.dev/}
    \newblock [Accessed: 18/06/2024].

    \bibitem[Amazon Web Services(2024)]{sqs_docs}
    Amazon Web Services (2024).
    \newblock {Co to jest Amazon SQS?}.
    \newblock \url{https://docs.aws.amazon.com/AWSSimpleQueueService/latest/SQSDeveloperGuide/welcome-to-sqs.html}
    \newblock [Accessed: 2024-05-20].

    \bibitem[MQTT(2024)]{mqtt_v5}
    MQTT (2024).
    \newblock {MQTT Version 5.0}.
    \newblock \url{https://docs.oasis-open.org/mqtt/mqtt/v5.0/os/mqtt-v5.0-os.html}
    \newblock [Accessed: 2024-05-20].

    \bibitem[AMQP(2024)]{ietf_mqtt_v5}
    AMQP (2024).
    \newblock {OASIS Advanced Message Queuing Protocol (AMQP) Version 1.0}.
    \newblock \url{https://www.researchgate.net/publication/373640610_MQTT_Protocol_for_the_IoT_-_Review_Paper}
    \newblock [Accessed: 2024-05-20].

    \bibitem[MQTT(2021)]{amqp_v1}
    MQTT (2021).
    \newblock {MQTT Protocol for the IoT - Review Paper}.
    \newblock \url{https://docs.oasis-open.org/amqp/core/v1.0/os/amqp-core-overview-v1.0-os.html}
    \newblock [Accessed: 2024-05-20].

    \bibitem[OPC UA(2024)]{opc_ua_spec}
    OPC UA (2024).
    \newblock {OPC Unified Architecture Specification}.
    \newblock \url{https://opcfoundation.org/developer-tools/specifications-unified-architecture/}
    \newblock [Accessed: 2024-05-20].

    \bibitem[Kleppmann(2017)]{kleppmann2017designing}
    Kleppmann, Martin (2017).
    \newblock {Designing data-intensive applications: The big ideas behind reliable, scalable, and maintainable systems}.
    \newblock O'Reilly Media.

    \bibitem[HTTP/1.1(2024)]{rfc_http1_1}
    HTTP/1.1 (2024).
    \newblock {Hypertext Transfer Protocol -- HTTP/1.1}.
    \newblock \url{https://www.rfc-editor.org/rfc/rfc2616}
    \newblock [Accessed: 2024-05-20].

    \bibitem[Chambers(2018)]{chambers2018spark}
    Chambers, Bill and Zaharia, Matei (2018).
    \newblock {Spark: The definitive guide: Big data processing made simple}.
    \newblock O'Reilly Media.

    \bibitem[JSON Schema(2024)]{json_schema_spec}
    JSON Schema (2024).
    \newblock {JSON Schema: A Media Type for Describing JSON Documents}.
    \newblock \url{https://json-schema.org/draft/2020-12/json-schema-core.html}
    \newblock [Accessed: 2024-05-20].

    \bibitem[Protocol Buffers(2024)]{protobuf_docs}
    Protocol Buffers (2024).
    \newblock {Protocol Buffers Developer Guide}.
    \newblock \url{https://developers.google.com/protocol-buffers/docs/overview}
    \newblock [Accessed: 2024-05-20].

    \bibitem[CSV(2024)]{csv_rfc}
    CSV (2024).
    \newblock {Common Format and MIME Type for Comma-Separated Values (CSV) Files}.
    \newblock \url{https://www.rfc-editor.org/rfc/rfc4180}
    \newblock [Accessed: 2024-05-20].

    \bibitem[YAML(2024)]{yaml_spec}
    YAML (2024).
    \newblock {YAML Ain't Markup Language (YAML™) Version 1.2}.
    \newblock \url{https://yaml.org/spec/1.2/spec.html}
    \newblock [Accessed: 2024-05-20].

    \bibitem[AWS(2024)]{aws_docs}
    AWS (2024).
    \newblock {Dokumentacja AWS}.
    \newblock \url{https://docs.aws.amazon.com/}
    \newblock [Accessed: 2024-05-20].

    \bibitem[Amazon EKS(2024)]{eks_docs}
    Amazon EKS (2024).
    \newblock {Co to jest Amazon EKS?}.
    \newblock \url{https://docs.aws.amazon.com/eks/latest/userguide/what-is-eks.html}
    \newblock [Accessed: 2024-05-20].

    \bibitem[Amazon EC2(2024)]{ec2_docs}
    Amazon EC2 (2024).
    \newblock {Co to jest Amazon EC2?}.
    \newblock \url{https://docs.aws.amazon.com/AWSEC2/latest/UserGuide/concepts.html}
    \newblock [Accessed: 2024-05-20].

    \bibitem[AWS CloudFormation(2024)]{cloudformation_docs}
    AWS CloudFormation (2024).
    \newblock {Co to jest AWS CloudFormation?}.
    \newblock \url{https://docs.aws.amazon.com/AWSCloudFormation/latest/UserGuide/Welcome.html}
    \newblock [Accessed: 2024-05-20].

    \bibitem[kubeadm(2024)]{kubeadm_docs}
    kubeadm (2024).
    \newblock {kubeadm}.
    \newblock \url{https://kubernetes.io/docs/reference/setup-tools/kubeadm/}
    \newblock [Accessed: 2024-05-20].

    \bibitem[JWT(2024)]{jwt_rfc}
    JWT (2024).
    \newblock {JSON Web Token (JWT)}.
    \newblock \url{https://www.rfc-editor.org/rfc/rfc7519}
    \newblock [Accessed: 2024-05-20].

    \bibitem[JJWT(2024)]{jjwt_docs}
    JJWT (2024).
    \newblock {JJWT: JSON Web Token for Java and Android}.
    \newblock \url{https://github.com/jwtk/jjwt}
    \newblock [Accessed: 2024-05-20].

    \bibitem[Keycloak(2024)]{keycloak_docs}
    Keycloak (2024).
    \newblock {Keycloak Documentation}.
    \newblock \url{https://www.keycloak.org/documentation}
    \newblock [Accessed: 2024-05-20].

    \bibitem[OAuth 2.0(2024)]{oauth2_rfc}
    OAuth 2.0 (2024).
    \newblock {The OAuth 2.0 Authorization Framework}.
    \newblock \url{https://www.rfc-editor.org/rfc/rfc6749}
    \newblock [Accessed: 2024-05-20].

    \bibitem[LDAP(2024)]{ldap_rfc}
    LDAP (2024).
    \newblock {Lightweight Directory Access Protocol (LDAP): Technical Specification Road Map}.
    \newblock \url{https://www.rfc-editor.org/rfc/rfc4510}
    \newblock [Accessed: 2024-05-20].

    \bibitem[Docker(2024)]{docker_docs}
    Docker (2024).
    \newblock {Docker Documentation}.
    \newblock \url{https://docs.docker.com/}
    \newblock [Accessed: 2024-05-20].

    \bibitem[containerd(2024)]{containerd_docs}
    containerd (2024).
    \newblock {containerd: An industry-standard container runtime}.
    \newblock \url{https://containerd.io/}
    \newblock [Accessed: 2024-05-20].

    \bibitem[NodePort(2024)]{nodeport_docs}
    NodePort (2024).
    \newblock {Publish Services (ServiceTypes) - NodePort}.
    \newblock \url{https://kubernetes.io/docs/concepts/services-networking/service/#nodeport}
    \newblock [Accessed: 2024-05-20].

    \bibitem[Application Load Balancer(2024)]{alb_docs}
    Application Load Balancer (2024).
    \newblock {What is an Application Load Balancer?}.
    \newblock \url{https://docs.aws.amazon.com/elasticloadbalancing/latest/application/introduction.html}
    \newblock [Accessed: 2024-05-20].

    \bibitem[MetalLB(2024)]{metallb_docs}
    MetalLB (2024).
    \newblock {MetalLB, bare-metal load-balancer for Kubernetes}.
    \newblock \url{https://metallb.universe.tf/}
    \newblock [Accessed: 2024-05-20].

    \bibitem[NGINX Ingress Controller(2024)]{nginx_ingress_docs}
    NGINX Ingress Controller (2024).
    \newblock {NGINX Ingress Controller}.
    \newblock \url{https://kubernetes.github.io/ingress-nginx/}
    \newblock [Accessed: 2024-05-20].

    \bibitem[Cert-Manager(2024)]{cert_manager_docs}
    Cert-Manager (2024).
    \newblock {cert-manager}.
    \newblock \url{https://cert-manager.io/docs/}
    \newblock [Accessed: 2024-05-20].

    \bibitem[Tanenbaum(2011)]{tanenbaum2011computer}
    Tanenbaum, Andrew S and Wetherall, David J (2011).
    \newblock {Computer networks}.
    \newblock Pearson Education.

    \bibitem[Luckham(2002)]{luckham2002power}
    Luckham, David C (2002).
    \newblock {The power of events: an introduction to complex event processing in distributed enterprise systems}.
    \newblock Addison-Wesley Professional.




\end{thebibliography}


\end{document}