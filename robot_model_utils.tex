\newpage
\section{Symulator Gazebo}
{
    % \large
    \subsection{Opis symulatora Gazebo}
    {
       Gazebo jest symulatorem służącym do symulowania robotów oraz ich otoczenia \cite{gazebo}. Można~w~nim zaprojektować model robota oraz otaczające go środowisko wraz z prawami fizyki, rządzącymi zasymulowanymi obiektami. Pozwala on na testowanie zaprojektowanych przez inżynierów algorytmów, nie wykorzystując przy tym często kosztownego sprzętu. 

       Symulator Gazebo posiada pełną integrację z ROS-em. Wszystkie składowe elementy symulatora, jego stan oraz procesy zachodzące w nim są publikowane na odpowiednie tematy. Na symulator również można oddziaływać z zewnątrz za pomocą serwisów oraz tematów.
       
       Dzięki niemu można przykładowo uniknąć uszkodzeń sprzętu w przypadku błędu implementacji algorytmu. Przyspiesza on również testowanie rozwiązań, gdyż ponowne uruchomienie sekwencji odbywa się bez resetowania ustawień sprzętu lub bez jego przenoszenia do pozycji początkowej. Zapewnia on również możliwość symulacji różnych czujników z możliwością wprowadzenia do nich dowolnego modelu zakłóceń.
    }
    \subsection{Sposób opisu modelu robota za pomocą URDF}
    {
         Model robota opisuje się w formacie URDF (ang. \textit{Unified Robot Description Format}), by potem ten opis w postaci pliku XML został przetworzony przez parser\footnote{Algorytm najpierw wykonujący analizę składniową danych w celu określenia ich zgodności z określonym językiem, a następnie przetwarzający te dane na obiekty w zaalokowane pamięci} URDF na model robota \cite{urdf}. Potem model ten może być osadzony w symulatorze Gazebo, mając możliwość interakcji z środowiskiem. Możliwość konstrukcji modelu robota jest możliwa dzięki pakietowi \textbf{urdf}.

        \vspace{2mm}
         
         \begin{lstlisting}[language=xml, caption=Przykładowy plik URDF z definicją prostego elementu]
            <robot name="cube">
                <link name="base_link">
                    <visual>
                        <geometry>
                            <box size="1 2 3"/>
                        </geometry>
                    </visual>
                    <collision>
                        <geometry>
                            <box size="1 2 3"/>
                        </geometry>
                    </collision>
                </link>
            </robot>
         \end{lstlisting}
    }
    \newpage
    \subsection{Język makr XACRO}
    {
         Istnieje również język makr zgodny z URDF zwany XACRO \cite{xacro}. Język ten pozwala na definiowanie generycznych elementów oraz ich właściwości pozwalając później na ich konkretyzację~w~celu stworzenia docelowych elementów robota. Ten język makr służy do przyspieszenia definiowania modelu robota.
         
         Na listingu \ref{xacro} podano przykład wzorca prostopadłościanu opisanego za pomocą XACRO. Zostały stworzone trzy konkretne modele tego wzorca.

         \vspace{2mm}
         
         \begin{lstlisting}[language=xml, caption=Przykładowy plik XACRO z definicją szablonu prostego elementu, label=xacro]
             <xacro:macro name="cube" params="x y z">
                <link name="base_link">
                    <visual>
                        <geometry>
                            <box size="${x} ${y} ${z}"/>
                        </geometry>
                    </visual>
                    <collision>
                        <geometry>
                            <box size="${x} ${y} ${z}"/>
                        </geometry>
                    </collision>
                </link>
            </xacro:macro>
            
            <xacro:cube x="1" y="1" z="1"/>
            <xacro:cube x="2" y="4" z="0.4"/>
            <xacro:cube x="1" y="5" z="6"/>
         \end{lstlisting}
    }
    
    \subsection{Elementy dostępne przy budowie modelu robota}
    {
        URDF zawiera swoją specyfikację XML \cite{urdf_xml}, czyli zbiór podstawowych obiektów, z których można zbudować model robota. 
        
        Niezbędne elementy, które pozwolą nam opisać model robota są następujące.
        \begin{itemize}
            \item Element typu \textbf{robot} określa korpus całego robota.
            \item Element typu \textbf{link} określa element robota.
            \item Element typu \textbf{joint} określa więzy kinematyczne.
            \item Element typu \textbf{xacro:macro} określa szablon elementu robota.
            \item Element typu \textbf{gazebo-plugin} dodaje do modelu funkcjonalność dodatku oferowanego przez Gazebo.
        \end{itemize}
    }
}