\section*{Abstract}
{
    The first purpose of this thesis is to conduct analyses and experiments which check the possibilities of autopilot software creation for ground vehicles. The next purpose is an analysis of data processing methods from sensors and getting familiar with data filtration methods.

    Chapters up to and including \ref{transformation_methods} are theoretical chapters. The description begins in Chapter \ref{robot_model} autopilot software implementation.
    
    The scope of the thesis includes issues related to robotics and computer science. The robotic scope includes sensors service, data filtration, data conversion from sensors, coordinate frame selection, and drive theory. The computer science scope includes software architecture, parallel programming, and interprocess communication.

    Made a review of available filters. Conducted experiments and analyses of acquired data during filtration processes by various methods of data filtering. In this way analyzed results for data about the robot's location and orientation. Made also mental analyses on the subject of necessary components ( such as robot drive ), which are required by autopilot software. In the end,~the~difference between the simulated robot and the real one was analyzed.

    The autopilot software was designed~to~allow the execution~of~automated
    missions. For this purpose, ROS was used as a development platform. To create this
    languages such~as~C++~and Python were used.

     Based on the analyzes results and experiments it was concluded that creating software can meet the goal with certain simplifications. Also found that the Kalman filter is the best filter for position and orientation filtration. Additionally, it was discovered that existing software should~be~upgraded.
    
    \textbf{Keywords}: autopilot, GPS, Kalman, filtering, ROS
}