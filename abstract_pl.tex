\section*{Streszczenie pracy}
{
    Pierwszym celem pracy dyplomowej jest przeprowadzenie analiz i eksperymentów sprawdzających możliwości stworzenia oprogramowania autopilota do pojazdów naziemnych. Następnym celem tej pracy były analiza metod procesowania danych z czujników oraz zapoznanie się z rodzajami filtracji danych.

    Rozdziały do \ref{transformation_methods} włącznie są rozdziałami teoretycznymi. Od rozdziału \ref{robot_model} rozpoczyna się opis implementacji oprogramowania autopilota.

    Zakres pracy dyplomowej obejmuje zagadnienia związane z robotyką oraz informatyką. W~zakres robotyki wchodzi obsługa czujników, filtracja danych, konwersja danych z czujników, dobór układu współrzędnych czy też teoria napędów. W zakres informatyki wchodzi architektura oprogramowania, programowanie równoległe oraz komunikacja międzyprocesowa.
    
    Dokonano przeglądu dostępnych filtrów. Przeprowadzono eksperymenty i analizy danych otrzymanych podczas procesów filtrowania różnymi metodami filtrowania danych. Przeanalizowano~w~ten sposób wyniki dla danych o położeniu oraz orientacji robota. Dokonano również analizy myślowej na temat doboru niezbędnych podzespołów (takich jak na przykład napęd robota), które wymagane są przez oprogramowanie autopilota. Na końcu poddano analizie różnice pomiędzy robotem obecnym~w~symulacji a rzeczywistą maszyną.

    Zaprojektowano oprogramowanie autopilota pozwalającego~na~wykonywanie zautomatyzowanych misji. Wykorzystano~w~tym celu ROS jako platformę programistyczną. Do stworzenia tego rozwiązania wykorzystano takie języki~jak~C++~oraz~Python.
    
    Na podstawie wyników analiz i eksperymentów wywnioskowano, że stworzone oprogramowanie jest w stanie sprostać postawionemu celowi z pewnymi uproszczeniami.
    Stwierdzono również, że filtr Kalmana jest najlepszym kandydatem na filtrowanie pozycji oraz orientacji robota. Stwierdzono dodatkowo, że istniejące oprogramowanie można i należałoby~je~rozwinąć.
    
    
    \textbf{Słowa kluczowe}: autopilot, GPS, Kalman, filtrowanie, ROS
}