\begin{abstract_pl}
\begin{center}
\textbf{\large Streszczenie}
\end{center}

\begin{center}
\noindent{Temat:} \textbf{Analiza danych w czasie rzeczywistym z systemów wieloczujnikowych \\ z wykorzystaniem klastra Kubernetes oraz narzędzi Big Data}\\
\end{center}

Niniejsza praca magisterska przedstawia projekt (rozdział \ref{sec:projekt_systemu}), implementację (rozdziały \ref{sec:implementacja_systemu}, \ref{sec:implementacja_generowania}, \ref{chap:konfiguracja_kubernetes}, \ref{chap:autoryzacja}) oraz analizę wydajności i działania (rozdziały \ref{sec:algorytmy_analizy}) systemu do przetwarzania danych w czasie rzeczywistym z wielu czujników, wykorzystując architekturę mikroserwisową opartą na klastrze Kubernetes oraz narzędziach Big Data, w tym Apache Kafka i Apache Spark.

Praca koncentruje się na efektywnym pozyskiwaniu, przetwarzaniu i analizie strumieni danych generowanych przez systemy wieloczujnikowe w środowiskach przemysłowych. Zaproponowany system umożliwia monitorowanie stanu instalacji przemysłowych w czasie rzeczywistym, wykrywanie anomalii oraz przewidywanie potencjalnych awarii.

Przeprowadzone badania i analizy wykazują, że zastosowanie klastra Kubernetes jako platformy do wdrożenia systemu analitycznego zapewnia elastyczne skalowanie zasobów w zależności od obciążenia. Natomiast wykorzystanie narzędzi do przetwarzania strumieniowego, takich jak Apache Spark, umożliwia efektywną analizę dużych wolumenów danych w czasie rzeczywistym.

Praca pokazuje również potencjalne praktyczne zastosowania (rozdział \ref{sec:zastosowania_praktyczne}) opracowanego rozwiązania w przemyśle, przedstawiając korzyści ekonomiczne i operacyjne wynikające z wdrożenia systemu analitycznego opartego na klastrze Kubernetes oraz narzędziach Big Data.

\begin{keywords}
Kubernetes, analiza danych w czasie rzeczywistym, Apache Kafka, Apache Spark, systemy wieloczujnikowe, architektura mikroserwisowa, przetwarzanie strumieniowe, Big Data
\end{keywords}
\end{abstract_pl}