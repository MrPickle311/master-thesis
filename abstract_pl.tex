\begin{abstract_pl}
\begin{center}
\textbf{\large Streszczenie}
\end{center}
\vspace{0.5em}

\noindent\textbf{Temat:} Analiza danych w czasie rzeczywistym z systemów wieloczujnikowych z wykorzystaniem klastra Kubernetes oraz narzędzi Big Data\\

Niniejsza praca magisterska przedstawia projekt, implementację i analizę wydajności systemu do przetwarzania danych w czasie rzeczywistym z wielu czujników, wykorzystując architekturę mikroserwisową opartą na klastrze Kubernetes oraz narzędzia Big Data, w szczególności Apache Kafka i Kafka Streams.

Praca koncentruje się na efektywnym pozyskiwaniu, przetwarzaniu i analizie strumieni danych generowanych przez systemy wieloczujnikowe w środowiskach przemysłowych, ze szczególnym uwzględnieniem procesów produkcyjnych, takich jak synteza amoniaku (proces Habera). Zaproponowany system umożliwia monitorowanie stanu instalacji przemysłowych w czasie rzeczywistym, wykrywanie anomalii oraz przewidywanie potencjalnych awarii.

Przeprowadzone badania eksperymentalne wykazują, że zastosowanie klastra Kubernetes jako platformy do wdrożenia systemu analitycznego zapewnia elastyczne skalowanie zasobów w zależności od obciążenia, a wykorzystanie narzędzi do przetwarzania strumieniowego, takich jak Apache Kafka, umożliwia efektywną analizę dużych wolumenów danych w czasie rzeczywistym.

Praca demonstruje praktyczne zastosowania opracowanego rozwiązania w przemyśle, przedstawiając korzyści ekonomiczne i operacyjne wynikające z wdrożenia systemu analitycznego opartego na klastrze Kubernetes oraz narzędziach Big Data.

\begin{keywords}
Kubernetes, analiza danych w czasie rzeczywistym, Apache Kafka, systemy wieloczujnikowe, architektura mikroserwisowa, przetwarzanie strumieniowe, Big Data
\end{keywords}
\end{abstract_pl}