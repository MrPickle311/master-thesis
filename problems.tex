\newpage
\label{problems}
\section{Eksperymenty}
{
    \subsection{Wstęp}
    {
        W tym rozdziale przedstawiono wykonane implementacje algorytmów filtrujących oraz ich analizę  wyników. Mają one za zadanie rozwiązać problemy opisane w rozdziałach \ref{comapss_data} oraz \ref{gps_analysis}.
    }

    
    \subsection{Niedokładność pomiarów bieżącej orientacji robota}\label{compass_analysis}
    {
        \subsubsection{Przebieg danych odczytanych z kompasu}\label{comapss_data}
        {
            W tym rozdziale została wykonana analiza surowych danych z kompasu. Poniżej przedstawiono przykładowy wykres zależności absolutnej orientacji robota od czasu. Czas jest mierzony~od~początku uruchomienia symulatora Gazebo. Wykres obecny na rysunku \ref{Przykładowy wykres odczytów orientacji robota podczas jego spoczynku} jest pewnym odcinkiem czasowym wyjętym z symulacji.
            
            \singlesizedimage{images/raw_stat.png}{Przykładowy wykres odczytów orientacji robota podczas jego spoczynku}{0.9}

            Dane na wykresie charakteryzują się obecnością stabilnego szumu wzbogaconego o piki, które pojawiają się w losowych chwilach. 
            Otrzymane dane charakteryzują się odchyleniem standardowym około 0,2 radiana. To oznacza, że precyzja obrotu wynosi około 23$^o$. Takie zachowanie stwarza niepożądaną sytuację, która polega na tym, że algorytmy wchodzące w skład oprogramowania autopilota nie są w stanie spełniać swoich zadań, w sposób umożliwiający bezproblemowe działanie autopilota. 
            W rzeczywistości szum może być spowodowany przez czynniki opisane w rozdziałach \ref{mag_noises} oraz \ref{imu_noises}.

            \newpage
            
            Przyjrzyjmy się teraz danym zebranym podczas powolnego obrotu robota.
            
            \singlesizedimage{images/raw_rot.png}{Przykładowy wykres odczytów orientacji robota podczas jego obrotu}{0.9}
            
            Odczytując surowe dane w kompasu podczas obrotu nie można być do końca pewnym~co~do~orientacji robota w punkcie czasowym $t_i$. Patrząc na powyższy wykres można zauważyć, że mimo obrotu ze stałą prędkością kątową, otrzymane dane miejscami wskazują na to, że robot wręcz obrócił~się~na~chwilę~w~przeciwną stronę. Algorytm precyzyjnego obrotu może zakończyć swoje działanie w nieoczekiwanej chwili. Robot może nawet przeskoczyć poszukiwaną wartość kąta~i~robot może wykonać obrót o dodatkowe 360$^o$.
            
            Dane z kompasu należy odfiltrować z pomocą dostępnych filtrów. Inaczej zachowanie zaimplementowanych algorytmów opisanych w rozdziale \ref{algorithms} będzie nieprzewidywalne.
            Przetestowano~w~tym celu 3 rodzaje filtrów wygładzających: filtr średniej ruchomej, filtr Sawickiego-Golaya oraz filtr Kalmana. Dokonano analizy wyników z wyjść z poszczególnych filtrów oraz wybrano filtr, którego działanie wprowadza jak najwięcej efektów pozytywnych na działanie autopilota przy czym nie wprowadzą efektów, które spowodują niestabilne zachowanie algorytmów autopilota.
        }
        \newpage
        \subsubsection{Filtracja odczytów orientacji robota filtrem Sawickiego-Golaya}
        {
        Wykorzystując bibliotekę \textbf{scipy.signal} użyto gotowej implementacji filtru Sawickiego-Golaya. Zaczerpnięto surowe dane z kompasu i przefiltrowano je wspomnianym filtrem. Zastosowano tutaj dopasowanie wielomianowe za pomocą wielomianu 3 stopnia.
        Pomiary zostały wykonane dla filtrów o następujących długościach:
        
        \begin{center}
            \begin{itemize}
                \item $L=7$ próbek,
                \item $L=15$ próbek,
                \item $L=35$ próbek,
                \item $L=51$ próbek.
            \end{itemize}
        \end{center}

        By wykresy były czytelne oraz analiza była wygodniejsza, to pomiary podzielono na dwie grupy względem liczby próbek. Na jednym wykresie umieszczono pomiary na $L=7$ i $L=15$ oraz $L=35$ i $L=51$.
        
        \singlesizedimage{images/sg_filter_L7_L15.png}{Wykres odczytów orientacji robota podczas jego spoczynku dla filtru Sawickiego-Golaya, L=7 i L=15}{0.8}

        Odnosząc się do wykresu obecnego na rysunku \ref{Wykres odczytów orientacji robota podczas jego spoczynku dla filtru Sawickiego-Golaya, L=7 i L=15} można zauważyć, że filtr sprawdził się niewystarczająco, odfiltrowane dane były prawie takie same jak oryginalne. Trzeba było zastosować powiększenie wykresu,~by~zobaczyć jakąkolwiek różnicę.
        Dane otrzymane z wyjścia filtru prawie się pokrywają. Częstotliwość zmian wartości danych z filtra jest niemal taka sama jak danych surowych. Taki wynik z wyjścia filtra spowodowany jest wielomianowym charakterem filtru (jego sposobem dopasowania do danych surowych). Uzyskany wynik jest niezadowalający, jest to spowodowane zbyt wąskim oknem pomiarowym. Użycie tej kombinacji (filtru Sawickiego-Golaya wraz z określonymi długościami okien pomiarowych) uniemożliwia działanie jakiegokolwiek algorytmu ze względu na zbyt duże skoki wartości.

        \newpage
        \singlesizedimage{images/skret_sg_filter_L7_L15.png}{Wykres odczytów orientacji robota podczas jego obrotu dla filtru Sawickiego-Golaya, L=7 i L=15}{0.7}
        
         W przypadku wykresu z rysunku \ref{Wykres odczytów orientacji robota podczas jego obrotu dla filtru Sawickiego-Golaya,
        L=7 i L=15} można zauważyć, że podczas obrotu odfiltrowywane dane były niemal takie same jak oryginalne. Podczas obrotu dane z filtru charakteryzują się brakiem opóźnienia, co akurat jest pożądaną cechą poszukiwanego filtru. Wyjście z filtra dla $L=15$ pozwala jedynie pozbyć się ostrych pików z danych surowych.

         Ostateczny wniosek co do użycia filtru Sawickiego-Golaya przy długościach okien pomiarowych $L=7$ oraz $L=15$ jest taki, że ten filtr w tej kombinacji nie jest w stanie sprostać wymaganiom dotyczących działania autopilota. Zbyt mocno dopasowuje się do oryginalnych danych nie dając satysfakcjonujących wyników.

        \singlesizedimageforced{images/sg_filter_L35_L51.png}{Wykres odczytów orientacji robota podczas jego spoczynku dla filtru Sawickiego-Golaya, L=35 i L=51}{0.8}
        
        \newpage

        \singlesizedimage{images/skret_sg_filter_L35_L51.png}{Wykres odczytów orientacji robota podczas jego obrotu dla filtru Sawickiego-Golaya, L=35 i L=51}{0.8}
        
        Nawiązując do wykresów \ref{Wykres odczytów orientacji robota podczas jego spoczynku dla filtru Sawickiego-Golaya, L=35 i L=51} oraz \ref{Wykres odczytów orientacji robota podczas jego obrotu dla filtru Sawickiego-Golaya, L=35 i L=51} można zauważyć silniejsze wygładzenie na wyjściu z filtra względem poprzednich konfiguracji. Częstotliwość zmian również uległa redukcji względem filtrów o krótszych oknach pomiarowych. Tutaj odfiltrowane dane wydają się łagodniejsze niż na wykresach \ref{Wykres odczytów orientacji robota podczas jego obrotu dla filtru Sawickiego-Golaya,
        L=7 i L=15} lub \ref{Wykres odczytów orientacji robota podczas jego spoczynku dla filtru Sawickiego-Golaya, L=7 i L=15}. Nadal można zauważyć brak opóźnienia mimo tego, że okno pomiarowe zostało zwiększone. Można z tego wywnioskować, że modyfikacja długości okna pomiarowego nie wpływa na opóźnienie na wyjściu z tego filtra. Piki zostały  wygładzone dla filtrów o długościach $L=35$ oraz $L=51$.
       
        Dla wszystkich rozpatrywanych okien pomiarowych, ten filtr cechuje się niewystarczającą skutecznością podczas odfiltrowywania absolutnej orientacji robota. Zwiększenie okna próbek pomiarowych nie zwiększyło znacząco jakości filtracji szumów.
        }
        
        \subsubsection{Filtracja odczytów orientacji robota filtrem średniej ruchomej}\label{av_filter}
        {
            Filtr średniej ruchomej zaimplementowano według wzorów opisanych w rozdziale \ref{avg_filter}. Implementację wykonano w języku Python. Następnie wykonano pomiary orientacji w spoczynku oraz podczas obrotu jednocześnie odczytując przefiltrowane wartości orientacji. 
            
            Pomiary zostały wykonane dla filtrów o następujących długościach:
            
            \begin{center}
                \begin{itemize}
                    \item $L=7$ próbek,
                    \item $L=15$ próbek,
                    \item $L=35$ próbek,
                    \item $L=51$ próbek.
                \end{itemize}
            \end{center}
            
            Aby wykresy były czytelne oraz analiza była wygodniejsza, to pomiary podzielono na dwie grupy względem liczby próbek. Na jednym wykresie umieszczono pomiary na $L=7$ i $L=15$ oraz $L=35$ i $L=51$.
            
            \singlesizedimage{images/av_filter_L7_L15.png}{Wykres odczytów orientacji robota podczas jego spoczynku dla filtru średniej ruchomej, L=7 i L=15}{0.75}
            
            \singlesizedimageforced{images/skret_av_filter_L7_L15.png}{Wykres odczytów orientacji robota podczas jego obrotu dla filtru średniej ruchomej, L=7 i L=15}{0.75}

            \newpage
            Nawiązując do wykresów \ref{Wykres odczytów orientacji robota podczas jego spoczynku dla filtru średniej ruchomej, L=7 i L=15} oraz \ref{Wykres odczytów orientacji robota podczas jego obrotu dla filtru średniej ruchomej, L=7 i L=15} można zauważyć, że dane otrzymane na wyjściu z filtra są wygładzone w większym stopniu oraz częstotliwość zmian jest niższa niż przy zastosowaniu filtra Sawickiego-Golaya dla okna pomiarowego o tej samej długości. 
            Podczas obrotu jakość filtracji jest podobna do tej, gdy robot jest nieruchomy. 
            
            \singlesizedimage{images/av_filter_L35_L51.png}{Wykres odczytów orientacji robota podczas jego spoczynku dla filtru średniej ruchomej, L=35 i L=51}{0.7}
            
            \singlesizedimageforced{images/skret_av_filter_L35_L51.png}{Wykres odczytów orientacji robota podczas jego obrotu dla filtru średniej ruchomej, L=35 i L=51}{0.7}

            \newpage

            Nawiązując do wykresów \ref{Wykres odczytów orientacji robota podczas jego spoczynku dla filtru średniej ruchomej, L=35 i L=51} oraz \ref{Wykres odczytów orientacji robota podczas jego obrotu dla filtru średniej ruchomej, L=35 i L=51} tutaj dane zostały wygładzone mocniej aniżeli w poprzedniej próbie oraz ich zachowanie jest stabilne. Dane wyjściu z tak skonfigurowanego filtra charakteryzują się znikomą tendencją do zmian lokalnych. Stosowanie szerszych okien pomiarowych nie przyniesie już znaczącego wzrostu wygładzenia. Zauważono, że wraz ze wzrostem szerokości okna pomiarowego, opóźnienie zwiększa się.
            
            Jednak patrząc na otrzymane wyniki z tego filtra (np. rys. \ref{Wykres odczytów orientacji robota podczas jego obrotu dla filtru średniej ruchomej,
            L=35 i L=51}), można zauważyć ,że  są opóźnione względem danych surowych. Jest to poważny problem, gdyż jest w stanie spowodować wprowadzeniem błędu rzędu kilku stopni. Jest to spowodowane tym, że każda próbka ma tę samą wagę, a nie powinno tak być, gdyż podczas obrotu możliwe są poślizgi kół robota, czyli przez ułamek sekundy prędkość kątowa będzie inna.

            Ten filtr jest lepszym rozwiązaniem niż filtr Sawickiego-Golaya, jednak nie jest idealny. Przyrost redukcji szumów, przy jednoczesnym zwiększaniu rozmiaru okna pomiarowego jest większy niż~w~przypadku filtru Sawickiego-Golaya.  Dużą wadą filtru średniej ruchomej jest to, że uśrednia on dane bezkrytycznie, nie stosuje on żadnych wag do poszczególnych próbek danych. Ta cecha ujawnia się gdy robot jest skierowany skierowany na północ, gdzie jego orientacja odczytana~z~kompasu przeskakuje pomiędzy 0, a $2\pi$. Zachowanie filtru w tej sytuacji nie jest deterministyczne.

            \singlesizedimage{images/av_filter_0_pos_bad.png}{Wykres pokazujący niedeterministyczne zachowanie filtru średniej ruchomej}{0.8}

            Nawiązując do wykresu \ref{Wykres pokazujący niedeterministyczne zachowanie filtru średniej ruchomej} można zauważyć, ta właściwość sprawia, że w takiej sytuacji orientacja robota jest losowa. By sobie z tym poradzić, to na wejściu filtra orientacja z kompasu zostaje zamieniona na liczbę zespoloną o module 1, a na wyjściu z powrotem należy ją przekonwertować na orientację wyrażoną w radianach.

            \singlesizedimage{images/av_filter_0_pos_good.png}{Wykres pokazujący deterministyczne zachowanie filtru średniej ruchomej po korekcie wejścia}{0.8}

            \newpage
            Nawiązując do wykresu \ref{Wykres pokazujący deterministyczne zachowanie filtru średniej ruchomej po korekcie wejścia} zostało pokazane, że problem przeskakującej orientacji robota został rozwiązany. Jednak pozostał jeszcze jeden problem. Gdy robot obraca się szybko, wyjście filtru nie nadąża za sygnałem oryginalnym jak pokazano na rysunku \ref{Wykres pokazujący problemy z nadążaniem dla filtru średniej ruchomej po korekcie wejścia}.

            \singlesizedimage{images/av_filter_delay.png}{Wykres pokazujący problemy z nadążaniem dla filtru średniej ruchomej po korekcie wejścia}{0.8}

            \newpage
            By rozwiązać problem pokazany na rysunku \ref{Wykres pokazujący problemy z nadążaniem dla filtru średniej ruchomej po korekcie wejścia} istnieje potrzeba, by filtr średniej ruchomej uczynić dynamicznym. Chodzi o to, że szerokość okna pomiarowego powinna zmieniać się wraz ze zmianą prędkości kątowej robota. Robot ma maksymalną dostępną prędkość kątową ustawioną na $\omega_{max}=1 \frac{rad}{s}$. Należy zauważyć, że dokładność jest potrzebna przy niskich prędkościach, tam, gdzie położenie robota się stabilizuje. Należy więc wyznaczyć wyrażenie pełniące rolę mnożnika maksymalnej długości filtra, by otrzymać aktualną długość filtra. Na podstawie przeprowadzonych prób, podczas których analizowano wymagany stopień zmian długości filtra w czasie, wyznaczono poniższy wzór:

            \begin{equation}
                 L= L_{max} (\frac{\omega_{max} - \omega_{current}}{\omega_{max}})^2
            \end{equation}

            gdzie:\\
            $L_{max}$ - maksymalna długość filtra\\
            $\omega_{max}$ - maksymalna prędkość kątowa\\
            $\omega_{current}$ - aktualna prędkość kątowa robota\\
        }
    }
    
    \subsubsection{Filtracja odczytów orientacji robota filtrem Kalmana}
    {
        Tutaj zastosowano możliwość skorzystania z pomiarów z kilku czujników, które mogą zapewnić pomiary tej samej wielkości. Do odczytu aktualnej orientacji robota wykorzystamy 2 czujniki.
        
        \begin{itemize}
            \item Wskazania wektora pola magnetycznego otrzymane z magnetometru.
            \item Wskazania prędkości kątowej robota wokół osi \textbf{z} otrzymane z żyroskopu zawartego w IMU.
        \end{itemize}

        Istnieje możliwość skorzystania z kilku czujników w celu estymacji orientacji w danym punkcie czasu.  W tym celu zaimplementowano rozwiązanie wykorzystujące jednowymiarowy filtr Kalmana oraz fuzję wskazań czujników (ang. \textit{sensor fusion}), po to aby z odczytów z dwóch czujników otrzymać jeden, lecz lepszej jakości. Fuzję czujników opisano w rozdziale \ref{fusion}. Jednowymiarowy filtr Kalmana opisano w rozdziale \ref{kalman_one_dim}.
        
        Orientacja robota jest bezwzględna (względem układu zerowego lokalnego przedstawionego~w~rozdziale \ref{coordinates}) i jest wyznaczana w radianach. Zakres wartości dla orientacji robota wynosi $0;2\pi$.
        Model jest liniowy i zmiana jego stanu jest również liniowa.
        Dane, jakie podlegają pomiarom to prędkość kątowa uzyskana z żyroskopu oraz wskazania magnetometru, zatem etap predykcji następnego stanu składa się ze wzorów jak poniżej. Te wzory są konkretyzacją wzorów opisanych w rozdziale \ref{kalman_one_dim} i zostały użyte to implementacji filtru.
        
        \begin{equation}
            \phi_{n+1,n}=\phi_{n,n}+\Delta t \cdot \phi_{gyro}
        \end{equation}
        \begin{equation}
            p_{n+1,n}=p_{n,n}+q_n
        \end{equation}
        gdzie:\\
        $\phi_{n,n}$ - aktualna estymacja orientacji,\\
        $\phi_{n+1,n}$ - prognoza estymacji orientacji,\\
        $\phi_{gyro}$ - odczyt prędkości kątowej z żyroskopu,\\
        $p_{n+1,n}$ - prognoza niepewności estymacji orientacji,\\
        $p_{n,n}$ - aktualna niepewność estymacji  orientacji,\\
        $q_n$ - szum procesowy.\\

        \newpage
        Następnie zostały przedstawione równania aktualizacji stanu
        \begin{equation}
            \hat{\phi}_{n,n}=\hat{\phi}_{n,n-1}+K_n(\phi_N - \hat{\phi}_{n,n-1})
        \end{equation}
        \begin{equation}
            K_n=\frac{p_{n,n-1}}{p_{n,n-1} + r_n}
        \end{equation}
        \begin{equation}
            p_{n,n}=(1-K_n)p_{n,n-1}
        \end{equation}
        gdzie:\\
        $\phi_{n,n-1}$ - poprzednia estymacja orientacji,\\
        $\phi_N$ - przychodzący pomiar orientacji z magnetometru,\\
        $r_n$ - niepewność pomiaru.

        Przyjęto wartość szumu procesowego $q_n$ jako 0.01 rad. Niepewność pomiaru $r_n$ przyjęto jako 1 rad. Po uruchomieniu robota, ten algorytm jest uruchomiony przez cały czas, a jego wywoływanie się wyzwalane jest przychodzącymi pomiarami z żyroskopu i magnetometru. Ten model nie jest idealny, gdyż nie uwzględnia przyspieszenia kątowego, ale jest to wystarczające.

        \singlesizedimage{images/rot_stat_kalman.png}{Wykres odczytów orientacji robota podczas spoczynku dla filtru Kalmana}{0.8}

        \newpage
        \singlesizedimage{images/rot_rot_kalman.png}{Wykres odczytów orientacji robota podczas obrotu dla filtru Kalmana}{0.8}

        Nawiązując do wykresów \ref{Wykres odczytów orientacji robota podczas spoczynku dla filtru Kalmana} oraz \ref{Wykres odczytów orientacji robota podczas obrotu dla filtru Kalmana} można zauważyć, że otrzymano stopień wygładzenia porównywalny do wyników otrzymanych z filtru średniej ruchomej dla okna pomiarowego o długości $L=35$. Wyniki te charakteryzuje niska częstotliwość zmian, ale trendy globalnie są zgodne z danymi surowymi. Filtr Kalmana sprawdził się w sposób wystarczający oraz nie wprowadza żadnych opóźnień względem danych surowych.

    }
    
    \subsection{Niedokładność wskazań odbiornika GPS}\label{gps_analysis}
    {
        Aby poradzić sobie z problemem niedokładności wskazań GPS, zaimplementowano wielowymiarowy filtr Kalmana na podstawie wzorów przedstawionych w rozdziale \ref{multi_kalman}. Mierzone są tutaj tylko dwie współrzędne \textbf{x} oraz \textbf{y} (po transformacji). Zatem wektor stanu będzie wyglądać następująco:

        \begin{center}
            $\bm{X} = $
            \begin{bmatrix}
                x  \\
                y 
            \end{bmatrix}
        \end{center}  

        Stan początkowy będzie przyjęty jak poniżej. Poniższe macierze są zerowe. Przyjęto je w taki sposób dlatego, że nie znano docelowych wartości ich elementów przed rozpoczęciem pomiaru.  

        \begin{center}
            $\bm{X} = $
            \begin{bmatrix}
                0  \\
                0 
            \end{bmatrix}
        \end{center}  

        \begin{center}
            $\bm{P} = $
            \begin{bmatrix}
            0 & 0 \\
            0 & 0
            \end{bmatrix}
        \end{center} 

        Macierz przejścia nie modyfikuje danych wejściowych, zatem jest macierzą jednostkową o wymiarze 2x2 ze względu na obecność dwóch filtrowanych zmiennych.

        \begin{center}
            $\bm{F} = $
            \begin{bmatrix}
            1 & 0 \\
            0 & 1
            \end{bmatrix}
        \end{center}  

        \newpage

        Zdefiniowano macierze, które są stałe i z góry znane podczas obliczeń.
        Poniżej zdefiniowano macierz obserwowalności. Obserwacji podlega pomiar wartości współrzędnych kartezjańskich x oraz y z konwertera GPS (po transformacji).

        \begin{center}
            $\bm{H} = $
            \begin{bmatrix}
            1 & 0 \\
            0 & 1
            \end{bmatrix}
        \end{center} 

        Macierz wejścia oraz wektor wejścia są zerowe. Jest to spowodowane brakiem wejścia do układu, które wpływa w sposób deterministyczny na wskazania GPS.

        \begin{center}
            $\bm{G} = $
            \begin{bmatrix}
            0 & 0 \\
            0 & 0
            \end{bmatrix}
        \end{center}

        \begin{center}
            $\bm{u}_n = $
            \begin{bmatrix}
            0
            \end{bmatrix}
        \end{center}

        Macierz niepewności pomiarowej przyjęto eksperymentalnie jak poniżej. Przyjęto wstępnie niepewność pomiarową jako 7 cm dla wartości x oraz y. 

        \begin{center}
            $\bm{R}_n = $
            \begin{bmatrix}
            0.005 & 0 \\
            0 & 0.005
            \end{bmatrix}
        \end{center}

        Macierz szumu procesowego również dobrano doświadczalnie. Szum procesowy można początkowo przyjąć jako 10 mm dla wartości x oraz y.

        \begin{center}
            $\bm{Q} = $
            \begin{bmatrix}
            0.0001 & 0 \\
            0 & 0.0001
            \end{bmatrix}
        \end{center}
        
        \singlesizedimageforced{images/x_filtered.png}{Wykres pozycji robota przy zastosowaniu filtru Kalmana}{0.8}

        \singlesizedimageforced{images/x_move_filtered_0_0001.png}{Wykres pozycji robota podczas jego ruchu przy zastosowaniu filtru Kalmana}{0.8}

        Patrząc na wykres z rysunku \ref{Wykres pozycji robota przy zastosowaniu filtru Kalmana} można zauważyć, że przebieg danych odfiltrowanych został wygładzony względem sygnału oryginalnego. Jednak na wykresie obecnym na rysunku \ref{Wykres pozycji robota podczas jego ruchu przy zastosowaniu filtru Kalmana} można zauważyć, że podczas przemieszczania się wyjście filtru nie nadąża za sygnałem oryginalnym. Jest~to~spowodowane zbyt niską wartością szumu procesowego, jaka została przyjęta. By zmniejszyć opóźnienie, należy zwiększyć wartość elementów macierzy szumu procesowego. Przyjęto teraz szum procesowy równy wartości około 3 mm dla wielkości x oraz y.

        \begin{center}
            $\bm{Q} = $
            \begin{bmatrix}
            0.0008 & 0 \\
            0 & 0.0008
            \end{bmatrix}
        \end{center}

        \singlesizedimageforced{images/x_move_filtered_0_0008.png}{Wykres pozycji robota podczas jego ruchu przy zastosowaniu filtru Kalmana bez opóźnienia w nadążaniu za sygnałem oryginalnym}{0.8}

        Patrząc na wyniki z wykresu \ref{Wykres pozycji robota podczas jego ruchu przy zastosowaniu filtru Kalmana bez
        opóźnienia w nadążaniu za sygnałem oryginalnym} można zauważyć, że filtr szybciej nadąża za sygnałem oryginalnym niż w przypadku pokazanym na rysunku \ref{Wykres pozycji robota podczas jego ruchu przy zastosowaniu filtru Kalmana}. Zwiększenie szybkości nadążania jest obarczone pogorszeniem jakości filtracji, co jest możliwe do zaakceptowania.

        Warto się przyjrzećm, jak dołożenie pomiarów z odometrii wpłynie na dokładność uzyskanych pomiarów położenia robota.
        Macierze stałych są tożsame z macierzami użytymi w obliczeniach dotyczących GPS.
        Cały algorytm został wzbogacony o dodatkowy etap aktualizacji pomiarów związanym~z~odometrią. Czyli ostatecznie zastosowane tutaj zostały dwa etapy aktualizacji pomiarów, jak pokazano na rysunku \ref{Schemat algorytmu fuzji odometrii i wskazań GPS}.

        \singlesizedimage{images/kalman_matrix_odom_gps.png}{Schemat algorytmu fuzji odometrii i wskazań GPS}{0.25}

        \singlesizedimage{images/gps_odom.png}{Wyniki fuzji odometrii i wskazań GPS}{0.7}

        Odwołując się do wykresów \ref{Wykres pozycji robota podczas jego ruchu przy zastosowaniu filtru Kalmana bez opóźnienia w nadążaniu za sygnałem oryginalnym} oraz \ref{Wyniki fuzji odometrii i wskazań GPS} należy zauważyć, że jakość filtracji jest podobna. Dlatego nie istnieje potrzeba, by do oprogramowania autopilota wyposażać w drugi etap aktualizacji danych.
    }
}