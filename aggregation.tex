\newpage

\section{Metody transformacji danych z wybranych czujników}\label{transformation_methods}
{
    \subsection{Transformacja układu współrzędnych geograficznych do kartezjańskich}
    {
        \subsubsection{Opis współrzędnych geograficznych}
        {
            Współrzędne te są wyrażone w postaci dwóch liczb, szerokości (ang. lattitude) oraz długości (ang. longitude) geograficznych. Obie te współrzędne wyraża się miarą kątową (stopnie) od środka układu współrzędnych geograficznych. Początkiem układu współrzędnych geograficznych jest przecięcie się południka zerowego z równikiem.
            
            Współrzędne te związane są następującymi symbolami:
            \begin{itemize}
                \item $\phi$ --  szerokość geograficzna,
                \item $\lambda$ --  długość geograficzna.
            \end{itemize}
            
            Orientacja środka układu współrzędnych geograficznych wygląda następująco. Wartości dodatnie dla szerokości geograficznej znajdują się na północy, a dla długości geograficznej na wschodzie od środka układu. Ujemne natomiast na południe i zachód od środka układu.
            
            Szerokość geograficzna jest kątem pomiędzy półprostą poprowadzoną od środka kuli ziemskiej, która przechodzi przez punkt na jej powierzchni, a płaszczyzną równika. Zawiera się w przedziale od $-90^{\circ}$ do $90^{\circ}$.
            Długość geograficzna jest kątem pomiędzy półprostą poprowadzoną od środka kuli ziemskiej, która przechodzi przez punkt na jej powierzchni, a płaszczyzną południka zerowego. Zawiera się w przedziale od $-180^{\circ}$ do $180^{\circ}$.
        }
        \label{gps_to_xy_transform}
        \subsubsection{Opis algorytmu transformacji}
        {
            Współrzędne geograficzne są wygodne przy opisywaniu położenia danego miejsca na globie. Jednakże często istnieje potrzeba skorzystania ze współrzędnych kartezjańskich $(x, y)$, by dany algorytm mógł wykonywać obliczenia na wektorach i wyznaczać trajektorie. Wykorzystamy tutaj algorytm konwersji zwany \textit{equirectangular projection} \cite{geo_cords}. 

            Metoda ta polega na przybliżeniu pewnego fragmentu terenu za pomocą płaszczyzny, gdzie błąd przeliczenia współrzędnych geograficznych na kartezjańskie (wynikający z tego, że Ziemia jest okrągła) będzie znikomy. 

            Transformację współrzędnych geograficznych na kartezjańskie wyraża się wzorami \ref{geo_x_eqtn} oraz \ref{geo_y_eqtn}. Współrzędne po konwersji wyrażone są w metrach \cite{geo_cords}. Długość geograficzna referencyjna jest taką, gdzie skala transformacji będzie~1:1. 
                
            \begin{equation}\label{geo_x_eqtn}
                x=R(\lambda-\lambda_0)cos(\phi_1)
            \end{equation}
            \begin{equation}\label{geo_y_eqtn}
                y=R(\phi - \phi_0)
            \end{equation}
            gdzie,\\
            \textbf{$\lambda$} -- długość geograficzna danej pozycji,\\
            \textbf{$\phi$} -- szerokość geograficzna danej pozycji,\\
            \textbf{$\lambda_0$} -- długość geograficzna wybranego środka układu współrzędnych,\\
             \textbf{$\phi_0$} -- szerokość geograficzna wybranego środka układu współrzędnych,\\
             \textbf{$\phi_1$} -- długość geograficzna referencyjna, \\
             \textbf{$R$} -- promień Ziemi wyrażony w metrach\\
        }
    }
    \label{mag_to_yaw_transform}
    \subsection{Transformacja danych z magnetometru na orientację obiektu}
    {
        Dane z magnetometru dostarczają nam podstawowe wskazania na temat aktualnej orientacji danego obiektu. By uzyskać pomiary aktualnej orientacji wyrażonej w radianach, to należy~się~posłużyć do tego wzorami \ref{mag_eqtns} \cite{magnetometer}:
        
        \begin{equation}\label{mag_eqtns}
            \phi_R=\begin{cases}
			y>0, & \text{$\frac{\pi}{2} - \arctan(\frac{H_x}{H_y})$}\\
            y<0, & \text{$\frac{3\pi}{4} - \arctan(\frac{H_x}{H_y})$}
    		\end{cases}
        \end{equation}
        gdzie,\\
        $\phi_R$ -- orientacja obiektu,\\
        $H_y$ -- składowa y wektora pola magnetycznego,\\
        $H_x$ -- składowa x wektora pola magnetycznego\\
    }
    
    \subsection{Odometria}
    {
        Znajomość pozycji pojazdu ma fundamentalne znaczenie dla jego prawidłowego działania. Odometria jest metodą na obliczenie dystansu przebytego przez pojazdu, względem jego pozycji początkowej mierząc obroty kół dzięki zastosowaniu enkoderów obrotowych \cite{odometry}. Jest to bardzo prosta metoda na poznanie aktualnej pozycji pojazdu, lecz jest ona zwykle obarczona błędem ze względu na dryft pozycji spowodowany poślizgiem kół \cite{odom_2}. Błędy odometrii akumulują się z czasem, zatem cechuje się odpowiednią dokładnością tylko przez pewien czas jazdy pojazdu. Na rynku obecne są zastosowania odometrii kołowej oraz odometrii wizualnej VO \cite{odom_2}. Odometria kołowa jest najprostszą oraz najbardziej rozpowszechnioną metodą na mierzenie dystansu pojazdu względem jego pozycji początkowej. Odometria wizualna pozwala na estymację pozycji dzięki bieżącej analizie tylko i wyłącznie strumienia zdjęć uzyskanego z jednej lub wielu kamer \cite{odom_2}. Idea polega na badaniu przemieszczenia się poszczególnych pikseli pomiędzy dwoma klatkami i na tej podstawie może być wyznaczona przebyta odległość. Metoda ta jest dobrym balansem pomiędzy dokładnością estymacji pozycji a ceną \cite{odom_2}. Odometria wizualna jest odporna na poślizgi kół oraz cechy przebywanego terenu. Dzięki zastosowaniu odometrii jesteśmy w stanie określić aktualną liniową oraz kątową prędkość pojazdu, a po scałkowaniu, jego pozycję i orientację. Wszystkie metody odometrii znajdą zastosowanie w pomieszczeniach oraz innych zamkniętych przestrzeniach, gdzie przykładowo dostęp do sygnału GPS jest niemożliwy. Odczyty z odometrii mogą być zintegrowane przykładowo~z~odczytami z GPS po to, aby zwiększyć precyzję pomiarów. Odometria stosowana jest głównie w branży robotycznej oraz automotive.
        Poniższe wzory opisują jak transformować dane z odometrii kołowej do współrzędnych \textit{(x,y)}.

        \begin{equation}
            x_L = \frac{n_L}{N} 
        \end{equation}
        \begin{equation}
            x_R = \frac{n_R}{N}
        \end{equation}
        \begin{equation}
            d = \frac{x_L + x_R}{2}
        \end{equation}
        \begin{equation}
            d\theta = \arcsin{\frac{x_R - x_L}{L} }
        \end{equation}
        \begin{equation}
            dx = \cos{d\theta} \cdot d
        \end{equation}
        \begin{equation}
            dy = \sin{d\theta} \cdot d
        \end{equation}
        gdzie,\\
        $N$ -- rozdzielczość enkoderów $[\frac{impuls}{m}]$,\\
        $L$ -- odległość między środkiem prawej i lewej opony${m}$\\
        $n_L$ -- liczba impulsów zebrana przez enkoder lewego koła w czasie $dt$,\\
        $n_R$ -- liczba impulsów zebrana przez enkoder prawego koła w czasie $dt$,\\
        $x_L$ -- dystans przebyty przez lewe koło w czasie $dt$,\\
        $x_R$ -- dystans przebyty przez prawe koło w czasie $dt$,\\
        $d$ -- dystans przebyty przez pojazdu w czasie $dt$,\\
        $d\theta$ -- kąt przebyty przez pojazdu w czasie $dt$,\\
        $dx$ -- rzut dystansu przebytego przez pojazdu na oś x w czasie $dt$,\\
        $d\theta$ -- rzut dystansu przebytego przez pojazdu na oś y w czasie  $dt$\\
    }
    \subsection{Określanie kątów nachylenia obiektu za pomocą wskazań akcelerometru}
    {
        W tym podrozdziale opisano sposób wyznaczania kątów obrotu obiektu między osiami układu zerowego oraz wtórnego. Do tego celu można wykorzystać wskazania akcelerometru, które dostarczają informacje na temat składowych przyspieszenia liniowego działającego na obiekt. Kąt pomiędzy osiami x wspomnianych układów oznaczono jako $\theta$. Kąt pomiędzy osiami y oznaczono jako $\Phi$. Znajomość kątów $\theta$ oraz $\Phi$ jest szczególnie istotna, dla systemów pokładowych maszyn latających (np. drony). By uzyskać informacje o wartości wspomnianych kątów, należy wykorzystać informacje o wektorze przyspieszenia liniowego działającego na obiekt \cite{rp_acc}. Wzory \ref{pitch_eqtn} oraz \ref{roll_eqtn} pozwalają na wyznaczenie wspomnianych kątów \cite{rp_acc}. Na rysunku \ref{Wizualizacja kątów nachylenia} przedstawiono wizualizację kątów $\Phi$ oraz $\theta$.

        \begin{equation}\label{pitch_eqtn}
            \Phi = \arctan( a_y / a_z)
        \end{equation}
        \begin{equation}\label{roll_eqtn}
            \theta = \arcsin( a_x / g)
        \end{equation}
        gdzie,\\
        $a_x,a_y,a_z$ -- składowe wektora przyspieszenia,\\
        $g$ -- przyspieszenie grawitacyjne
        \singlesizedimageforced{images/acc_rpy.png}{Wizualizacja kątów nachylenia}{0.3}
        
    }
}